\documentclass[12pt, a4paper]{article}

\usepackage[english]{babel}
% Here the text area, the left margin and the top margin are set.
% The right and bottom margins are automatically computed to fit the page.
\usepackage{geometry}
\geometry{
    a4paper,    % 210 x 297 mm
    total={170mm,257mm},
    left=20mm,  % 210 - 170 = 40, so right margin will also be 20mm ?
    top=15mm,   % 297 - 257 = 40, so bottom margin will be 25mm ?
}
\usepackage{amssymb}
\usepackage{amsmath}
\usepackage{graphicx}
\usepackage{float}
\usepackage{hyperref}
\hypersetup{pdfborder=0 0 0, colorlinks=true, linkcolor=blue, urlcolor=blue, citecolor=blue}
\usepackage{todonotes}
\usepackage{enumerate}
\usepackage{afterpage}    % Hack to remove page number from one specific page
\usepackage{transparent}  % Use transparent image to move float to right of two-col figure

\usepackage{natbib}
\usepackage{aas_macros}
\citestyle{aa}
\bibpunct{(}{)}{;}{a}{,}{,}

\graphicspath{{../img/}}   % Set image path

\renewcommand{\thefootnote}{\fnsymbol{footnote}}
\renewcommand*{\thesection}{4.\arabic{section}}
\setcounter{section}{-1}

\makeatletter
\renewcommand{\fnum@figure}{Fig. \thefigure}
\makeatother

\begin{document}
\part*{Sec. 4 \quad Results}
\textbf{Question.} Is the star formation model implemented for the Auriga simulations
capable of producing a population of star particles that is consistent with the
observed properties of the Milky Way (MW) globular cluster system (GCS)?
\begin{itemize}
    \item What data do we have available for MW and M31? \\
        \null\qquad $\rightarrow$ Sec.~\ref{sec:data} on p.~\pageref{sec:data}
    \item Does Auriga's star formation model produce sufficient star particles
        with the right age and metallicity ([Fe/H]) to be consistent with the
        MW GCS? \\
        \null\qquad $\rightarrow$ Sec.~\ref{sec:FeH} on p.~\pageref{sec:FeH}
    \item Does Auriga's star formation model produce sufficient star particles
        with the right age and radial distribution to be consistent with the
        MW GCS? \\
        \null\qquad $\rightarrow$ Sec.~\ref{sec:Rgc} on p.~\pageref{sec:Rgc}
    \item Does Auriga's star formation model produce sufficient star particles
        with the right age, metallicity ([Fe/H]) and radial distribution to be
        consistent with the MW GCS? \\
        \null\qquad $\rightarrow$ Sec.~\ref{sec:FeH-Rgc} on p.~\pageref{sec:FeH-Rgc}
\end{itemize}
\textbf{Question.} What does the picture look like when broaden the scope to
consider spirals in the Local Group (i.e. including Andromeda/M31)? \\

\noindent \textbf{Assumption I.} All star particles in the Auriga simulations with age
$> 10$~Gyr are globular cluster candidates. \\

\textbf{Justification}
\begin{enumerate}[i)]
    {\setlength\itemindent{25pt} \item
        \citet{2013ApJ...775..134V} measured [Fe/H] of 55 globular clusters in
        the MW and obtained age-estimates. The mean age of the MW GCS is 11.9~Gyr
        with a dispersion of 0.9~Gyr. Furthermore, only one of the 55 GC
        age-estimates is below 10~Gyr.
    }
    {\setlength\itemindent{25pt} \item
        \citet{2017MNRAS.465.3622R} performed one simulation of a MW-like galaxy
        (down to z=0.5, T$_{\rm lookback}\approx$~5~Gyr) and performs the entire
        analysis using a subset of star particles with ages $>10$~Gyr that is
        referred to as `globular cluster candidates'.
    }
\end{enumerate}

\textbf{Weaknesses}
\begin{enumerate}[i)]
    {\setlength\itemindent{25pt} \item
        \citet{2018MNRAS.475.4309P} show that star clusters do not simply follow
        the same distribution as the field stars. In our approach we don't just
        oversample the `real' population of globular clusters (by ignoring/excluding
        dynamical evolution), but the retrieved distributions of old star particles in
        the simulations may not even faithfully represent the `true' distribution of
        globulars.
    }
    {\setlength\itemindent{25pt} \item
        \citet{2011AJ....141...61C} measured [Fe/H] of 87 globular clusters
        in M31 and obtained age-estimates. The mean age is 10.8~Gyr with a
        dispersion of 2.3~Gyr. Furthermore, 27 GCs have age-estimates below 10~Gyr,
        with a minimum age-estimate of 4.8~Gyr. Perhaps an age cut of 6~Gyr
        would would be more appropriate for M31, see Fig.~\ref{fig:MW-M31-age}.
    }
\end{enumerate}

\afterpage{%
\section{Available data for the Milky Way and Andromeda}
\label{sec:data}
\thispagestyle{empty}
\begin{figure*}[h!]
    \includegraphics[width=0.5\textwidth]{{MW_M31_Age_HistogramMassWeighted}.png}
    \includegraphics[width=0.49\textwidth]{{MW_M31_FeH_HistogramMassWeighted}.png}
    \caption{
        \emph{Left:} Mass-weighted age distribution of 55 GCs in the MW
        \citep[data from][]{2013ApJ...775..134V} and 87 GCs
        in M31 \citep[data from][]{2011AJ....141...61C}.
        \emph{Right:} Mass-weighted [Fe/H] distribution of 151 GCs in the MW
        \citep[data from][2010 ed.]{1996AJ....112.1487H} and 314 GCs in M31.
        \label{fig:MW-M31-ageFeH}
    }

    \includegraphics[width=0.49\textwidth]{{MW_M31_Rgc_HistogramMassWeighted}.png}
    \includegraphics[width=0.53\textwidth]
        {{MW_M31_RgcFeH_HistogramMassWeighted_Harris1996ed2010data}.png}
    \caption{
        \emph{Left:} Mass-weighted R$_{\rm GC}$ distribution of 156 GCs in the MW.
        \emph{Right:} Mass-weighted [Fe/H]-R$_{\rm GC}$ distribution of
        151 GCs in the MW \citep[data from][2010 ed.]{1996AJ....112.1487H}, which
        is 98.19 \% of the total MW GCS mass.
        \label{fig:MW-M31-Rgc-FeHRgc}
    }

    \transparent{0.0}\includegraphics[width=0.49\textwidth]
        {{MW_M31_Rgc_HistogramMassWeighted}.png}
    \transparent{1.0}\includegraphics[width=0.53\textwidth]
        {{MW_M31_RgcFeH_HistogramMassWeighted_VandenBerg2013data}.png}
    \caption{
        Mass-weighted [Fe/H]-R$_{\rm GC}$ distribution of
        55 GCs in the Milky Way \citep[data from][]{2013ApJ...775..134V}, which
        is 43.27\% of the total MW GCS mass.
        \label{fig:MW-M31-FeHRgc2}
    }
\end{figure*}
\clearpage
}
\newpage

\afterpage{%
\newgeometry{
    a4paper,  % 210 x 297 mm
    total={170mm,257mm},
    left=20mm,  % 210 - 170 = 40, so right margin will also be 20mm ?
    top=5mm,   % 297 - 257 = 40, so bottom margin will be 25mm ?
}
\begin{table}[ht!]
    \centering
    \caption{
        The [Fe/H]-R$_{\rm GC}$ plot is generated using
        \textsc{scipy.stats.binned\_statistic\_2d}. This table shows a manual
        calculation of Msum and Ngc in each bin for debug purposes, just to check
        that \textsc{scipy}'s built-in method is used correctly.
        Data from \citet[][2010 ed.]{1996AJ....112.1487H}.
        \label{tab:MW-FeH-Rgc}
    }
    \begin{tabular}{rrrrrrc}
        \hline
        xmin   & xmax   & ymin   & ymax   & Ngc    & Msum   & log$_{10}$(Msun) \\
        \hline
        1      & 3      & -2.5   & -1.5   & 7      & 1.7e+06 & 6.24       \\
        1      & 3      & -1.5   & -1.0   & 15     & 1.9e+06 & 6.28       \\
        1      & 3      & -1.0   & -0.5   & 10     & 8.2e+05 & 5.91       \\
        1      & 3      & -0.5   & -0.3   & 4      & 1.0e+06 & 6.02       \\
        1      & 3      & -0.3   & 0      & 2      & 3.2e+05 & 5.51       \\
        \hline
        3      & 8      & -2.5   & -1.5   & 16     & 4.6e+06 & 6.66       \\
        3      & 8      & -1.5   & -1.0   & 15     & 2.8e+06 & 6.44       \\
        3      & 8      & -1.0   & -0.5   & 11     & 2.8e+06 & 6.44       \\
        3      & 8      & -0.5   & -0.3   & 7      & 1.9e+06 & 6.29       \\
        3      & 8      & -0.3   & 0      & 1      & 3.0e+04 & 4.48       \\
        \hline
        8      & 15     & -2.5   & -1.5   & 11     & 3.0e+06 & 6.47       \\
        8      & 15     & -1.5   & -1.0   & 7      & 2.4e+06 & 6.38       \\
        8      & 15     & -1.0   & -0.5   & 1      & 6.5e+03 & 3.81       \\
        8      & 15     & -0.5   & -0.3   & 1      & 8.5e+04 & 4.93       \\
        8      & 15     & -0.3   & 0      & 0      & 0.0e+00 & -inf       \\
        \hline
        15     & 30     & -2.5   & -1.5   & 16     & 2.1e+06 & 6.31       \\
        15     & 30     & -1.5   & -1.0   & 6      & 2.2e+06 & 6.34       \\
        15     & 30     & -1.0   & -0.5   & 2      & 1.0e+04 & 4.02       \\
        15     & 30     & -0.5   & -0.3   & 1      & 1.5e+04 & 4.17       \\
        15     & 30     & -0.3   & 0      & 0      & 0.0e+00 & -inf       \\
        \hline
        30     & 125    & -2.5   & -1.5   & 6      & 1.1e+06 & 6.04       \\
        30     & 125    & -1.5   & -1.0   & 4      & 3.1e+05 & 5.49       \\
        30     & 125    & -1.0   & -0.5   & 1      & 1.4e+03 & 3.15       \\
        30     & 125    & -0.5   & -0.3   & 0      & 0.0e+00 & -inf       \\
        30     & 125    & -0.3   & 0      & 0      & 0.0e+00 & -inf       \\
        \hline
        \multicolumn{7}{c}{Data from \citet[][55 GCs]{2013ApJ...775..134V}} \\
        \hline
        xmin   & xmax   & ymin   & ymax   & Ngc    & Msum   & log$_{10}$(Msun) \\
        \hline
        1      & 3      & -2.5   & -1.5   & 3      & 5.5e+05 & 5.74       \\
        1      & 3      & -1.5   & -1.0   & 2      & 2.2e+05 & 5.35       \\
        1      & 3      & -1.0   & -0.5   & 2      & 2.3e+05 & 5.37       \\
        1      & 3      & -0.5   & -0.3   & 2      & 2.6e+05 & 5.42       \\
        1      & 3      & -0.3   & 0      & 0      & 0.0e+00 & -inf       \\
        \hline
        3      & 8      & -2.5   & -1.5   & 9      & 1.7e+06 & 6.22       \\
        3      & 8      & -1.5   & -1.0   & 6      & 1.1e+06 & 6.03       \\
        3      & 8      & -1.0   & -0.5   & 4      & 9.6e+05 & 5.98       \\
        3      & 8      & -0.5   & -0.3   & 1      & 1.1e+05 & 5.04       \\
        3      & 8      & -0.3   & 0      & 1      & 1.9e+05 & 5.29       \\
        \hline
        8      & 15     & -2.5   & -1.5   & 10     & 3.0e+06 & 6.48       \\
        8      & 15     & -1.5   & -1.0   & 5      & 1.9e+06 & 6.27       \\
        8      & 15     & -1.0   & -0.5   & 0      & 0.0e+00 & -inf       \\
        8      & 15     & -0.5   & -0.3   & 0      & 0.0e+00 & -inf       \\
        8      & 15     & -0.3   & 0      & 0      & 0.0e+00 & -inf       \\
        \hline
        15     & 30     & -2.5   & -1.5   & 6      & 6.8e+05 & 5.83       \\
        15     & 30     & -1.5   & -1.0   & 3      & 1.9e+06 & 6.29       \\
        15     & 30     & -1.0   & -0.5   & 1      & 8.9e+03 & 3.95       \\
        15     & 30     & -0.5   & -0.3   & 0      & 0.0e+00 & -inf       \\
        15     & 30     & -0.3   & 0      & 0      & 0.0e+00 & -inf       \\
        \hline
        30     & 125    & -2.5   & -1.5   & 0      & 0.0e+00 & -inf       \\
        30     & 125    & -1.5   & -1.0   & 0      & 0.0e+00 & -inf       \\
        30     & 125    & -1.0   & -0.5   & 0      & 0.0e+00 & -inf       \\
        30     & 125    & -0.5   & -0.3   & 0      & 0.0e+00 & -inf       \\
        30     & 125    & -0.3   & 0      & 0      & 0.0e+00 & -inf       \\
        \hline
    \end{tabular}
\end{table}
\clearpage
\restoregeometry
}
\clearpage

\afterpage{%
\section{Distribution of metallicity [Fe/H]}
\label{sec:FeH}
\thispagestyle{empty}
\begin{figure*}[h!]
    \includegraphics[width=0.49\textwidth]{{logMFeH_OldStars_plot-all}.png}
    \includegraphics[width=0.49\textwidth]{{logMFeH_OldStars_normalization_fit0}.png}
    \caption{
        \scriptsize \emph{Left:} Mass-weighted [Fe/H]-R$_{\rm GC}$ distribution of 
        all Auriga haloes (level 3, 4 and 5).
        \emph{Right:} There seems to be a trent of constant mass increase with
        increasing metallicity in the range -3 to -1, and roughly 1 order of 
        magnitude scatter in mass normalization. We use linear regression to 
        obtain log$_{10}$(mass normalization) (i.e. we fit $a\cdot(x+3) + b$)
        \label{fig:todo1}
    }

    \includegraphics[width=0.49\textwidth]{{logMFeH_OldStars_normalization_fit1}.png}
    \includegraphics[width=0.49\textwidth]{{logMFeH_OldStars_normalization_fit3}.png}
    \caption{
        \scriptsize \emph{Left:} Scatter plot of log$_{10}$(mass normalization)
        [i.e. $b$ obtained above] against the virial mass M$_{200,c}$ of the 
        Auriga haloes. We use linear regression to see whether there is a 
        correlation, and find $a_2 = 0.00388$, $b_2 = 6.597$. Colours indicate 
        resolution: L3 red, L4 green, and L5 blue.
        \emph{Right:} Same `data', but MCMC fit. Red dotted line shows 
        `central' x value; yellow region 1$\sigma$ interval; blue dashed lines 
        intrinsic scatter. Fake Error bars (1\% of the obtained values).
        \label{fig:todo2}
    }

    { \centering
    \includegraphics[width=0.4\textwidth]{{logMFeH_OldStars_normalization_fit2}.png}
    \caption{
        \scriptsize TODO: corner plot of the MCMC. Conclusion: there is a positive 
        correlation of the normalization of old stellar mass with the virial
        mass $M_{200,c}.$
        \label{fig:todo3}
    }
    }
\end{figure*}
\clearpage
}
\clearpage

\begin{figure*}[h!]
    \includegraphics[width=0.49\textwidth]{{logMFeH_OldStars_plot-errorbar}.png}
    \includegraphics[width=0.49\textwidth]{{logMFeH_OldStars_plot-hist2d}.png}
    \caption{
        % Au3-6, 16, 21, 23, 24, 27; Au4-1-30; Au5-6, 9 16 24
        \emph{Left:} Mean value of all 40 simulations in orange.
        \emph{Right:} Colorbar shows the number of simulations that fall in
        (arbitrary) bins in y-direction.
        \label{fig:todo4}
    }

    \includegraphics[width=0.49\textwidth]
        {{logMFeH_OldAccreted_plot-hist2d-with-mean}.png}
    \includegraphics[width=0.49\textwidth]
        {{logMFeH_OldInsitu_plot-hist2d-with-mean}.png}
    \caption{
        \emph{Left:} Mass-weighted [Fe/H] distribution of 
        \textbf{Old accreted} (age $>$~10~Gyr) star particles.
        Solid orange line shows the mean of all old star particles in all Auriga
        simulation. The dotted orange line shows the mean of all star particles
        in all Auriga simulations (i.e. not enforcing an age cut).
        \emph{Right:} Mass-weighted [Fe/H] distribution of \textbf{Old insitu} 
        (age $>$~10~Gyr) star particles. 
        \emph{Both / Conclusion:} The metallicity range $-3 <$ [Fe/H] $< -1$ is
        only populated star particles older than ten Gyr; [Fe/H] $> -1$ is dominated
        by star particles younger than 10~Gyr (two orders of magnitude difference). 
        Furthermore, the old accreted star particles contribute most significantly 
        to the range $-3 <$ [Fe/H] $< -1$, and the contribution of the old insitu
        star particles at these metallicities declines steeper with declining
        metallicity than the old accreted population. The old insitu population
        provides the dominant contribution of the old population for 
        [Fe/H] $> -1$.
        \label{fig:todo5}
    }
\end{figure*}
\clearpage

\begin{figure*}[h!]
    \includegraphics[width=\textwidth]{{logMFeH}.png}
    \caption{
        This figure summarises above findings. TODO: write caption for 
        paper.
        \label{fig:todo6}
    }
\end{figure*}


\clearpage


\section{Distribution of Galactocentric radii R$_{\rm GC}$}
\label{sec:Rgc}


\section{Distribution of [Fe/H]-R$_{\rm GC}$}
\label{sec:FeH-Rgc}

\bibliographystyle{../mnras}
\bibliography{../AurigaGCS}


\end{document}
