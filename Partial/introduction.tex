\documentclass[a4paper,fleqn,usenatbib]{mnras}
\usepackage{newtxtext,newtxmath}
\usepackage[T1]{fontenc}
\usepackage{ae,aecompl}
\usepackage{graphicx} 
\usepackage{amsmath} 
\usepackage{amssymb}
\graphicspath{{img/}} 
\newcommand{\Sun}[0]{\ensuremath{_{\odot}}}
\renewcommand{\deg}{\ensuremath{^{\circ}}}

\usepackage[most]{tcolorbox} 
\definecolor{block-gray}{gray}{0.95}
\newtcolorbox{zitat}[2][]{%
    colback=block-gray,
    grow to right by=-10mm,
    grow to left by=-10mm, 
    boxrule=0pt,
    boxsep=0pt,
    breakable,
    enhanced jigsaw,
    borderline west={4pt}{0pt}{gray},
    title={#2\par},
    colbacktitle={block-gray},
    coltitle={black},
    fonttitle={\large\bfseries},
    attach title to upper={},
    #1,
}

\title[Auriga GCS]{The Globular Cluster System of the Auriga Simulations}

\author[T. L. R. Halbesma et al.]{\parbox[t]{\textwidth}{
    Timo L. R. Halbesma$^{1}$\thanks{E-mail: Halbesma@MPA-Garching.MPG.DE},
    Wilma Trick$^{1}$,
    Robert J. J. Grand$^{1}$, 
    Volker Springel$^{1}$, 
    Facundo A. G\'{o}mez$^{2,3}$, 
    Federico Marinacci$^{4,5}$,
    R\"{u}diger Pakmor$^{1}$, 
    Simon D. M. White$^{1}$
} \vspace{10pt} \\
$^{1}$ Max-Planck-Institut f\"ur Astrophysik, Postfach 1317, D-85741 Garching, Germany \\
$^{2}$ Instituto de Investigaci\'{o}n Multidisciplinar en Ciencia yTecnolog\'{i}a, Universidad de La Serena, Ra\'{u}l Bitr\'{a}n 1305, La Serena, Chile \\
$^{3}$ Departamento de F\'{i}sica y Astronom\'{i}a, Universidad de La Serena, Av. Juan Cisternas 1200 N, La Serena, Chile \\
$^{4}$ Department of Physics, Kavli Institute for Astrophysics and Space Research, MIT, Cambridge, MA 02139, USA \\
$^{5}$ Harvard-Smithsonian Center for Astrophysics, 60 Garden Street, Cambridge, MA 02138, USA \\
}

% These dates will be filled out by the publisher
\date{Accepted XXX. Received YYY; in original form ZZZ}

% Enter the current year, for the copyright statements etc.
\pubyear{2019}

% Don't change these lines
\begin{document}
\label{firstpage}
\pagerange{\pageref{firstpage}--\pageref{lastpage}}
\maketitle

% Abstract of the paper
\begin{abstract}
We investigate whether the galaxy formation model used for the Auriga simulations 
can produce a realistic globular cluster population at redshift zero. We compare
properties of the simulated star particles in the Auriga haloes with
catalogues of observations of the Milky Way globular cluster population available
in the literature. We find that the Auriga simulations produce sufficient mass
at radii and metallicities that are typical for the MW GCS, although we observe
a varying mass-excess for the different $R_{\text{GC}}$-[Fe/H] bins. This implies
different values for the combined product of the bound cluster formation efficiency
and the globular cluster disruption rate. We investigate wether these differences
could result from formation insitu vs. accreted star particles. We find ...
\end{abstract}

% Select between one and six entries from the list of approved keywords.
% Don't make up new ones.
\begin{keywords}
methods: numerical -- galaxies: formation -- galaxies: star clusters: general.
\end{keywords}

%%%%%%%%%%%%%%%%%%%%%%%%%%%%%%%%%%%%%%%%%%%%%%%%%%

%%%%%%%%%%%%%%%%% BODY OF PAPER %%%%%%%%%%%%%%%%%%

\section{Introduction}

\subsection{GC form under special circumstances in high-z Universe}
\begin{zitat}{\citet{2017MNRAS.465.3622R}}
GCs are among the oldest astrophysical objects. They form in the early Universe in
highest density peaks \citep[e.g.][]{2005MNRAS.364..367D, 2009ApJ...706L.192B}
\end{zitat}

\begin{zitat}{\citet{2005MNRAS.364..367D}}
The radial profile of the stellar halo and metal-poor globular
clusters of the Milky Way suggest that these components formed in rare
early peaks above $2.5 \sigma$ at redshift above 10.
\end{zitat}


\begin{zitat}{\citet{2005MNRAS.364..367D}}
The  clustering  properties  of  metal-poor  globular  clusters  contain
clues on their formation sites and the epoch when star formation
was suppressed by feedback processes (e.g. reionization, supernova-
driven winds) in low-mass haloes. In the Milky Way metal-poor
globular clusters follow the same radial profile as halo stars, suggesting
within the framework of our model a common origin within
early $2.5\sigma$ peak progenitors at $z\approx12$. Observations of the
distribution of globular clusters within present-day haloes of different
masses could provide information on feedback effects as a function
of environment (Moore et al. 2005). The suppression of globular
cluster formation at some early epoch may also explain the bimodality 
observed in cluster metallicities (e.g. Strader et al. 2005). The
widely  used  assumption  that  globular  cluster  formation  is  a  fair
tracer of star formation, combined with the suppression of the formation  
of  metal-poor  globulars  after  reionization,  imply  that  the
amount of high-$\sigma$ material in a halo is proportional to the number of
metal-poor globular clusters. From the results in Section 3.4 it follows 
then that a simple universal reionization epoch would lead to a
constant abundance of metal-poor globular clusters per virial mass.
Deviations from this simplest case may provide information about
the local reionization epoch, and whether regions with more (less)
metal-poor globulars per virial mass were reionized later (earlier)
(see Moore et al. 2005).
\end{zitat}

\subsection{YMC and GC formation governed by same physics; differences 
due to nearly a Hubble time of evolution}
``The widely  used  assumption  that  globular  cluster  formation  is  a  fair
tracer of star formation'' $\rightarrow$ disputed?!


\subsection{Understand GC formation \& evolution $\rightarrow$ 
understand MW formation \& evolution}
This seems to shift the problem from `why care about GC formation and evolution'
to `why care about MW formation and evolution'. People seems to accept this
form of what-about-ism / diversion of focus. Actually, this is expansion of focus
so probably people accept this as the bigger picture that we're all after.

\begin{zitat}{\citet{2017MNRAS.465.3622R}}
Hence, they witness most of the formation and evolution processes of galaxies, and can be used to probe them \citep{2006ARA&A..44..193B}
\end{zitat}

\subsection{Colour bimodality}
\begin{zitat}{\citet{2017MNRAS.465.3622R}}
colour bimodality, blue and red clusters \citep[e.g.][]{1985ApJ...293..424Z, 1999AJ....118.1526G, 2001AJ....121.2974L, 2006ApJ...639...95P}
\end{zitat}

\begin{zitat}{\citet{2017MNRAS.465.3622R}}
blue metal-poor (with distribution peaking at [Fe/H] $\approx -1.5$ for the Milky Way), no sign of rotation as a population (..) more metal-rich (peak at [Fe/H] $\approx -0.5$ in the Milky  Way) more spatially concentrated and rotating with the galaxy. \citep{1996AJ....112.1487H}
\end{zitat}





\bibliographystyle{mnras}
\bibliography{AurigaGCS} % if your bibtex file is called example.bib


\end{document}
