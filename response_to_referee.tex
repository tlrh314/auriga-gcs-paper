\documentclass{article}

\begin{document}

\section*{Introduction}
We thank the referee for a thoughtful and elaborate report that we feel greatly
improved the manuscript. We address the comments in sections with names that match
the headings of the referee report, starting with the following listing of our 
response to the 3 paragraphs in the introduction.
\begin{enumerate}
\item We updated the abstract and conclusion to clarify that we meant to describe
what would be needed for the environmental dependence of the formation and disruption
of bound clusters for the Auriga simulations to be consistent with the observations,
rather than to argue that the Auriga simulations reflect these trends.
\item
We feel that our conclusion would be too strongly phrased if we would generalise 
our results obtained using the Auriga simulations to cosmological simulations 
in general. We find that age-selected stellar particles in the Auriga simulations 
do not reproduce key properties of observed GC systems in the Milky Way and M31.
However, our results might reflect characteristics that are specific to Auriga, 
e.g. that the galaxy formation model overmixes metals at early times. 
Therefore we did add a final bulletpoint to our conclusion to clarify the above, 
but we limit the scope to the Auriga simulations. Thanks for pointing this out!

Furthermore, we clarified in the manuscript that we investigate the metallicity
and galactocentric radius because these properties are available for the observed
GC systems and in the simulations.
\item TODO: disussion of implications for wider context.
\end{enumerate}

\section*{General comments about style}
\begin{itemize}
\item Numerical quantities are now described in arabic numbers instead of words.
\item We now first use the name of the physical quantities in the given example,
    followed by the symbolic representation.
\item The text has been updated to reduce the usage of parenthesis.
\item Colloquial expressions have been rephrased.
\item We replaced `Sec.' by `Section' to refer to sections in the present work.
Moreover, we changed `Sec.' to `section' to refer to a section in one of the references.
\item All acronyms are introduced the first time they appear in the text
(MW, M31, GC, GCS, YMC, ISM, SN, AGB, DM, and Au). The subset thereof that appears 
in the abstract is also introduced the first time it appears.
\item The colour palette has been updated to improve colourblind-friendliness.
\item We no longer use blue and red to prevent confusion because they are 
typically used for metal-poor and metal-rich populations.
\end{itemize}


\section*{Major comments and concerns}
\begin{itemize}
\item TODO: add a paragraph in introduction to justify why we consider metallicity
and galactocentric radius.
\item TODO: justify age cut instead of using a more elaborate selection function.
\end{itemize}


\section*{Additional major comments}
\begin{itemize}
\item TODO justify the modelling philosophy.
\item TODO: clarify why observational data has different numbers
\item We normalise the Auriga simulations by the virial radius of the
dark matter halo to compensate for scatter between different simulation runs. We
deliberately chose for $r_{vir}$ because there are several stellar length scales
that could be used (e.g. the effective radius of the bulge, the radial scalelength
or vertical scaleheight of the disk, the optical radius, the extent of the stellar 
halo, etc). The simulations show a wide variety between properties of the Auriga
galaxies with little correlation between these different stellar length scales.
The mean virial radius of Auriga L4 is 299.67 $\pm$ 19.41 kpc.

\end{itemize}


\section*{Comments for each section}
\subsection*{Introduction}
\begin{itemize}
\item Item
\end{itemize}

\subsection*{Sect. 2 -}
\begin{itemize}
\item Item
\end{itemize}

\subsection*{Sect. 3 -}
\begin{itemize}
\item Item 
\end{itemize}

\subsection*{Sect. 4 -}
\begin{itemize}
\item
\item Figure 4 - now has a vertical line to indicate the metallicity cut 
between metal-poor and metal-rich GC subpopulations in the Milky Way.
\item
\item
\item Figure 6: no changes were made for rescaling the galaxies (i.e. we still
use the virial radius)
\item Figure 5 and 7: shaded regions now show the 25-75th percentiles.
\item Figure 9: The values are removed from the masked upper right corner bins.
\item Figure 2,8,9: The values of the upper and right axes have been removed from
Figure 2 and 9 (observations) for clarity (rather than the suggestion to add 
axis labels and units). However, we did not mask these bins for Figure 8 (Auriga)
because it illustrates our finding that the simulations produce a considerable
amount of mass in GC candidates with low metallicities at large radii.
\end{itemize}

\subsection*{Sect. 5 -}
\begin{itemize}
\item Item
\end{itemize}



\section*{Minor comments}
\begin{itemize}
\item We clarified that `blue' GC subpopulation means metal-poor with [Fe/H]~$<~1$
\item The units of mass-to-light ratio have been corrected.
\item The indicated typos have been corrected.
\item `missing error bars and references for the virial radii' $\rightarrow$
    The reference was given in a footnote, but is now moved to the main text.
\item 4.1: `which model do the authors refer to?' $\rightarrow$ 
    added `star formation'.
\item 4.1: `The top half of the left figure' $\rightarrow$ 
    `The top panel of Figure 3'.
\end{itemize}


\end{document}
