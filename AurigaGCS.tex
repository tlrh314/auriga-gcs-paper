% mnras_template.tex
%
% LaTeX template for creating an MNRAS paper
%
% v3.0 released 14 May 2015
% (version numbers match those of mnras.cls)
%
% Copyright (C) Royal Astronomical Society 2015
% Authors:
% Keith T. Smith (Royal Astronomical Society)

% Change log
%
% v3.0 May 2015
%    Renamed to match the new package name
%    Version number matches mnras.cls
%    A few minor tweaks to wording
% v1.0 September 2013
%    Beta testing only - never publicly released
%    First version: a simple (ish) template for creating an MNRAS paper

%%%%%%%%%%%%%%%%%%%%%%%%%%%%%%%%%%%%%%%%%%%%%%%%%%
% Basic setup. Most papers should leave these options alone.
\documentclass[a4paper,fleqn,usenatbib]{mnras}

% MNRAS is set in Times font. If you don't have this installed (most LaTeX
% installations will be fine) or prefer the old Computer Modern fonts, comment
% out the following line
\usepackage{newtxtext,newtxmath}
% Depending on your LaTeX fonts installation, you might get better results with one of these:
%\usepackage{mathptmx}
%\usepackage{txfonts}

% Use vector fonts, so it zooms properly in on-screen viewing software
% Don't change these lines unless you know what you are doing
\usepackage[T1]{fontenc}
\usepackage{ae,aecompl}


%%%%% AUTHORS - PLACE YOUR OWN PACKAGES HERE %%%%%

% Only include extra packages if you really need them. Common packages are:
\usepackage{graphicx}    % Including figure files
\usepackage{amsmath}    % Advanced maths commands
\usepackage{amssymb}    % Extra maths symbols

\usepackage{todonotes}
\graphicspath{{img/}}   % Set image path

%%%%%%%%%%%%%%%%%%%%%%%%%%%%%%%%%%%%%%%%%%%%%%%%%%

%%%%% AUTHORS - PLACE YOUR OWN COMMANDS HERE %%%%%

% Please keep new commands to a minimum, and use \newcommand not \def to avoid
% overwriting existing commands. Example:
%\newcommand{\pcm}{\,cm$^{-2}$}    % per cm-squared
\newcommand{\Sun}[0]{\ensuremath{_{\odot}}}
\renewcommand{\deg}{\ensuremath{^{\circ}}}


%%%%%%%%%%%%%%%%%%%%%%%%%%%%%%%%%%%%%%%%%%%%%%%%%%

%%%%%%%%%%%%%%%%%%% TITLE PAGE %%%%%%%%%%%%%%%%%%%

% Title of the paper, and the short title which is used in the headers.
% Keep the title short and informative.
\title[Auriga GCS]{The Globular Cluster System of the Auriga Simulations}

% The list of authors, and the short list which is used in the headers.
% If you need two or more lines of authors, add an extra line using \newauthor
\author[T. L. R. Halbesma et al.]{\parbox[t]{\textwidth}{
    Timo L. R. Halbesma$^{1}$\thanks{E-mail: Halbesma@MPA-Garching.MPG.DE},
    Robert J. J. Grand$^{1}$,
    Volker Springel$^{1}$,
    Facundo A. G\'{o}mez$^{2,3}$,
    Federico Marinacci$^{4,5}$,
    R\"{u}diger Pakmor$^{1}$,
    Wilma Trick$^{1}$,
    Philipp Busch$^{1}$,
    Simon D. M. White$^{1}$
} \vspace{10pt} \\
$^{1}$ Max-Planck-Institut f\"ur Astrophysik, Karl-Schwarzschild-Str. 1,
    85741 Garching, Germany \\
$^{2}$ Instituto de Investigaci\'{o}n Multidisciplinar en Ciencia yTecnolog\'{i}a,
    Universidad de La Serena, Ra\'{u}l Bitr\'{a}n 1305, La Serena, Chile \\
$^{3}$ Departamento de F\'{i}sica y Astronom\'{i}a, Universidad de La Serena, Av.
    Juan Cisternas 1200 N, La Serena, Chile \\
$^{4}$ Department of Physics, Kavli Institute for Astrophysics and Space Research,
    MIT, Cambridge, MA 02139, USA \\
$^{5}$ Harvard-Smithsonian Center for Astrophysics, 60 Garden Street, Cambridge,
    MA 02138, USA \\
}

% These dates will be filled out by the publisher
\date{Accepted XXX. Received YYY; in original form ZZZ}

% Enter the current year, for the copyright statements etc.
\pubyear{2019}

% Don't change these lines
\begin{document}
\label{firstpage}
\pagerange{\pageref{firstpage}--\pageref{lastpage}}
\maketitle

% Abstract of the paper
\begin{abstract}
We investigate whether the galaxy formation model used for the Auriga simulations
can produce a realistic globular cluster population at redshift zero. We compare
properties of the simulated star particles in the Auriga haloes with catalogues
of observations of the Milky Way and Andromeda globular cluster populations available
in the literature. We find that the Auriga simulations do produce sufficient mass
at radii and metallicities that are typical for the MW GCS, although we observe
a varying mass-excess for the different $R_{\text{GC}}$-[Fe/H] bins. This implies
different values for the combined product of the bound cluster formation efficiency
and the globular cluster disruption rate. Furthermore we test whether any of the
Auriga galaxies has a metallicity and radial distribution that is consistent with
the MW (M31) GCS. For any one of the Auriga haloes we reject the null hypothesis that
the simulated and observed metallicities are drawn from the same distribution at
$\geq98.32$~\% confidence level, for the GCS of the Milky Way as well as that of
the Andromeda galaxy. The same holds true for the distribution of galactocentric
radius. Overall, the Auriga simulations produce old star particles with higher
metallicities than the MW and M31 GCS and at larger radii. The formation efficiency
would have to linearly decrease with increasing metallicity for the Auriga GC
candidates to be consistent with the MW GCS.
\end{abstract}

% Select between one and six entries from the list of approved keywords.
% Don't make up new ones.
\begin{keywords}
methods: numerical -- galaxies: formation -- galaxies: star clusters: general.
\end{keywords}

%%%%%%%%%%%%%%%%%%%%%%%%%%%%%%%%%%%%%%%%%%%%%%%%%%

%%%%%%%%%%%%%%%%% BODY OF PAPER %%%%%%%%%%%%%%%%%%

\section{Introduction}
Globular clusters (GC)s are omnipresent, bright (with ${<}M^0_V{>} = -7.1 \pm 0.4$
that is nearly independent of the host galaxy size and environment,
\citealt{1991ARA&A..29..543H}), old ($\gtrsim$10~Gyr, \citealt{2005A&A...439..997P,
2005AJ....130.1315S, 2009ApJ...694.1498M, 2013ApJ...775..134V}), and amongst the
simplest stellar configurations in the Universe. The typical luminosity gives
a characteristic mass of $1.4\cdot 10^5$~M\Sun \, (assuming a mass to light ratio
$M/L_V \simeq 2$), and GCs have typical half-mass radii $r_h$ of a few parsec.
GCs within the Milky Way can thus be studied in great detail (at the level of 
individual stars), whereas integrated quantities can be explored with reasonable 
certainty for GCs within - and even beyond \citep[e.g.][]{1991ARA&A..29..543H} -
the Local Group. 

The extensive literature on GCs is summarised in several 
books and review articles \citep[e.g.][]{1991ARA&A..29..543H, Harris2001, 2004Natur.427...31W, 
2006ARA&A..44..193B, 2012A&ARv..20...50G, 2014CQGra..31x4006K, 2018RSPSA.47470616F},
and recent analyses of the Gaia data shed new light on the chemical- and structural 
properties, and on the phase-space distribution of the MW GC system \citep[e.g.][]{
2018A&A...616A..12G, 2018ApJ...859L..13B, 2019A&A...621A..56P, 2019MNRAS.484.2832V}.
A clear picture of the formation GC systems has yet to emerge, despite decades 
of studies and the tremendous advances the last couple of years. It is clear, 
however, that a fundamental understanding of GC formation and evolution promises 
exciting insight into the detailed histories of their host galaxies due to their
old ages and several correlations between GC systems and their parent galaxies.

% There are several correlations between globular cluster systems and their
% host galaxy. For example, the total mass of globular cluster systems
% correlates with the halo mass \citep[e.g.][]{2010MNRAS.406.1967G}, the
% the total number of GCs increases linearly with increasing virial mass
% \citep[e.g.][]{2019arXiv190100900B}, and THIRD EXAMPLE.

The literature offers a wealth of formation channels, most of which can be broadly
classified as ex situ, or as in situ scenarios. In the former picture, GCs form
in smaller, more metal-poor progenitors that are accreted in subsequent mergers,
as can naturally be expected in the framework of hierarchical galaxy formation
\citep[e.g.][]{1978ApJ...225..357S, 1984ApJ...277..470P}, and Rosenblatt et al. 1988;
Cen 2001; Bromm \& Clarke 2002; Diemand et al. 2005; Moore et al. 2006; Boley et al. 2009;
Griffen et al. 2010; Trenti et al. 2015; Kimm et al. 2016; Ricotti et al. 2016;
Boylan-Kolchin 2017; Creasey et al. 2019. As for in situ scenarios,
\citet{1985ApJ...298...18F} propose that thermal instabilities in the collapsing
proto-galaxy may lead to GC formation, \citet{1992ApJ...384...50A, 2010ApJ...718.1266M,
2018MNRAS.480.2343C} suggest that GC formation is triggered by (major) gas-rich
galaxy-galaxy mergers and/or interactions, and \citet{2005ApJ...623..650K} envision
GCs to form in the cores of supergiant molecular clouds located within gaseous disks
of (high-redshift) galaxies.

Another broad distinction between formation mechanisms are models that rely on
special conditions in the early Universe, or those that test the hypothesis that
GC formation at high redshift is governed by the same physical processes of young
massive clusters \citep[see][for a review]{2010ARA&A..48..431P} in the Local
Universe, where differences arise as the result of nearly a Hubble time of
(dynamical) evolution. This (latter) scenario is extensively tested in the recent
E-MOSAICS project \citep{2018MNRAS.475.4309P, 2019MNRAS.486.3134K}, which
comprises a suite of cosmological zoom simulations of MW-like galaxies. As for
the former category, \citet{2017Ap&SS.362..183R} investigate what fraction of
(multiple) stellar populations ends up in bound star clusters after their
formation in gravitationally unstable expanding supershells, which arise from
hypernova explosions of massive pop. III stars. \citet{2019arXiv190408941C}
speculate in a recent study that GCs may form in supersonically induced gasseous
objects at high redshifts, and \citet{2019arXiv190508951M} propose that high speed
collisions leave cooling gas without their DM subhaloes which could form GC
populations that can reproduce the observed $N_{\text{GC}}-M_{\text{vir}}$
correlation \citep[e.g.][]{2019arXiv190100900B} with bimodal metallicity
distributions and ages consistent with observations.



One important constraint on plausible formation mechanisms comes from observed
bimodality in the colour distribution of most galaxies, believed to be indicative
of bimodality in the underlying metallicity distribution of GC systems. GCs in
the MW can indeed be split up into a metal-rich (red), and a metal-poor (blue)
(sub)population. The former group is associated to the galactic disk because it
is a flattened, rotating population found at radii smaller than solar. The latter,
on the other hand, is more radially extended, more populated (with three times
more blue than red GCs), and may be $\leq 1.5$~Gyr older \citep{2005AJ....130..116D}.
The exact dividing metallicity slightly differs between authors, e.g.
[Fe/H]~$=-0.8$ \citep[according to][]{1985ApJ...293..424Z}, or
[Fe/H]~$=-1.0$ \citep[according to][]{Harris2001}, and may also differ between
galaxies \citep[e.g.][]{1985ApJ...293..424Z, 1999AJ....118.1526G, 2001AJ....121.2974L,
2006ApJ...639...95P}. In contrast, it has long been debated whether the Andromeda
GCS metallicity distribution is uni- bi- or multimodal, and \citet{2016ApJ...824...42C}
argue that the data favours three groups (split at [Fe/H] $= -0.4$ and $-1.5$)
after more accurate age-measurements improved classification of star clusters
as globular (or otherwise). Moreover, the Andromeda GCS has $2-3$ times as many
GCs at large radii, believed to be indicative of a more rich merger history.


The latter is supported by observed similarities between the spatial distribution
and chemical signature of the `blue' GCs and the Galactic stellar halo
\citep{2008A&ARv..15..145H}. Moreover, relative age-estimates indicate that the
metal-poor subpopulation may be $1.5$~Gyr younger than the metal-rich counterpart
\citep{2005AJ....130..116D}, providing further support that the GC population indeed
consists of two distinct classes.

People postulate that the frequent mergers that occur during hierarchical build-up
of galaxies naturally produce (sub)populations of GCs from enriched gas on tight
orbits, if the mass ratios and gas fractions of the mergers are sufficiently high.

Boylan-Kolchin: the blue population could form in high density regions along the
cosmic filament before or during the collapse of the proto-galaxy itself.
This formation scenario would explain the presence of the oldest GCs with
primordial metallicities in the (outer) haloes of galaxies as the GCs are
accreted over time. Moreover, the tidal fields experienced by GCs at (outer) halo
orbits is significantly lower than by GCs in the inner galaxy, thus such GCs
are expected to be long-lived.


An alternative paradigm is that GC formation does not rely on special conditions
at high redshifts. Rather, the same physics that governs the formation of young
massive clusters (YMCs) in the Local Group can be applied to the high redshift
Universe. The observed differences between YMCs and GCs, in this scenario, arise
naturally as the result intrinsic GC evolution within galactic tidal fields over
a Hubble time. This interesting hypothesis is tested numerically by the E-MOSAICS
people.


\citet{2010MNRAS.404.1203F} collected age-metallicity observations from the
literature and found one track with constant age of 12.8~Gyr, and one track
with branches towards younger ages. These branches are dominated by GCs associated
with the Sagittarius, Canis Major and other accreted dwarf galaxies, ultimately
indicating that 27-47 GCs (about one 25\% of the MW GCS) have been accreted
from six to eight dwarf galaxies. \citet{2019MNRAS.486.3134K, 2019MNRAS.486.3180K}
show that GC simulated age-metallicity space of the E-MOSAICS suite indeed allows
for a reconstruction of the merger history of the Milky Way-like galaxies, and
conclude that 40\% of the MW GCS has been accreted from galaxies with a stellar
mass range of $2 \cdot 10^{7} - 2 \cdot 10^{9}$ M\Sun, and $6\pm1$ are core remnants
of accreted dwarfs.



% \begin{itemize}
%     \item \citet{2005MNRAS.364..367D}: ``The radial profile of the stellar halo
%     and metal-poor globular clusters of the Milky Way suggest that these components
%     formed in rare early peaks above $2.5 \sigma$ at redshift above 10. ''
%
%     \item \citet{2017MNRAS.465.3622R}: ``GCs among oldest astrophysical objects.
%     GCs form in the early Universe in highest density peaks
%     \citep[e.g.][]{2005MNRAS.364..367D, 2009ApJ...706L.192B}''
%
%     \item ``Hence, they witness most of the formation and evolution processes
%     of galaxies, and can be used to probe them'' \citep{2006ARA&A..44..193B}
% \end{itemize}


\begin{itemize}
    \item Special conditions in early Universe?
    \item Formation in collapsing proto-galaxies (in-situ-ish?)
    \item Formation as a result of (wet) mergers?
    \item Formation in satellites that are later accreted?
    \item No special conditions at high z, but connection between low-z YMC-
        and high-z GC-formation where differences could result from Hubble time
        of evolution?
    \item This means that in order to test formation models against observations,
        one not only needs needs to model GC formation, but also include the full
        evolution over their lifetime in a cosmological tidal field.
\end{itemize}

\todo[inline]{Paragraph: ingredients for numerical studies}
\begin{itemize}
    \item High resolution in the ISM
    \item Full cosmological evolution
\end{itemize}

% \todo[inline]{Paragraph: scientific context / state of the field: briefly summarise
%     previous work and references work of other groups}
% \begin{itemize}
% \item Origin of the Milky Way globular clusters \citep{2017MNRAS.465.3622R}
% \item GCs in FIRE \citep{2018MNRAS.474.4232K}
% \item EMOSAICS project \citep{2018MNRAS.475.4309P}
% \item GC-DMhalo connection \citep{2017MNRAS.472.3120B}
% \item Origin of GC bimodality? \citep{2018MNRAS.479..200F}
% \item \textit{GAIA} DR2: GC kinematics \citep{2018A&A...616A..12G}
% \item Dating GC Tidal Disruption \citep{2018ApJ...859L..13B}
% \item GC in N-body simulation \citep{2018ApJ...861...69C}
% \item (Role of GC mass evolution on stream properties \citep{2018MNRAS.474.2479B})
% \item GC formation from dwarfs to giants \citep{2018MNRAS.480.2343C}
% \item (GC contribution to EOR \citep{2018MNRAS.479..332B})
% \item Early Universe supermassive star / GC formation \citep{2018MNRAS.478.2461G}
% \item GC formation in cold filaments \citep{2018ApJ...861..148M}
% \item GC formation in high-redshift dwarf galaxies \citep{2018MNRAS.477..480Z}
% \item GCs in MW outer region \citep{2017arXiv170804542P}
% \item Impact of the Cutoff of the Cluster Initial Mass Function \citep{2018arXiv181001888C}
% \item Metallicity gradients in the globular cluster systems of early-type galaxies: in situ and accreted components \citep{2018MNRAS.479.4760F}
% \item Globular clusters in M31, Local Group, and external galaxies \citep{2016IAUS..317..120L}
% \item Globular Clusters Formed within Dark Halos I: present-day abundance, distribution and kinematics \citep{2019MNRAS.482..219C}
% \item (The mass of the Milky Way from satellite dynamics \citep{2018arXiv180810456C})
% \item Globular cluster formation and evolution in the context of cosmological galaxy assembly: open questions \citep{2018RSPSA.47470616F}
% \item The kinematics of globular clusters systems in the outer halos of the Aquarius simulations \citep{2016A&A...592A..55V}
% \item Star Cluster Formation in Cosmological Simulations \citep{2017ApJ...834...69L, 2018ApJ...861..107L, 2018arXiv181011036L}
% \item (A systematic analysis of star cluster disruption by tidal shocks - I. Controlled N-body simulations and a new theoretical model \citep{2018arXiv181200014W})
% \item Spatial mixing of binary stars in multiple-population globular clusters \citep{2018MNRAS.tmp.3147H}
% \item Star Clusters Across Cosmic Time \citep{2018arXiv181201615K}
% \item Kinematics of Subclusters in Star Cluster Complexes: Imprint of their Parental Molecular Clouds \citep{2018arXiv181201858F}
% \item Investigating the population of Galactic star formation regions and star clusters within a Wide-Fast-Deep Coverage of the Galactic Plane  \citep{2018arXiv181203025P}
% \item Globular cluster number density profiles using \textit{Gaia} DR2 \citep{2019arXiv190108072D}
% \item Clustering Clusters: unsupervised learning on GC observations (of log $M$,
%     log $\sigma_0$, log $R_e$ , [Fe/H], log $|Z|$) to determine number of groups
%     $k=2$ (disk/halo) or $k=3$ (disk/inner-halo/outer-halo): the optimal number
%     of groups given the data is $k=3$. \citep{2019arXiv190105354P}.
% \item There is a well-known hypothesis that Open Stars Clusters (OSCs) could form as a result
%     of GCs crossing the disk. New kinematical data does not support
%     this claim for two particular OSC-GC pairs, and does seem consistent
%     with this scenario for six OSC-GC pairs \citep{2019arXiv190106481B}. (Note
%     that GC crossing the disk could also trigger formation of multiple OSCs!)
% \item DM halo mass can be inferred from number of globular cluster in galaxy
%     \citep{2019arXiv190100900B}
% \item GCs in M31 outer halo \citep{2019arXiv190111229W}
% % TODO: last checked arXiv Mar 2, 2019
% \end{itemize}




% \todo[inline]{Paragraph, narrowing open questions in GC research down to formation,
%     possibly ``Bimodality suggests two formation mechanisms''. In-situ
%     formation vs. accretion in hierarchical build-up of galaxies naturally produces
%     two populations of globular cluster.}
% \begin{itemize}
%     \item ``colour bimodality, blue and red clusters \citep[e.g.][]{1985ApJ...293..424Z, 1999AJ....118.1526G, 2001AJ....121.2974L, 2006ApJ...639...95P}
%     \item ``blue metal-poor (with distribution peaking at [Fe/H] $\approx -1.5$ for the Milky Way), no sign of rotation as a population (..) more metal-rich (peak at [Fe/H] $\approx -0.5$ in the Milky  Way) more spatially concentrated and rotating with the galaxy.'' \citep{1996AJ....112.1487H}
%     \item ``Blue clusters from in early Universe in galaxies that merge later. In (wet) merger process starbursts generate red population \cite{1992ApJ...384...50A, 1987nngp.proc...18S}''
%     \item ``\citet{1997AJ....113.1652F} propose instead that blue globulars form when the protogalaxy itself collapses, in a metal-poor and turbulence media. The red population would form later, once the galactic disc has settled. The formation of globular clusters would then be a multiphase process, with the first phase being interrupted possibly by cosmic reionization \citep{2002MNRAS.333..383B}.''
%     \item ``\citet{2005ApJ...623..650K, 2014ApJ...796...10L} advocate
% that major mergers are at the origin of both sub-populations: blue
% clusters form during early mergers (z > 4) while the red ones appear
% in mergers at lower redshifts (even after z = 1). Although, this
% scenario, combined with star formation enhancement in mergers,
% seems appropriate in dense galactic environment leading to the
% assembly of massive elliptical galaxies, like in the Virgo Cluster
% as tested by \citet{2014ApJ...796...10L}, it does not apply to Milky Way-
% like systems where no recent major merger took place (Wyse 2001;
% Deason et al. 2013; Ruchti et al. 2014, 2015).
% \item ``\citet{1998ApJ...501..554C} argue that red clusters form in situ while the blue ones are accreted, either via merging satellite galaxies, or by tidal capture of the clusters themselves \citep[see also][]{2013ApJ...762...39T}.''
% \end{itemize}



\todo[inline]{Paragraph: narrow down to this work}
The star formation model implemented in the Auriga simulations is capable of
producing realistic Milky Way-like (spiral disk) galaxies at redshift zero. Therefore
the question naturally arises whether or not the Auriga simulations also produce
a subset of star particles (GC candidates) which is consistent with the GC system
of the Milky Way and/or the Andromeda galaxy (M31). The typical mass scale of
globular clusters is comparable to the numerical (mass) resolution of cosmological
zoom simulations. The detailed physics of GC formation in the Universe appears in
observations as the combined effect of the $10^{3-6}$ M\Sun, compared to a mass
resolution of $10^{3-5}$ M\Sun in the Auriga simulations. Globular clusters can
therefore serve as an ultimate test to the star formation model that is implemented
in the numerical simulations. Moreover, cosmological zoom simulations provide a
detailed (formation and merger) history of the simulated galaxies. This is
important because theoretical paradigms for globular cluster formation in the
literature know two distinct classes of GCs that are separated by their exact
formation sites: an in situ versus an accreted population. Cosmological zoom
simulations uniquely allow for an investigation into globular cluster formation
with a particular focus on the properties of the in situ versus accreted subpopulations.


\todo[inline]{Paragraph: Paper outline}
We summarise the relevant characteristics of the Auriga simulations in
section~\ref{sec:auriga}, followed by a summary of the observations of the
Milky Way (MW) globular cluster system (GCS) in section~\ref{sec:observations}
that we use to compare our simulations to in section~\ref{sec:results}. We
discuss our findings in section~\ref{sec:discussion} to come to our conclusions
in section~\ref{sec:conclusions}.


\section{The Auriga simulations}
\label{sec:auriga}
We use the Auriga simulations \citep[][hereafter G17]{2017MNRAS.467..179G}, a
suite of high-resolution cosmological zoom simulations of Milky Way-mass
selected initial conditions. The simulations are performed with the
state-of-the art code \textsc{arepo} \citep{2010MNRAS.401..791S,
2016MNRAS.455.1134P}, that solves the magnetohydrodynamical equations on a
moving mesh, and an elaborate galaxy formation model that produces realistic
spiral galaxies at redshift $z=0$.

The interstellar medium is modelled using a sub-grid approach which implements
the physical processes most relevant to galaxy formation and evolution.
This model was tailored to the \textsc{arepo} code and calibrated to reproduce
key observables of galaxies, such as the history of the cosmic star formation rate
density, the stellar mass to halo mass relation, and galaxy luminosity functions.

The sub-grid includes primordial and metal-line cooling with self-shielding
corrections. Reionization is completed at redshift six by a time-varying
spatially uniform UV background \citep{2009ApJ...703.1416F, 2013MNRAS.436.3031V}
The interstellar medium is described by an equation of state for a two-phase medium
in pressure equilibrium \citep{2003MNRAS.339..289S} with stochastic star formation
in thermally unstable gas with a density threshold of $n = 0.13 \text{cm}^{-3}$,
and consecutive stellar evolution is accounted for. Stars provide feedback by
stellar winds \citep{2014MNRAS.437.1750M, 2017MNRAS.467..179G}, and further
enrich the ISM with metals from SNIa, SNII, and AGB stars \citep{2013MNRAS.436.3031V}.
The formation of black holes is modelled which results in feedback from active
galactic nuclei \citep{2005MNRAS.361..776S, 2014MNRAS.437.1750M, 2017MNRAS.467..179G}.
Finally, the simulations follow the evolution of a magnetic field of $10^{-14}$
(comoving)~G seeded at $z = 127$ \citep{2013MNRAS.432..176P, 2014ApJ...783L..20P}.
See G17 for further details of the numerical setup as well as the galaxy formation
model.

The Auriga suite has a fiducial resolution level L4, accompanied by the lower
(higher) level L5 (L3) that is available for selected initial condition runs.
The baryonic mass resolution in order of increasing level is $m_b$~=~[$4 \times 10^5$,
$5 \times 10^4$, $6 \times 10^3$]~M\Sun \, with gravitational softening of
collissionless particles $\epsilon$~=~[$738, 369, 184$]~pc. The mass resolution
of the Auriga simulations is thus close to the characteristic peak mass of the
lognormal GC mass distribution of $10^{5}$~M\Sun \citep{1991ARA&A..29..543H},
although the gravitational softening is two orders of magnitudes larger than
typical GC radii. High-density gaseous regions are thus not expected to produce
surviving stellar clumps (GCs) with masses and radii consistent with GCs because
such objects would numerically disperse, even in the highest-resolution runs.
On the other hand, we can investigate (statistical) properties of age-selected
GC candidates because each star particle represents a single stellar population
with a total mass that could be consistent with one globular cluster.



% \todo[inline]{
% TODO: paraphrase ``The diversity in morphological properties of these simulated
% galaxies reflects the stochasticity inherent to the process of galaxy formation
% and evolution
% \citep[e.g.][]{2005ApJ...635..931B, 2010MNRAS.406..744C, 2010ApJ...708.1398T}.''
% }
%
% \todo[inline]{Paragraph about Auriga's stellar haloes.
% ``The Auriga Stellar Haloes: Connecting stellar population properties with
% accretion and merging history'' \citet{2018arXiv180407798M}
% }


\section{Observational data}
\label{sec:observations}
We describe the observations of the MW GCS in Sec.~\ref{sec:milkyway},
and of the Andromeda (M31) GCS in Sec.~\ref{sec:andromeda}


\subsection{Milky Way}
\label{sec:milkyway}
\citet[][2010 edition; hereafter H96e10]{1996AJ....112.1487H} provides a
catalogue\footnote{See \url{https://www.physics.mcmaster.ca/Fac_Harris/mwgc.dat}}
of the Milky Way globular cluster system that contains properties of
157 GCs. The authors initially estimated the size of the MW GCS to be 180~$\pm$~10,
thus, their catalogue to be $\sim$85\% complete. However, an additional 59 GCs
have since been discovered by various authors. The total confirmed number of GCs
in the MW adds up to 216 with new estimates now anticipating an additional thirty
GCs yet to be discovered \citep[e.g.][and references therein]{2018ApJ...863L..38R}.
We still use data from the Harris catalogue, but caution that it may (only) be
53-72\% complete. Specifically, the relevant data fields that we use from H96e10
are the metallicity [Fe/H], the Galactic distance components $X$, $Y$, and $Z$ (in
kpc)\footnote{In a Sun-centered coordinate system: $X$ points toward Galactic
center, $Y$ in direction of Galactic rotation, and $Z$ toward the North Galactic
Pole. We calculate the galactocentric radius $R_{\text{GC}}=\sqrt{(X-R_\odot)^2
+ Y^2 + Z^2}$, assuming the solar radius $R_\odot=8$~kpc.}, and absolute
magnitude in the V-band $M_V$. We use the latter to calculate mass-estimates by
assuming $M_{V,\odot}=4.83$ and a mass to light ratio $M/L_V = 1.7$~M/L$_{\odot}$,
the mean for MW clusters \citep{2005ApJS..161..304M}.

% \citet{2019AJ....157...12B} communicate the latest efforts to aggregate the
% available data, presented in their CatClu catalog. Amongst 10978 star clusters
% and alike objects in the Milky Way, the catalog contains 200 GCs and 94 GC
% candidates. The CatClu catalog contains reference papers, positions, distances,
% and total absolute V magnitude. Therefore we rely on the H96e10 dataset for all
% other quantities, but

We supplement the catalogue with age-estimates from isochrone fits to stars
near the main-sequence turnoff in 55 GCs \citep[][hereafter V13]{2013ApJ...775..134V}.
The mean value of the age-estimates in this data set is $11.9 \pm 0.1$~Gyr and
the dispersion is $0.8$~Gyr. Furthermore, only one of the 55 GC age-estimates is
below $10$~Gyr.

\subsection{Andromeda}
\label{sec:andromeda}
The fifth revision of the revised bologna catalogue (RBC~5, last updated
August, 2012) is the latest edition of three decades of systematically
collecting integrated properties of the globular cluster system of the
Andromeda galaxy \citep[][and references therein]{2004A&A...416..917G}. One
contribution to RBC~5 is the work by \citet[][hereafter C11]{2011AJ....141...61C},
subsequently updated by \citet[][hereafter CR16]{2016ApJ...824...42C}.

C11 and CR16 present a uniform set of spectroscopic observations calibrated
on the Milky Way GCS of the inner $1.6^\circ~(\sim21)$~kpc that
is believed to be 94\% complete. GCs in the outer stellar halo, up to
$R_{\text{proj}}\sim150$~kpc, are observed in the Pan-Andromeda Archaeological
Survey \citep[PAndAS, ][hereafter H14]{2014MNRAS.442.2165H}, but see also
\citet{2014MNRAS.442.2929V} and \citet{2019MNRAS.484.1756M}. H14 presents the
discovery of 59 new GCs and publishes updates to RBC~5. The work of H14 is
incorporated in the latest public release\footnote{Last revised 23 Sep 2015, see
\url{https://www.cfa.harvard.edu/oir/eg/m31clusters/M31_Hectospec.html}}
of the C11 dataset, further revised by CR16. It seems that CR16 is the most
recent aggregated dataset of M31's GCS that contains properties of interest
for our study as it contains GCs in the inner region and in the outer halo. The
relevant fields in the CR16 dataset that we use are the age, metallicity, and the
mass-estimate\footnote{The authors assumed $M/L_V = 2$ independent of [Fe/H]}.

The M31 GCS has a mean age of $11.0 \pm 0.2$~Gyr with a dispersion of $2.2$~Gyr,
and $24$ GCs have age-estimates below $10$~Gyr with a minimum age of $4.8$~Gyr.
We show a histogram of the age-estimates of the $55$ MW GCs in V13 and $88$ GCs
in M31 for which age-estimates are available in CR16, see Figure~\ref{fig:MW-M31-age}.

\begin{figure}
    \includegraphics[width=\columnwidth]{{MW_M31_Age_Histogram-trim}.png}
    \caption{
        Age distribution of 55 GCs in the MW
        \citep[data from][]{2013ApJ...775..134V} and 88 GCs
        in M31 \citep[data from][]{2016ApJ...824...42C}.
        \label{fig:MW-M31-age}
    }
\end{figure}
% \begin{figure}
%     \includegraphics[width=\columnwidth]{{MW_M31_Age_HistogramMassWeighted_Msun}.png}
%     \caption{
%         Mass-weighted age distribution of 55 GCs in the MW
%         \citep[data from][]{2013ApJ...775..134V} and 85 GCs
%         in M31 \citep[data from][]{2016ApJ...824...42C}.
%         \label{fig:MW-M31-age}
%     }
% \end{figure}

\subsection{Total GC mass in metallicity-radial space}
\label{sec:observations_FeHRgc}
We show the two-dimensional mass-weighted metallicity-radial distribution of the
MW (M31) GCS in the top (bottom) panel of Figure~\ref{fig:observations_FeHRgc}. Both
quantities are readily available in H96e10 (assuming $R_{\odot}=8.0$~kpc), but
the galactocentric radius of GCs in M31 is not available in CR16. Therefore we
follow \citet[][Sec.~4.1]{2019arXiv190111229W} to calculate the projected radius
$R_{\text{proj}}$ from the observed positions, adopting M31's central position
from the NASA Extragalactic Database\footnote{\url{https://ned.ipac.caltech.edu/}}
$(\alpha_0, \, \delta_0) =
(0^{\text{h}}42^{\text{m}}44.35^{\text{s}}, \, +41^{\circ}16'08.63")$
and distance $D_{\text{M31}} = 778$~kpc (McConnachie et al. 2005; Conn  et  al. 2012).
We calculate $R_{\text{GC}}$ as `average deprojected distance`
$R_{\text{GC}} = R_{\text{proj}} \times (4/\pi)$. The observations indicate that
no GCs with high metallicities are to be expected at large radii (the three bins
in the upper right corner, both for MW and M31), and relatively few GCs at large
radii in general ($R_{\text{GC}} > 30$~kpc; right column: $11$ GCs or 7.3\% in
the MW and $17$ or 4.6\% in M31). We compare these observations to the Auriga
simulations later on in Sec.~\ref{sec:results_FeHRgc}.

\begin{figure}
    \includegraphics[width=\columnwidth]
        {{MW_RgcFeH_HistogramMassWeighted_Harris1996ed2010data}.png}
    \includegraphics[width=\columnwidth]
        {{M31_RgcFeH_HistogramMassWeighted_CaldwellRomanowsky2016data}.png}
    \caption{
        \emph{Top}: Mass-weighted $R_{\text{GC}}$-[Fe/H] distribution of
        151 GCs in the MW \citep[data from][2010 ed.]{1996AJ....112.1487H}, which
        is 98.6 (92.2) \% of the total mass (clusters) of the MW GCS in the Harris
        catalog. \emph{Bottom}: Same for M31, showing 366 GCs and 88.4 (83.9) \%
        of the total mass (clusters) in CR16 \citep[data from][]{2016ApJ...824...42C}.
        Note that the range of the colourmap differs in both figures.
        \label{fig:observations_FeHRgc}
    }
\end{figure}


\section{Results}
\label{sec:results}
We define GC candidates in the Auriga simulations as all star particles older
than $10$~Gyr based on the age distribution of the MW GCS (Figure~\ref{fig:MW-M31-age}),
and following the analysis of \citet{2017MNRAS.465.3622R}.

Trough out our analysis we compare the distributions of three subsets of star
particles: \emph{old stars} (age $>10$~Gyr, or GC candidates), \emph{old insitu}
stars (defined as those bound to the most-massive halo/subhalo in the first
snapshot that the particle was recorded), and \emph{old accreted} star particles
(those that have formed ex-situ and are bound to the most-massive halo/subhalo
at $z=0$). For comparison we also include the results for \emph{all stars} (when
no additional selection criterion is applied to the star particles). We consider
the metallicity distribution in Sec.~\ref{sec:results_FeH}, the distribution of
galactocentric radii in Sec.~\ref{sec:results_Rgc}, and the combination of both
in Sec.~\ref{sec:results_FeHRgc}.


\subsection{Metallicity distribution}
\label{sec:results_FeH}
We investigate whether the star formation model implemented in Auriga produces
metallicity distributions consistent with the MW (M31) GC system, and whether
the subgrid generates sufficient total mass at metallicities typical for the
MW (M31) GCS.

\begin{figure*}
    \includegraphics[width=\columnwidth]{{Au4-10_FeH_cleaner-trim}.png}
    \includegraphics[width=\columnwidth]{{Au4-4_and_Au4_21_FeH-trim}.png}
    \caption{
        \emph{Left:} Metallicity distribution of Au4-10 (bottom panel). We show
        the GG candicates in green. We split the GC candidates into two
        subpopulations, those that have formed insitu (blue), and those that have
        been accreted (red). The dotted green line shows all star particles.
        The solid purple (magenta) line in the top panel shows the GC system
        of the MW (M31). \emph{Right:} Au4-4 (bottom), and Au4-21 (top).
        \label{fig:Au4-10and21_FeH}
    }
\end{figure*}
Figure~\ref{fig:Au4-10and21_FeH} shows the normalized metallicity distribution
of \mbox{Au4-10}\footnote{The nomenclature is `Au' for Auriga, followed by the resolution
level (4) and halo number (10 - indicating which set of initial conditions was
used the run).} and \mbox{Au4-21}, showing the GC candidates in green. The accreted GC
candidates are shown in red, and the insitu subpopulation in blue. The top panels
show the MW (M31) GC system in purple (magenta). We overplot a double Gaussian
for the MW GCS (the purple dashed lines), adopting literature values of the
mean $\mu$ and standard deviation $\sigma$ of the metal-rich and metal-poor
populations \citep[][p. 38]{Harris2001}.

We find that the age cut lowers the mean metallicity from $0.0$ to $-0.6$.
Furthermore, the old accreted stars generally have lower metallicities than the
old insitu stars. The difference between the mean metallicity of the old insitu
and old accreted stars for the majority of the simulations is \textasciitilde{}$0.3$ (e.g.
\mbox{Au4-21}), and \textasciitilde{}$0.5$ dex for \mbox{Au4-10}, \mbox{Au4-16}, \mbox{Au4-17},
\mbox{Au4-18} and \mbox{Au4-22}. This trend is only reversed for \mbox{Au4-1}
and \mbox{Au4-4}, for which the old insitu population has a lower mean metallicity
instead. For \mbox{Au4-1} we find $\mu = -1.51$ (-0.74) for the old insitu (accreted),
but the former consists of only $1019$ particles ($1.3$\% of all GC candidates, and
with a total mass of 5e7 M\Sun) thus we believe our classification of insitu is
flawed for this halo due to misclassification of the primary halo in the merger
tree by subfind. For \mbox{Au4-4}, $10.8$ \% of the GC candidates is classified
as insitu (compared to insitu fractions of $40-80$\% for other haloes). Overall
we find that the simulations produce (sub)populations of GC candidates that are
more metal-rich than the MW and M31 GC systems. Moreover, none of the simulations
has a population of GC candidates with a bimodal metallicity distribution
(the green curves).

\begin{figure}
    \includegraphics[width=\columnwidth]{{FeH_mu_sigma-trim}.png}
    \caption{
        First vs second central moment of the Auriga L4 metallicity distributions.
        Each cross (for a given colour) represents one simulation. The green (blue)
        [red] crosses show the values calculated using the old (old insitu)
        [old accreted] star particles. Green triangles indicate that all stars
        were used. The purple (magenta) cross denotes our calculation using all
        MW (M31) observations (which would be appropriate for a unimodal
        distribution). The black solid (open) dots indicate the literature values
        of a bimodal Gaussian fit to the data \citep[values from][p. 38]{1998gcs..book.....A},
        showing the metal-rich (metal-poor) component of the MW.
        \label{fig:FeH_mu_sigma}
    }
\end{figure}

We show the mean metallicity and standard deviation of all thirty Auriga L4 haloes
in Figure~\ref{fig:FeH_mu_sigma}. The green crosses are to be compared to the
purple (magenta) cross, which shows the mean value of all MW (M31) GCs.
We test the null hypothesis that MW (M31) and the GC candidates (all, insitu, and
accreted) are drawn from the same underlying distribution by calculating the KS
test statistic (i.e. six KS tests per simulation). We reject $H_0$ at the 100.00~\%
confidence level for virtually all subsets of all simulations, both for MW and M31,
except when comparing the old accreted stars of \mbox{Au4-10} to the MW GCS. In this
case we still reject $H_0$ with, but with $\leq 98.68$~\% confidence. When comparing
to the M31 GCS, the `best' matches are the old accreted subpopulations of \mbox{Au4-13},
\mbox{Au4-15}, and \mbox{Au4-17} ($H_0$ rejected with $\leq$ 99.82~\% confidence, 99.95~\%,
and 98.32~\%). Here we find that the cumulative distributions are more or less
similar up to [Fe/H]~$-0.5$, above which the simulations yield higher number
counts which drives the KS test statistic over the critical values. This affirms
our finding that the simulations produce GC candidates that are more metal-rich
than the MW (M31) GCS. The main reason reasons that $H_0$ is rejected with such
high confidence are under-production of old stars with [Fe/H]~$\leq$~-1.5,
and over-production of stars with [Fe/H]$\geq$~-0.5.

In addition, we show the metal-rich (metal-poor) population of the MW GCS using
a solid (open) dot. The mean metallicities of the old insitu populations appear
roughly consistent with that of the metal-rich population of the MW, but the
simulations show larger dispersions. We are uncertain whether definitive consensus
is reached concerning uni- or bimodality in the [Fe/H] distribution of M31, but
CR16 argues that the data, after removal of younger objects due to improved age
classification, leans towards unimodality. Therefore we do not show two data
points for the M31 GCS.


\begin{figure}
    \includegraphics[width=\columnwidth]{{logMFeH_withRatios-trim}.png}
    \caption{
        Mass-weighted metallicity distribution of star particles in the Auriga
        simulations. We show the median value of all Auriga haloes for all
        stars (green dotted) and globular cluster candidates (i.e. stars with
        age $>$~10~Gyr; green solid). The latter sub set is further split up
        between stars that formed in-situ (blue solid), and those that were accreted
        (red solid). Shaded regions indicate the $1\sigma$ interval. The MW (M31)
        GCS is shown in purple (magenta). The middle (bottom) panel shows the
        ratio of the simulated mass to the mass in the MW (M31) GCS.
        \label{fig:FeH}
    }
\end{figure}
We now turn to the total mass in GC candidates produced by the Auriga simulations.
We show the median (coloured lines) for all thirty Auriga L4 haloes with the
$1\sigma$ interval around it (shaded regions, which shows the scatter between
runs that have different initial conditions, thus, have unique merger histories).
We chose to aggregate the data to indicate a general trend that we find when the
GC candidates are split up according to birth location, rather than selecting
typical examples of individual galaxies. Figure~\ref{fig:FeH} shows the mass-weighted
metallicity distribution of all Auriga L4 galaxies. Once again we notice that the
peak metallicity shifts down from $0$ to $-0.5$ for old stars (green solid)
compared to all stars (green dotted), and we learn that the mass at the peak
lowers by roughly one dex. The mass budget of the old stars is dominated by the
old insitu population (blue solid) below [Fe/H]~=~$-1$, and by the old accreted
stars (red solid) above this value. We show the MW (M31) GCS in purple (magenta)
and we show the ratio of the simulated to the observed profiles in the middle
(bottom) panel. This mass excess can be thought of the `mass budget' that the
Auriga GC candidates can `afford to lose' due to a combination of smaller than
unity bound cluster formation efficiencies combined with a Hubble time of dynamical
evolution, while still producing a GC system with sufficient mass to be consistent
with the MW (M31) GCS. The cluster formation efficiency would have to linearly
decrease with decreasing metallicity for Auriga GC candidates to produce a population
of GC candidates that is consistent with the MW. For the GC candidates in M31
we find a constant mass excess up to $-0.9$, above which the
simulations produce a higher mass excess with increasing metallicity. If dynamical
evolution is not expected to more efficiently disrupt GCs of higher metallicity,
then we would find that the efficiency to form bound star clusters would have to
decrease with increasing metallicity.


\subsection{Radial distribution}
\label{sec:results_Rgc}

\begin{figure}
    \includegraphics[width=\columnwidth]{{Rgc_mu_sigma-trim}.png}
    \caption{
        Mean and standard deviation of the radial distribution of star particles
        in each of the thirty Auriga L4 haloes compared to the MW (M31) GCS
        shown in purple (magenta).
        \label{fig:Rgc_mu_sigma}
    }
\end{figure}
Figure~\ref{fig:Rgc_mu_sigma} shows the mean and standard deviation of the
radial distribution of star particles in all Auriga L4 simulations. We notice
that the old insitu populations are much more centrally distributed, whereas
the old accreted component has a larger radial extent. Moreover, the dispersion
increases with increasing mean value of the radial distribution as may be expected
for a non-negative quantity.

\begin{figure}
    \includegraphics[width=\columnwidth]{{logMRgc_withRatios-trim}.png}
    \caption{
        Mass-weighted radial distribution of star particles in the Auriga
        simulations. We show the median value of all Auriga haloes for all
        stars (green dotted) and globular cluster candidates (i.e. stars with
        age $>$~10~Gyr; green solid). The latter sub set is further split up
        between stars that formed in-situ (blue solid), and those that were accreted
        (red solid). Shaded regions indicate the $1\sigma$ interval. The MW (M31)
        GCS is shown in purple (magenta). The middle (bottom) panel shows the
        ratio of the simulated mass to the mass in the MW (M31) GCS.
        \label{fig:Rgc}
    }
\end{figure}
Figure~\ref{fig:Rgc} shows the mass-weighted radial distribution of the
Auriga L4 haloes. We notice a subtle peak around $10$~kpc for all star
particles that is not present for the GC candidates, indicating that the
stellar disc is no longer present when applying the latter selection criterion.
Furthermore, we find that the dominant contribution to the total mass in GC
candidates changes from those formed insitu to the accreted population around
$10$~kpc. Again we show the mass excess of the simulations compared to the Milky
Way and Andromeda GCS. We find a decreasing mass excess with increasing radius
in the range $0.2$ to \textasciitilde{}$5$ kpc, followed by an increase attributed
to the accreted subpopulation. We further investigate a breakdown of the total mass
in Auriga GC candidates into bins of both metallicity and radius in the following
section.

\subsection{Total mass in metallicity-radial space}
\label{sec:results_FeHRgc}

We investigate whether the Auriga simulations still produce sufficient mass when
the GC candidates are two-dimensionally binned in [Fe/H] and $R_\text{GC}$. First
we sum the total simulated mass in each bin (for an individual Auriga simulation),
then we calculate the median over all thirty Auriga L4 haloes (in each bin, see
Figure~\ref{fig:Au-FeHRgc} which can be compared to
Figure~\ref{fig:observations_FeHRgc}). Finally, we divide these value by the total
mass in the MW (M31) GCS to obtain the median mass excess produced by the star
formation model implemented in the Auriga simulations. See the top (bottom) panel
of Figure~\ref{fig:Au-FeHRgc-ratio} for mass excess with respect to the MW (M31) GCS.

\begin{figure}
    \includegraphics[width=0.49\textwidth]
        {{Au4-median_RgcFeH_HistogramMassWeighted_iold-trim}.png}
    \caption{
        Mass-weighted [Fe/H]-$R_{\text{GC}}$ distribution of all thirty Auriga
        L4 haloes. Here we consider only the GC candidates (age $>$~10~Gyr) stars
        in and color-code by the \textbf{median} (values also shown in each bin).
        The numbers in parenthesis show how many star particles fall within the
        bin. Note that the range of the colourmap again differs (for improved
        contrast within the plot).
        \label{fig:Au-FeHRgc}
    }
\end{figure}

\begin{figure}
    \includegraphics[width=0.49\textwidth]
        {{Au4-median_RgcFeH_HistogramMassWeighted_MW_iold-trim}.png}
    \includegraphics[width=0.49\textwidth]
        {{Au4-median_RgcFeH_HistogramMassWeighted_M31_iold-trim}.png}
    \caption{
        The top (bottom) panel shows the logarithm of the ratio of simulated mass
        to mass in the MW (M31) GCS, i.e. the logarithm of the mass excess. The
        color-coded values are also shown in each bin. Note that the three bins
        in the upper right corner are left blank because the observations have
        zero mass there, and that the numbers indicate how much (median) mass is
        produced by the simulations in those bins.
        \label{fig:Au-FeHRgc-ratio}
    }
\end{figure}



\section{Discussion}
\label{sec:discussion}
\subsection{Metallicity distribution}
\label{sec:discussion_FeH}
We find that the metallicity distributions of GC candidates in the Auriga
simulations are more metal-rich than the MW (M31) GCS. Although the (old) star
particles represent single stellar populations with a mass (resolution) comparable
to the peak of the GCMF, they are in fact statistical tracers of the stellar
population of the galaxy as a whole. Therefore only a (small) fraction of the
star particles may represent plausible formation sites of GCs, whereas the majority
represent (halo) field stars - the disk component effectively falls outside
our selection of star particles due to the age cut. Globular clusters are
$\sim$~$0.5$ dex more metal-poor than spheroid stars observed at the same radius
for almost all galaxies \citep{1991ARA&A..29..543H}. Our finding that star particles older
than 10~Gyr have higher metallicities than the MW (M31) GCS may simply be a
reflection thereof.

\citet[][p. 234]{1998gcs..book.....A} and \citet[][p. 38]{Harris2001} quantify
the bimodality of the [Fe/H] distribution of the MW GCS, the latter fit a double
Gaussian which peaks at [Fe/H] = $-1.59$ (metal-poor) and $-0.51$ (metal-rich)
and dispersions of $0.34$ and $0.23$. However, the metallicity distribution of
GC candidates in the Auriga simulations is not bimodal, although an offset between
the mean values does emerge when further splitting them up according to birth location
(accreted or in-situ). The majority of the simulations show an offset in the mean
values of roughly $0.3$~dex between both subpopulations, while a small number of
simulations has an offset of order $0.5$~dex. Our finding that the mean values of
the old insitu GC candidates are roughly consistent with the metal-rich MW GCs
could be interpreted as an indication that this subpopulation may have formed
in-situ. However, we also notice that a substantial number of simulations has
similar mean values for all GC candidates, and the simulated insitu GC candidates
have larger dispersions than the metal-rich GCs. Moreover, the offset between the
mean of the metal-rich and metal-poor populations in the MW is $1$ dex, a factor
$2-3$ larger than the offset we find between the insitu and accreted populations.
The simulations are not consistent with the picture that all metal-rich GCs
formed insitu and that all metal-poor GCs were accreted.

\citet{2006ARA&A..44..193B} compare the number of metal-poor GCs to the stellar
halo mass and find\footnote{the quantity $T$ is the number of GCs per $10^9$ M\Sun \,
of galaxy stellar mass} $T^n_{\text{blue}} \sim 100$, while the
number of metal-rich GCs compared to the bulge mass yields $T^n_{\text{red}} \sim 5$,
and therefore conclude that the formation efficiency of metal-poor GCs is
twenty times higher than the metal-rich GCs with respect to field stars.

% Interestingly, the cross-over point above (below) which the mass-weighted
% metallicity distribution of GC candidates is dominated by those that have
% formed insitu (were accreted) coincides with the separation between the metal-rich
% and metal-poor population for the MW GCS at [Fe/H]~=~$-1$.


\subsection{Radial distribution}
\label{sec:discussion_Rgc}
The M31 GCS has a factor 2-3 more GCs at large radii compared to the MW GCS,
which may indicate that M31 has a richer merger history than the MW. In the
Auriga simulations we find that the GC candidates at radii larger than
$\sim20$~kpc are indeed dominated by accreted star particles. However, we select
star particles that are bound to the main halo and main subhalo, which means that
we include particles up to the virial radius $R_{200}$. The Auriga simulations
have no problem populating the stellar halo up to the virial radius, even with
our additional age cut. On the other hand, the MW and M31 have fewer GCs at large
radii. We expect tidal disruption to be less efficient at larger radii, thus,
the formation efficiencies of the accreted GC candidates would have to be lower.
Our classification as GC candidates could, for example, be improved by selecting
a number of globular clusters based on the $M_{vir}-N_{\text{GC}}$ correlation
found by \citet{2019arXiv190100900B}, and taking a virial mass below which none
of the star particles is treated as GC candidate.

% \todo[inline]{
%     Harris: `Somewhat arbitrarily, I will take the region $r_p > 3$ kpc
%     (containing 75 clusters) as the fiducial Milky Way sample. If we were to
%     view the Milky Way at the same inclination angle to the disk as we see
%     M31, this cutoff in projected distance would correspond roughly to the
%     inner distance limits in the M31 halo sample.'
% }


\subsection{Star particles are not globular clusters}
\label{sec:discussion_mass_excess}
A major improvement to our classification of star particles as GC candidates
could be made by explicitly taking into the formation efficiency of bound star
clusters \citep{2012MNRAS.426.3008K}, which depends on the local ISM properties
(density, pressure, sound speed). However, the only information available in
post-processing are the properties of the star particles recorded in the first
snapshot after formation. We could attempt to obtain the properties of the nearest
gas cell, but this would have to be limited to a very narrow time window between
the time of formation and the time of the snapshot because the star particle could
otherwise be in a completely different location than where it was born initially.
Moreover, even if the birth time and snapshot time would coincide, it would not
be obvious that the gas cell will yield the correct quantities as the formation
of the star itself alters the properties of the gas. We therefore did not further
attempt a more sophisticated modeling of the cluster formation efficiency in
post-processing.

Furthermore, we compare star particles to present-day globular clusters, thereby
ignoring the effects of (dynamical) disruption of globular clusters over nearly
a Hubble time. As shown by \citep{2018MNRAS.475.4309P}, a detailed model of the tidal
history of star clusters requires a temporal resolution of order mega year. For
the Auriga L4 simulations we have a total of 128 snapshots available over the
age of the Universe, thus, far too coarse temporal resolution for meaningful
calculations of tidal disruption along the orbit of our GC candidates. Therefore
we can only combine these two shortcomings of our analysis into the `mass excess',
which includes the cluster formation efficiency and consecutive globular cluster
disruption.


% \todo[inline]{
% \citet{2019MNRAS.486.3134K} ``Due to the multitude of low-mass progenitors and
% rapid merging during this early phase of galaxy growth, we recommend to reserve
% the term `ex situ GCs' for accretion events taking place at $z < 2$, when the
% central spheroid has formed and accretion events unambiguously contribute to the
% spatially extended (`halo') GC population.''
% }
%
%
% \citet{2016ApJ...824...42C}
% ``We find that the clusters can be divided into three major metallicity groups
% based on their radial distributions: (1) an inner metal-rich group ([Fe/H] $> -0.4$);
% (2) a group with intermediate metallicity (with median [Fe/H] $= -1$); and (3)
% a metal-poor group, with [Fe/H] $< -1.5$.''
%
% \citet{2016ApJ...824...42C}
% ``Although this disk mode of GCs is intriguing, the very dominant subpopulations
% in M31 are the lower-metallicity GCs. It seems likely that these objects were
% brought in by the accretion of satellite galaxies (minor mergers). The populous
% GC system of M31, relative to the MW, would then reflect a more active accretion
% history, and the radial gradient of GC metallicities could arise through
% correlations of both metallicity and dynamical friction with satellite galaxy
% mass (e.g., Amorisco 2016).''



\section{Summary and conclusions}
\label{sec:conclusions}

We investigate GC candidates in the Auriga simulations and draw the following
conclusions.

\begin{itemize}
    \item The star formation model implemented in the Auriga simulations produces
    metallicity distributions that are more metal-rich the Milky Way and M31
    globular cluster systems.
    \item We reject the null hypothesis $H_0$ that MW (M31) and the GC candidates
    (all, insitu, and accreted) are drawn from the same underlying distribution
    for all Auriga L4 galaxies with $\geq 98.68$~\% confidence.
    \item GC candidates in the Auriga simulations may be found out to $R_{200}$,
    given our selection function of star particles selects all stars bound to the
    main subhalo in the main halo. The stellar mass is dominated by accreted star
    particles at radii beyond $20$~kpc The GCs in the MW and M31, on the other hand,
    have a much smaller radial extend.
    \item The cluster formation efficiency would have to increase with decreasing
    metallicity for GC candidates in the Auriga simulations to be consistent
    with the Milky Way GC system, given that we expect dynamical evolution to more strongly affect GC
    candidates at smaller radii.
    \item The Auriga simulations are not consistent with the picture that all
    metal-rich GCs formed insitu and that all metal-poor GCs were accreted.
\end{itemize}


\section*{Acknowledgements}
TLRH acknowledges support from the International Max-Planck Research School (IMPRS) on Astrophysics.


\todo[inline]{Check Auriga boilerplate that we need to acknowledge}
RG and VS acknowledge support by the DFG Research Centre SFB-881 `The
Milky Way System' through project A1. This work has also been
supported by the European Research Council under ERC-StG grant
EXAGAL- 308037. Part of the simulations of this paper used the
SuperMUC system at the Leibniz Computing Centre, Garching,
under the project PR85JE of the Gauss Centre for Supercomputing.
This work used the DiRAC Data Centric system at Durham
University, operated by the Institute for Computational Cosmology
on behalf of the STFC DiRAC HPC Facility `www.dirac.ac.uk'.
This equipment was funded by BIS National E-infrastructure capital
grant ST/K00042X/1, STFC capital grant ST/H008519/1 and
STFC DiRAC Operations grant ST/K003267/1 and Durham University.
DiRAC is part of the UK National E-Infrastructure.

%%%%%%%%%%%%%%%%%%%%%%%%%%%%%%%%%%%%%%%%%%%%%%%%%%

%%%%%%%%%%%%%%%%%%%% REFERENCES %%%%%%%%%%%%%%%%%%

% The best way to enter references is to use BibTeX:

\bibliographystyle{mnras}
\bibliography{AurigaGCS} % if your bibtex file is called example.bib


% Alternatively you could enter them by hand, like this:
% This method is tedious and prone to error if you have lots of references
%\begin{thebibliography}{99}
%\bibitem[\protect\citeauthoryear{Author}{2012}]{Author2012}
%Author A.~N., 2013, Journal of Improbable Astronomy, 1, 1
%\bibitem[\protect\citeauthoryear{Others}{2013}]{Others2013}
%Others S., 2012, Journal of Interesting Stuff, 17, 198
%\end{thebibliography}

%%%%%%%%%%%%%%%%%%%%%%%%%%%%%%%%%%%%%%%%%%%%%%%%%%

%%%%%%%%%%%%%%%%% APPENDICES %%%%%%%%%%%%%%%%%%%%%
\clearpage
\appendix
\section{Scatter between individual Auriga haloes, and numerical convergence}
\label{sec:scatter-convergence}
We check whether the properties of the Auriga globular cluster candidates
are well converged between the three different resolution levels used for the
Auriga simulations. Here we consider all three Auriga haloes for which simulation
runs were performed at all three resolution levels: Au6, Au16, and Au24. Here
we can investigate differences between individual haloes.

Figure~\ref{fig:logMFeH_res} shows the mass-weighted metallicity distribution,
Figure~\ref{fig:logMRgc_res} shows the mass-weighted radial distribution, and
Figure~\ref{fig:logMRgcFeH_res}



\begin{figure*}
    \includegraphics[width=0.31\textwidth]{{logMFeH_Au6}.png}
    \includegraphics[width=0.31\textwidth]{{logMFeH_Au16}.png}
    \includegraphics[width=0.31\textwidth]{{logMFeH_Au24}.png}
    \caption{Similar to Figure~\ref{fig:FeH}, but showing one individual Auriga
        halo, where colours indicate the resolution level: L3 green, L4 orange,
        and L5 blue. \emph{Left:} Auriga halo 6. \emph{Mid:} Auriga halo 16.
        \emph{Right:} Auriga halo 24. For all three haloes we find marginal
        increases in the mass normalization with increasing resolution level.
        \label{fig:logMFeH_res}
    }

\end{figure*}
\begin{figure*}
    \includegraphics[width=0.31\textwidth]{{logMRgc_Au6}.png}
    \includegraphics[width=0.31\textwidth]{{logMRgc_Au16}.png}
    \includegraphics[width=0.31\textwidth]{{logMRgc_Au24}.png}
    \caption{Similar to Figure~\ref{fig:Rgc}, but showing one individual Auriga
        halo, where colours indicate the resolution level: L3 green, L4 orange,
        and L5 blue. \emph{Left:} Auriga halo 6. \emph{Mid:} Auriga halo 16.
        \emph{Right:} Auriga halo 24. For all three haloes we find marginal
        increases in the mass normalization with increasing resolution level.
        \label{fig:logMRgc_res}
    }

\end{figure*}
%%%%%%%%%%%%%%%%%%%%%%%%%%%%%%%%%%%%%%%%%%%%%%%%%%


% Don't change these lines
\bsp    % typesetting comment
\label{lastpage}
\end{document}

% End of mnras_template.tex
