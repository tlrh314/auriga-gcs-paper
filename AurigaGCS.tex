% mnras_template.tex
%
% LaTeX template for creating an MNRAS paper
%
% v3.0 released 14 May 2015
% (version numbers match those of mnras.cls)
%
% Copyright (C) Royal Astronomical Society 2015
% Authors:
% Keith T. Smith (Royal Astronomical Society)

% Change log
%
% v3.0 May 2015
%    Renamed to match the new package name
%    Version number matches mnras.cls
%    A few minor tweaks to wording
% v1.0 September 2013
%    Beta testing only - never publicly released
%    First version: a simple (ish) template for creating an MNRAS paper

%%%%%%%%%%%%%%%%%%%%%%%%%%%%%%%%%%%%%%%%%%%%%%%%%%
% Basic setup. Most papers should leave these options alone.
\documentclass[a4paper,fleqn,usenatbib]{mnras}

% MNRAS is set in Times font. If you don't have this installed (most LaTeX
% installations will be fine) or prefer the old Computer Modern fonts, comment
% out the following line
\usepackage{newtxtext,newtxmath}
% Depending on your LaTeX fonts installation, you might get better results with one of these:
%\usepackage{mathptmx}
%\usepackage{txfonts}

% Use vector fonts, so it zooms properly in on-screen viewing software
% Don't change these lines unless you know what you are doing
\usepackage[T1]{fontenc}
\usepackage{ae,aecompl}


%%%%% AUTHORS - PLACE YOUR OWN PACKAGES HERE %%%%%

% Only include extra packages if you really need them. Common packages are:
\usepackage{graphicx}    % Including figure files
\usepackage{amsmath}    % Advanced maths commands
\usepackage{amssymb}    % Extra maths symbols

\usepackage{todonotes}
\graphicspath{{img/}}   % Set image path

%%%%%%%%%%%%%%%%%%%%%%%%%%%%%%%%%%%%%%%%%%%%%%%%%%

%%%%% AUTHORS - PLACE YOUR OWN COMMANDS HERE %%%%%

% Please keep new commands to a minimum, and use \newcommand not \def to avoid
% overwriting existing commands. Example:
%\newcommand{\pcm}{\,cm$^{-2}$}    % per cm-squared
\newcommand{\Sun}[0]{\ensuremath{_{\odot}}}
\renewcommand{\deg}{\ensuremath{^{\circ}}}


%%%%%%%%%%%%%%%%%%%%%%%%%%%%%%%%%%%%%%%%%%%%%%%%%%

%%%%%%%%%%%%%%%%%%% TITLE PAGE %%%%%%%%%%%%%%%%%%%

% Title of the paper, and the short title which is used in the headers.
% Keep the title short and informative.
\title[Auriga GCS]{The Globular Cluster System of the Auriga Simulations}

% The list of authors, and the short list which is used in the headers.
% If you need two or more lines of authors, add an extra line using \newauthor
\author[T. L. R. Halbesma et al.]{\parbox[t]{\textwidth}{
    Timo L. R. Halbesma$^{1}$\thanks{E-mail: Halbesma@MPA-Garching.MPG.DE},
    Philipp Busch$^{1}$,
    Wilma Trick$^{1}$,
    Robert J. J. Grand$^{1}$, 
    Volker Springel$^{1}$, 
    Facundo A. G\'{o}mez$^{2,3}$, 
    Federico Marinacci$^{4,5}$,
    R\"{u}diger Pakmor$^{1}$, 
    Simon D. M. White$^{1}$
} \vspace{10pt} \\
$^{1}$ Max-Planck-Institut f\"ur Astrophysik, Postfach 1317, D-85741 Garching, Germany \\
$^{2}$ Instituto de Investigaci\'{o}n Multidisciplinar en Ciencia yTecnolog\'{i}a, 
    Universidad de La Serena, Ra\'{u}l Bitr\'{a}n 1305, La Serena, Chile \\
$^{3}$ Departamento de F\'{i}sica y Astronom\'{i}a, Universidad de La Serena, Av.
    Juan Cisternas 1200 N, La Serena, Chile \\
$^{4}$ Department of Physics, Kavli Institute for Astrophysics and Space Research,
    MIT, Cambridge, MA 02139, USA \\
$^{5}$ Harvard-Smithsonian Center for Astrophysics, 60 Garden Street, Cambridge,
    MA 02138, USA \\
}

% These dates will be filled out by the publisher
\date{Accepted XXX. Received YYY; in original form ZZZ}

% Enter the current year, for the copyright statements etc.
\pubyear{2019}

% Don't change these lines
\begin{document}
\label{firstpage}
\pagerange{\pageref{firstpage}--\pageref{lastpage}}
\maketitle

% Abstract of the paper
\begin{abstract}
We investigate whether the galaxy formation model used for the Auriga simulations 
can produce a realistic globular cluster population at redshift zero. We compare
properties of the simulated star particles in the Auriga haloes with
catalogues of observations of the Milky Way globular cluster population available
in the literature. We find that the Auriga simulations produce sufficient mass
at radii and metallicities that are typical for the MW GCS, although we observe
a varying mass-excess for the different $r_{\text{gal}}$-[Fe/H] bins. This implies
different values for the combined product of the bound cluster formation efficiency
and the globular cluster disruption rate. We investigate wether these differences
could result from formation insitu vs. accreted star particles. We find ...
\end{abstract}

% Select between one and six entries from the list of approved keywords.
% Don't make up new ones.
\begin{keywords}
methods: numerical -- galaxies: formation -- galaxies: star clusters: general.
\end{keywords}

%%%%%%%%%%%%%%%%%%%%%%%%%%%%%%%%%%%%%%%%%%%%%%%%%%

%%%%%%%%%%%%%%%%% BODY OF PAPER %%%%%%%%%%%%%%%%%%

\section{Introduction}
\todo[inline]{Paragraph: General introduction of GCs}

\citet{2005MNRAS.364..367D}: ``The radial profile of the stellar halo and metal-poor globular
clusters of the Milky Way suggest that these components formed in rare early peaks above $2.5 \sigma$ at redshift above 10. ''

\citet{2017MNRAS.465.3622R}: ``GCs among oldest astrophysical objects. GCs form in the early Universe in highest density peaks \citep[e.g.][]{2005MNRAS.364..367D, 2009ApJ...706L.192B}''




\begin{itemize}
    \item ``Hence, they witness most of the formation and evolution processes of galaxies, and can be used to probe them'' \citep{2006ARA&A..44..193B}
    \item ``colour bimodality, blue and red clusters \citep[e.g.][]{1985ApJ...293..424Z, 1999AJ....118.1526G, 2001AJ....121.2974L, 2006ApJ...639...95P}
    \item ``blue metal-poor (with distribution peaking at [Fe/H] $\approx -1.5$ for the Milky Way), no sign of rotation as a population (..) more metal-rich (peak at [Fe/H] $\approx -0.5$ in the Milky  Way) more spatially concentrated and rotating with the galaxy.'' \citep{1996AJ....112.1487H}
\end{itemize}
 
 


\todo[inline]{Paragraph: ``Bimodality suggests two formation mechanisms''. In-situ vs. Accreted}
\begin{itemize}
    \item ``Blue clusters from in early Universe in galaxies that merge later. In (wet) merger process starbursts generate red population \cite{1992ApJ...384...50A, 1987nngp.proc...18S}''
    \item ``\citet{1997AJ....113.1652F} propose instead that blue globulars form when the protogalaxy itself collapses, in a metal-poor and turbulence media. The red population would form later, once the galactic disc has settled. The formation of globular clusters would then be a multiphase process, with the first phase being interrupted possibly by cosmic reionization \citep{2002MNRAS.333..383B}.''
    \item ``\citet{2005ApJ...623..650K, 2014ApJ...796...10L} advocate
that major mergers are at the origin of both sub-populations: blue
clusters form during early mergers (z > 4) while the red ones appear
in mergers at lower redshifts (even after z = 1). Although, this
scenario, combined with star formation enhancement in mergers,
seems appropriate in dense galactic environment leading to the
assembly of massive elliptical galaxies, like in the Virgo Cluster
as tested by \citet{2014ApJ...796...10L}, it does not apply to Milky Way-
like systems where no recent major merger took place (Wyse 2001;
Deason et al. 2013; Ruchti et al. 2014, 2015).
\item ``\citet{1998ApJ...501..554C} argue that red clusters form in situ while the blue ones are accreted, either via merging satellite galaxies, or by tidal capture of the clusters themselves \citep[see also][]{2013ApJ...762...39T}.''
\end{itemize}


Dark Matter - GC connection

\todo[inline]{Paragraph: scientific motivation}
\begin{itemize}
    \item The star formation model implemented in the Auriga simulations is capable of producing a suite/population of realistic Milky Way-like galaxies at redshift zero.
    \item (However) State of the art simulations still face numerical restrictions that requires a subgrid approach to star formation and feedback because individual stars (and their evolution) cannot yet be accounted for. Star formation thus occurs in a heuristic/probabilistic fashion for gas cells that fulfill some star formation criterion. The star particles enrich the gas with metals and energy, both according to pre-defined/pre-calculated yields for specific feedback processes (supernova type I and II, strong stellar winds of asymptotic giant branch stars, ). In addition, black holes are seeded in eligible haloes to account for feedback associated with an active galactic nucleus.
    \item The end-result of the star formation model is the production of simulated Milky Way-like galaxies. Therefore the question naturally arises whether or not the Auriga simulations are also capable of faithfully producing a globular cluster population as observed in the Milky Way.
    \item Globular cluster formation in cosmological zoom simulations is very interesting for two reasons. First of all, extragalactic observations typically show the integrated properties of globular clusters rather than that of the individual stars within the clusters. Moreover, the typical mass scale of globular clusters is comparable to the numerical (mass) resolution of cosmological zoom simulations. The detailed small scale physics that is at play for real world globular clusters appears in observations as the combined effect of the $10^{3-6}$ M\Sun, compared to a mass resolution of $10^{3-5}$ M\Sun for the Auriga simulations. Globular clusters can therefore serve as an ultimate test to the star formation model that is implemented in the numerical simulations. Secondly, cosmological zoom simulations provide an accurate recording of the full and detailed merger history of the simulated galaxy. This is important because theoretical paradigms for globular cluster formation in the literature know two distinct classes of GCs that are separated by their exact formation sites: an in-situ versus an accreted population. Cosmological zoom simulations are uniquely allow for an investigation into globular cluster formation with particular focus on the in-situ and accreted populations.
\end{itemize}



\todo[inline]{Paragraph: previous work / work of other groups}
This goes before the GC formation mechanism paragraph.
\begin{itemize}
\item Origin of the Milky Way globular clusters \citep{2017MNRAS.465.3622R}
\item GCs in FIRE \citep{2018MNRAS.474.4232K}
\item EMOSAICS project \citep{2018MNRAS.475.4309P}
\item Origin of GC bimodality? \citep{2018MNRAS.479..200F}
\item \textit{GAIA} DR2: GC kinematics \citep{2018A&A...616A..12G}, Dating GC Tidal Disruption \citep{2018ApJ...859L..13B}
\item GC in N-body simulation \citep{2018ApJ...861...69C}
\item Tangentially related? role of GC mass evolution on stream properties \citep{2018MNRAS.474.2479B}
\item GC formation from dwarfs to giants \citep{2018MNRAS.480.2343C}
\item GC contribution to EOR \citep{2018MNRAS.479..332B}
\item Early Universe supermassive star / GC formation \citep{2018MNRAS.478.2461G}
\item GC formation in cold filaments \citep{2018ApJ...861..148M}
\item GC formation in high-redshift dwarf galaxies \citep{2018MNRAS.477..480Z}
\item GCs in MW outer region \citep{2017arXiv170804542P}
\item Impact of the Cutoff of the Cluster Initial Mass Function \citep{2018arXiv181001888C}
\item Metallicity gradients in the globular cluster systems of early-type galaxies: in situ and accreted components \citep{2018MNRAS.479.4760F}
\item Globular clusters in M31, Local Group, and external galaxies \citep{2016IAUS..317..120L}
\item Globular Clusters Formed within Dark Halos I: present-day abundance, distribution and kinematics \citep{2019MNRAS.482..219C}
\item The mass of the Milky Way from satellite dynamics \citep{2018arXiv180810456C}
\item Globular cluster formation and evolution in the context of cosmological galaxy assembly: open questions \citep{2018RSPSA.47470616F}
\item The kinematics of globular clusters systems in the outer halos of the Aquarius simulations \citep{2016A&A...592A..55V}
\item Star Cluster Formation in Cosmological Simulations \citep{2017ApJ...834...69L, 2018ApJ...861..107L, 2018arXiv181011036L}
\item A systematic analysis of star cluster disruption by tidal shocks -- I. Controlled N-body simulations and a new theoretical model \citep{2018arXiv181200014W}
\item Spatial mixing of binary stars in multiple-population globular clusters \citep{2018MNRAS.tmp.3147H}
\item Star Clusters Across Cosmic Time \citep{2018arXiv181201615K}
\item Kinematics of Subclusters in Star Cluster Complexes: Imprint of their Parental Molecular Clouds \citep{2018arXiv181201858F}
\item Investigating the population of Galactic star formation regions and star clusters within a Wide-Fast-Deep Coverage of the Galactic Plane  \citep{2018arXiv181203025P}
% TODO: last checked arXiv Dec 15, 2018
\end{itemize}


\todo[inline]{Paper outline}
We summarise the relevant characteristics of the Auriga simulations in 
section~\ref{sec:auriga}, followed by a summary of the observations of the
Milky Way (MW) globular cluster system (GCS) in section~\ref{sec:observations}
that we use to compare our simulations to in section~\ref{sec:results}. We
discuss our findings in section~\ref{sec:discussion} to come to our conclusions 
in section~\ref{sec:conclusions}.


\section{The Auriga simulations}
\label{sec:auriga}
We use the Auriga simulations \citep[][hereafter G17]{2017MNRAS.467..179G}, a 
suite of high-resolution cosmological zoom simulations ran with a galaxy
formation model that produces realistic Milky Way-like galaxies at redshift
$z=0$. The simulations are performed with the state-of-the art code \textsc{arepo}
\citep{2010MNRAS.401..791S, 2016MNRAS.455.1134P} that solves the 
magnetohydrodynamical equations on a moving mesh.

The interstellar medium is modelled using a sub-grid approach which implements
the physical processes most relevant to galaxy formation and evolution. 
This model was tailored to the \textsc{arepo} code and calibrated to reproduce
key observables of galaxies, such as the history of the cosmic star formation rate 
density, the stellar mass to halo mass relation, and galaxy luminosity functions.

The sub-grid includes primordial and metal-line cooling with self-shielding corrections.
Reionization is completed at redshift six by a time-varying spatially uniform UV 
background \citep{2009ApJ...703.1416F, 2013MNRAS.436.3031V}
The interstellar medium is described by an equation of state for a two-phase medium
in pressure equilibrium \citep{2003MNRAS.339..289S} with stochastic star formation
in thermally unstable gas with a density threshold of $n = 0.13 \text{cm}^{-3}$,
and consecutive stellar evolution is accounted for. Stars provide feedback by 
stellar winds \citep{2014MNRAS.437.1750M, 2017MNRAS.467..179G}, and further 
enrich the ISM with metals from SNIa, SNII, and AGB stars \citep{2013MNRAS.436.3031V}.
The formation of black holes is modelled which results in feedback from active 
galactic nuclei \citep{2005MNRAS.361..776S, 2014MNRAS.437.1750M, 2017MNRAS.467..179G}.
Finally, the simulations follow the evolution of a magnetic field of $10^{-14}$
(comoving)~G seeded at $z = 127$ \citep{2013MNRAS.432..176P, 2014ApJ...783L..20P}.
See G17 for further details of the numerical setup as well as the galaxy formation
model.

\todo[inline]{
``The diversity in morphological properties of these simulated galaxies reflects
the stochasticity inherent to the process of galaxy formation and evolution 
\citep[e.g.][]{2005ApJ...635..931B, 2010MNRAS.406..744C, 2010ApJ...708.1398T}.''
}

\subsection{Definition of stellar halo}
\todo[inline]{Possibly also relevant here. See \citet{2018arXiv180407798M}.}


\subsection{Definition of accreted and in-situ component}
\todo[inline]{Possibly also relevant here. See \citet{2018arXiv180407798M}.}



\section{Relevant observational data}
\label{sec:observations}
The galaxy formation model implemented for the Auriga simulations produces
realistic spiral galaxies at redshift zero. Therefore we aim to compare the
simulations to the globular cluster systems of the spiral galaxies in the 
Local Group (i.e. the Milky Way and Andromeda galaxy or M31). Here we summarise 
the relevant data available in the literature.

\subsection{Milky Way: Harris catalogue}
\label{sec:harris}
\citet[][2010 edition; hereafter H96]{1996AJ....112.1487H} provides a fairly
up-to-date, comprehensive catalogue of the Milky Way globular cluster system and
contains properties of 157 globular clusters. The catalogue is believed to be 
roughly 90\% complete. The relevant data fields that we use from H96 are the 
metallicity [Fe/H], galactocentric radius $r_{\text{gal}}$, and absolute magnitude 
in the V-band $M_V$. 
\todo[inline]{
We use the latter to calculate a mass-estimate by assuming
$M_{V,\odot}=4.83$ and a mass to light ratio $M/L_V = 1.7$~M/L$_{\odot}$ 
(the mean for MW clusters McLaughlin \& van der Marel 2005). 
}

In Fig.~\ref{fig:MW-RgcFeH} we show a two-dimensional mass-weighted distribution,
splitting the data up in bins of both radius and metallicity. Later on we 
investigate whether the star formation model implemented for the Auriga simulations 
can produce a sufficient total mass in globular cluster candidates at the right
values of radius, metallicity, and radius-metals, see Sec.~\ref{sec:RgcFeH}

\begin{figure}
    \includegraphics[width=\columnwidth]
        {{MW_M31_RgcFeH_HistogramMassWeighted_Harris1996ed2010data}.png}
    \caption{
        Mass-weighted r$_{\text{gc}}$-[Fe/H] distribution of
        151 GCs in the MW \citep[data from][2010 ed.]{1996AJ....112.1487H}, which
        is 98.19 \% of the total MW GCS mass.
        \label{fig:MW-RgcFeH}
    }
\end{figure}

\subsection{Milky Way: VandenBerg catalogue}
\label{sec:vandenberg}
\citet[][hereafter V13]{2013ApJ...775..134V} used photometric data to obtain 
[Fe/H] measurements and age-estimates for 55 globular clusters in the Milky Way.
The mean value of the age-estimates in this data set is $11.9 \pm 0.1$ Gyr and the
dispersion is 0.9 Gyr. Furthermore, only one of the 55 GC age-estimates is
below 10 Gyr.


\todo[inline]{
How is this particular sample selected? What biases does this introduce?
Is this sample drawn from the same underlying distribution as the Harris sample 
(plot distributions of FeH, compare mean/std, do t-test)
}


\subsection{Andromeda: Caldwell catalogue}
\label{sec:m31}
\citet[][, hereafter C11]{2011AJ....141...61C} studied globular clusters in the
Andromeda galaxy. The relevant fields in this data set are the age, metallicities,
and absolute visual magnitude.
\todo[inline]{
`For M31 GC masses we combine the catalogues of Caldwell et al. (2011, using the 
given masses) and Huxor et al. (2014, again assuming $M/L_V = 1.7$~M/L$_{\odot}$, 
e.g. Strader, Caldwell \& Seth 2011).'
}
For M31 we find an age distribution with a mean value of 10.8$\pm 10.8$~Gyr and a 
dispersion of 2.3~Gyr. Furthermore, 27 GCs have age-estimates below 10~Gyr, and
the minimum age is 4.8~Gyr. Perhaps an age cut of 6~Gyr would would be more 
appropriate for M31, see Fig.~\ref{fig:MW-M31-age}.

\begin{figure}
    \includegraphics[width=\columnwidth]{{MW_M31_Age_HistogramMassWeighted}.png}
    \caption{
        Mass-weighted age distribution of 55 GCs in the MW
        \citep[data from][]{2013ApJ...775..134V} and 87 GCs
        in M31 \citep[data from][]{2011AJ....141...61C}.
        \label{fig:MW-M31-age}
    }
\end{figure}

\todo[inline]{
Furthermore, we supplement observational data of globular clusters in M31 with 
\citet{2014MNRAS.442.2165H, 2014MNRAS.442.2929V}
}


\section{Results}
\label{sec:results}
We investigate globular cluster candidates in the Auriga simulations using an age
cut to select of stars that formed at a lookback time greater than $10$~Gyr. We 
chose this particular value because the mean age-estimate of the $55$ in V13 is
$11.9\pm0.1$~Gyr with a dispersion of $0.8$~Gyr, and only one observed globular 
cluster falls below this selection criterion. 

In Sec.~\ref{sec:metallicity} we investigate the metallicity distribution of the
GC candidates, in Sec.~\ref{sec:spatial} we show the distribution of galactocentric
radii of the globular cluster candidates within the Auriga simulations, and we 
combine both in Sec.~\ref{sec:RgcFeH}. We continue our analysis with a deeper 
analysis of the properties of the proto-galaxies at times of birth of the accreted 
globular cluster population in Sec.~\ref{sec:birth-properties-of-accreted-gc-candidates}


\subsection{Metallicity distribution}
\label{sec:metallicity}
\todo[inline]{
Can the Auriga simulations produce star particles of $>10$~Gyr (GC candidates) 
with a metallicity distribution that is consistent with the MW GCS?
}

In Fig.~\ref{fig:FeH} we show a mass-weighted metallicity distribution of star 
particles in the Auriga simulations. We show the median value of all Auriga 
haloes for all stars (orange dotted lines) and globular cluster candidates 
orange solid). The latter sub set is further split up between stars that formed 
in-situ (red solid), and those that were accreted (blue solid). The shaded regions 
indicate the $1\sigma$ interval. The MW GCS is shown in purple and that of M31 
in pink. We use the same bin sizes for the simulations as for the observations, 
explicitly plotted for the observed profiles.

\begin{figure}
    \includegraphics[width=\columnwidth]{{logMFeH_noStepsMid-trim}.png}
    \caption{
        Mass-weighted metallicity distribution of star particles in the Auriga 
        simulations. We show the median value of all Auriga haloes for all
        stars (orange dotted) and globular cluster candidates (i.e. stars with 
        age $>$~10~Gyr; orange solid). The latter sub set is further split up
        between stars that formed in-situ (red solid), and those that were accreted
        (blue solid). Shaded regions indicate the $1\sigma$ interval. The MW GCS
        is shown in purple and that of M31 in pink. We use the same bin sizes
        for the simulations as for the observations, explicitly plotted for the
        observed profiles.
        \label{fig:FeH}
    }
\end{figure}

We find that the metallicity range $-3 <$ [Fe/H] $< -1$ is only populated by 
star particles older than ten Gyr, while [Fe/H] $> -1$ is dominated by star 
particles younger than 10~Gyr (with a difference of two orders of magnitude). 
Furthermore, the old accreted star particles contribute most significantly to 
the range $-3 <$ [Fe/H] $< -1$, and the contribution of the old in-situ star
particles at these metallicities declines steeper with declining metallicity 
than the old accreted population. The old in-situ population provides the 
dominant contribution of the old population for [Fe/H] $> -1$.

We notice a scatter\footnote{Here scatter means the width of the 1$\sigma$ interval.
The difference between the minimum and maximum values is $0.9$~dex.} in mass 
normalization of $0.3$ dex between the different Auriga galaxies. We investigate 
whether there is a correlation between the total stellar mass in (old) star particles
and the virial mass\footnote{The virial mass is defined as the mass contained inside the
radius $r_{200,c}$ at which the average (spherical) mass density equals two hundred 
times the critical density of the Universe} $M_{200,c}$ of the host halo.
We fit $a \cdot \left(\text{M}_{200,c}[1\text{e}10 \text{M\Sun}] - 140\right) + b$
to log$_{10}\left(\Sigma_{i} m_i(\text{\scriptsize [Fe/H] = -3})\right)$.
% \limits_{i, \text{if age(i) $>$ 10~Gyr}}
Our fit thus provides the slope $a$ and normalization of (old) stellar mass $b$
at [Fe/H]=-3 for $M_{200,c}$ at 140e10~M\Sun. We find $a = 0.00388$, $b = 6.597$
and conclude that there is a small positive correlation of mass in (old) star 
particles with the virial mass of the host halo.

\begin{figure}
\includegraphics[width=0.49\textwidth]{{logMFeH_OldStars_normalization_fit3}.png}
    \caption{
        Plot of log$_{10}$(mass normalization) [i.e. $b$ obtained above] against 
        the virial mass M$_{200,c}$ of the Auriga haloes. We fit a linear relation 
        to see whether there is a correlation, and find $a = 0.00388$, $b = 6.597$. 
        % Colours indicate resolution: L3 green, L4 orange, and L5 blue.
        The red dotted line shows the `central' value of M$_{200,c}$, the yellow 
        region shows the 1$\sigma$ interval around the best-fit relation, and the
        blue dashed lines shows the intrinsic scatter. The error bars show 1\% 
        of the obtained values.
        \label{fig:Mnorm-fit}
    }
\end{figure}

Furthermore we compare the simulated metallicity distribution to observations
of the Milky Way and M31.
\todo[inline]{
    yes/no bimodal
    
    M31 slope between -3 and -1 consistent with old insitu?

    MW above 0 no more GCs, but Auriga well populated (mostly insitu, but
    also too many accreted.

    should also comment on all vs old (Auriga simulation) btw

}





bla. Conclusion: too many metal-rich star particles.



\subsection{Spatial distribution}
\label{sec:spatial}
Is the spatial distribution of the GC candidates in the Auriga simulations consistent with the MW GCS?


\todo[inline]{
Look into Pandromeda survey. Star counts, very wide angle survey
}

\begin{figure}
    \includegraphics[width=\columnwidth]{{logMRgc_noStepsMid-trim}.png}
    \caption{
        Mass-weighted distribution of galactocentric radii at which star 
        particles in the Auriga simulations are found.
        \label{fig:Rgc}
    }
\end{figure}


\subsection{Radius-metallicity distribution}
\label{sec:RgcFeH}
What age-metallicity distribution is produced by star formation events in the Auriga simulations?
?
\begin{figure}
    \includegraphics[width=0.49\textwidth]
        {{Au-all_RgcFeH_HistogramMassWeighted_iold_mean}.png}
    \caption{
        Mass-weighted [Fe/H]-r$_{\text{gc}}$ distribution of all Auriga haloes 
        (level 3, 4 and 5). Here we consider the old ($>$~10~Gyr) stars in all 
        simulations and color-code the \textbf{mean value} (of 40 Auriga haloes)
        \label{fig:Au-RgcFeH}
    }
\end{figure}


\begin{figure}
    \includegraphics[width=\columnwidth]
        {{logMRgcFeH_accreted-insitu-notitle}.png}
    \caption{
        OMG what a beautiful plot - now only need 1$\sigma$ interval around both
        relations, and need to know exactly what model I fit...
        \label{fig:todo2}
    }
\end{figure}


\subsection{Properties of birth haloes of the accreted population}
\label{sec:birth-properties-of-accreted-gc-candidates}



\subsection{Age-metallicity distribution}
\label{sec:agemetallicity}
What age-metallicity distribution is produced by star formation events in the Auriga simulations?




\subsection{Mass budget} 
\label{sec:mass}
Does the star formation model implemented in the Auriga simulations produce sufficient mass in star particles with properties that are consistent with the MW GCS?

What efficiencies could we afford if we would take into account the combined mass loss effect of converting from star particles to bound star clusters and globular cluster disruption?

How does the total stellar mass in $r_\textbf{gal}$-[Fe/H] bins compare to the MW GCS?



\subsection{Formation history} 
\label{sec:history}
Can we identify particular star formation events that generate GC candidates with the correct age, metallicity, and radial properties as expected or the MW GCS?

Can we distinguish between particles that have formed in-situ and those that have been accreted? Can we identify specific features in the age-metallicity plane, or in the $r_\textbf{gal}$-[Fe/H] plane, that result from one of both populations? How does this connect to proposed mechanisms for globular cluster formation in the literature?


Orbits: are the pericentres different? Look at velocity + specific angular momentum distribution in the different FeH/Rgc bins as proxy for the pericenter




\section{Discussion}
\label{sec:discussion}

We investigate all star particles in the Auriga simulations that are older than 10~Gyr,
an approach equal to the method of \citep{2017MNRAS.465.3622R}. This approach does not
take the bound cluster fraction \citep[e.g.][]{2012MNRAS.426.3008K} into account. This
means that our sub set, which is based on selection by age, comprises both stars in the
field as well as globular clusters. We compare the total mass in the simulations in 
metallicity ([Fe/H]), galactocentric radius $r_{\text{gc}}$, and [Fe/H]-$r_{\text{gal}}$ 
bins to the total mass in the MW GCS (using the H96 data set). The mass excess in the
simulations gives a maximum mass loss `budget' for the product of cluster formation 
efficiency and dynamical evolution. We find fractions that vary with metallicity,
radius and metallicity-and-radius.

\todo[inline]{
``The fraction of all star formation that occurs in bound stellar clusters (the cluster formation efficiency, hereafter CFE) follows by integration of these local clustering and survival properties over the full density spectrum of the ISM, and hence is set by galaxy-scale physics. We derive the CFE as a function of observable galaxy properties, and find that it increases with the gas surface density'' \citep{2012MNRAS.426.3008K}
}


\section{Summary and conclusions}
\label{sec:conclusions}


\section*{Acknowledgements}
TLRH acknowledges support from the International Max-Planck Research School (IMPRS) on Astrophysics.

\todo[inline]{Check Auriga boilerplate that we need to acknowledge}
RG and VS acknowledge support by the DFG Research Centre SFB-881 `The
Milky Way System' through project A1. This work has also been
supported by the European Research Council under ERC-StG grant
EXAGAL- 308037. Part of the simulations of this paper used the
SuperMUC system at the Leibniz Computing Centre, Garching,
under the project PR85JE of the Gauss Centre for Supercomputing.
This work used the DiRAC Data Centric system at Durham
University, operated by the Institute for Computational Cosmology
on behalf of the STFC DiRAC HPC Facility `www.dirac.ac.uk'.
This equipment was funded by BIS National E-infrastructure capital 
grant ST/K00042X/1, STFC capital grant ST/H008519/1 and
STFC DiRAC Operations grant ST/K003267/1 and Durham University. 
DiRAC is part of the UK National E-Infrastructure.

%%%%%%%%%%%%%%%%%%%%%%%%%%%%%%%%%%%%%%%%%%%%%%%%%%

%%%%%%%%%%%%%%%%%%%% REFERENCES %%%%%%%%%%%%%%%%%%

% The best way to enter references is to use BibTeX:

\bibliographystyle{mnras}
\bibliography{AurigaGCS} % if your bibtex file is called example.bib


% Alternatively you could enter them by hand, like this:
% This method is tedious and prone to error if you have lots of references
%\begin{thebibliography}{99}
%\bibitem[\protect\citeauthoryear{Author}{2012}]{Author2012}
%Author A.~N., 2013, Journal of Improbable Astronomy, 1, 1
%\bibitem[\protect\citeauthoryear{Others}{2013}]{Others2013}
%Others S., 2012, Journal of Interesting Stuff, 17, 198
%\end{thebibliography}

%%%%%%%%%%%%%%%%%%%%%%%%%%%%%%%%%%%%%%%%%%%%%%%%%%

%%%%%%%%%%%%%%%%% APPENDICES %%%%%%%%%%%%%%%%%%%%%
\clearpage
\appendix
\section{Scatter between individual Auriga haloes, and numerical convergence} 
\label{sec:scatter-convergence}
We check whether the properties of the Auriga globular cluster candidates 
are well converged between the three different resolution levels used for the 
Auriga simulations. Here we consider all three Auriga haloes for which simulation
runs were performed at all three resolution levels: Au6, Au16, and Au24. Here
we can investigate differences between individual haloes. 

Fig.~\ref{fig:logMFeH_res} shows the mass-weighted metallicity distribution, 
Fig.~\ref{fig:logMRgc_res} shows the mass-weighted radial distribution, and
Fig.~\ref{fig:logMRgcFeH_res}



\begin{figure*}
    \includegraphics[width=0.31\textwidth]{{logMFeH_Au6}.png}
    \includegraphics[width=0.31\textwidth]{{logMFeH_Au16}.png}
    \includegraphics[width=0.31\textwidth]{{logMFeH_Au24}.png}
    \caption{Same as Fig.~\ref{fig:FeH}, but here the colours indicate resolution
        level: L3 green, L4 orange, and L5 blue. \emph{Left:} Auriga halo 6.
        \emph{Mid:} Auriga halo 16. \emph{Right:} Auriga halo 24. For all three
        haloes we find marginal increases in the mass normalization with increasing 
        resolution level.
        \label{fig:logMFeH_res}
    }

\end{figure*}
\begin{figure*}
    \includegraphics[width=0.31\textwidth]{{logMRgc_Au6}.png}
    \includegraphics[width=0.31\textwidth]{{logMRgc_Au16}.png}
    \includegraphics[width=0.31\textwidth]{{logMRgc_Au24}.png}
    \caption{Same as Fig.~\ref{fig:Rgc}, but here the colours indicate resolution
        level: L3 green, L4 orange, and L5 blue. \emph{Left:} Auriga halo 6.
        \emph{Mid:} Auriga halo 16. \emph{Right:} Auriga halo 24. For all three 
        haloes we find marginal increases in the mass normalization with increasing 
        resolution level.
        \label{fig:logMRgc_res}
    }

\end{figure*}
%%%%%%%%%%%%%%%%%%%%%%%%%%%%%%%%%%%%%%%%%%%%%%%%%%


% Don't change these lines
\bsp    % typesetting comment
\label{lastpage}
\end{document}

% End of mnras_template.tex
