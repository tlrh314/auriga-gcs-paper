% mnras_template.tex
%
% LaTeX template for creating an MNRAS paper
%
% v3.0 released 14 May 2015
% (version numbers match those of mnras.cls)
%
% Copyright (C) Royal Astronomical Society 2015
% Authors:
% Keith T. Smith (Royal Astronomical Society)

% Change log
%
% v3.0 May 2015
%    Renamed to match the new package name
%    Version number matches mnras.cls
%    A few minor tweaks to wording
% v1.0 September 2013
%    Beta testing only - never publicly released
%    First version: a simple (ish) template for creating an MNRAS paper

%%%%%%%%%%%%%%%%%%%%%%%%%%%%%%%%%%%%%%%%%%%%%%%%%%
% Basic setup. Most papers should leave these options alone.
\documentclass[a4paper,fleqn,usenatbib]{mnras}

% MNRAS is set in Times font. If you don't have this installed (most LaTeX
% installations will be fine) or prefer the old Computer Modern fonts, comment
% out the following line
\usepackage{newtxtext,newtxmath}
% Depending on your LaTeX fonts installation, you might get better results with one of these:
%\usepackage{mathptmx}
%\usepackage{txfonts}

% Use vector fonts, so it zooms properly in on-screen viewing software
% Don't change these lines unless you know what you are doing
\usepackage[T1]{fontenc}
\usepackage{ae,aecompl}


%%%%% AUTHORS - PLACE YOUR OWN PACKAGES HERE %%%%%

% Only include extra packages if you really need them. Common packages are:
\usepackage{graphicx}    % Including figure files
\usepackage{amsmath}    % Advanced maths commands
\usepackage{amssymb}    % Extra maths symbols

\usepackage{todonotes}
\graphicspath{{img/}}   % Set image path

%%%%%%%%%%%%%%%%%%%%%%%%%%%%%%%%%%%%%%%%%%%%%%%%%%

%%%%% AUTHORS - PLACE YOUR OWN COMMANDS HERE %%%%%

% Please keep new commands to a minimum, and use \newcommand not \def to avoid
% overwriting existing commands. Example:
%\newcommand{\pcm}{\,cm$^{-2}$}    % per cm-squared
\newcommand{\Sun}[0]{\ensuremath{_{\odot}}}
\renewcommand{\deg}{\ensuremath{^{\circ}}}


%%%%%%%%%%%%%%%%%%%%%%%%%%%%%%%%%%%%%%%%%%%%%%%%%%

%%%%%%%%%%%%%%%%%%% TITLE PAGE %%%%%%%%%%%%%%%%%%%

% Title of the paper, and the short title which is used in the headers.
% Keep the title short and informative.
\title[Auriga GCS]{The Globular Cluster System of the Auriga Simulations}

% The list of authors, and the short list which is used in the headers.
% If you need two or more lines of authors, add an extra line using \newauthor
\author[T. L. R. Halbesma et al.]{\parbox[t]{\textwidth}{
    Timo L. R. Halbesma$^{1}$\thanks{E-mail: Halbesma@MPA-Garching.MPG.DE},
    Robert J. J. Grand$^{1}$, 
    Volker Springel$^{1}$, 
    Facundo A. G\'{o}mez$^{2,3}$, 
    Federico Marinacci$^{4,5}$,
    R\"{u}diger Pakmor$^{1}$, 
    Wilma Trick$^{1}$,
    Philipp Busch$^{1}$,
    Simon D. M. White$^{1}$
} \vspace{10pt} \\
$^{1}$ Max-Planck-Institut f\"ur Astrophysik, Karl-Schwarzschild-Str. 1,
    85741 Garching, Germany \\
$^{2}$ Instituto de Investigaci\'{o}n Multidisciplinar en Ciencia yTecnolog\'{i}a, 
    Universidad de La Serena, Ra\'{u}l Bitr\'{a}n 1305, La Serena, Chile \\
$^{3}$ Departamento de F\'{i}sica y Astronom\'{i}a, Universidad de La Serena, Av.
    Juan Cisternas 1200 N, La Serena, Chile \\
$^{4}$ Department of Physics, Kavli Institute for Astrophysics and Space Research,
    MIT, Cambridge, MA 02139, USA \\
$^{5}$ Harvard-Smithsonian Center for Astrophysics, 60 Garden Street, Cambridge,
    MA 02138, USA \\
}

% These dates will be filled out by the publisher
\date{Accepted XXX. Received YYY; in original form ZZZ}

% Enter the current year, for the copyright statements etc.
\pubyear{2019}

% Don't change these lines
\begin{document}
\label{firstpage}
\pagerange{\pageref{firstpage}--\pageref{lastpage}}
\maketitle

% Abstract of the paper
\begin{abstract}
\todo[inline]{Rob Grand: `for many Auriga papers Carlos Frenk and Adrian Jenkins 
are offered co-authorship. Perhaps you could ask Simon about this.'}
We investigate whether the galaxy formation model used for the Auriga simulations 
can produce a realistic globular cluster population at redshift zero. We compare
properties of the simulated star particles in the Auriga haloes with
catalogues of observations of the Milky Way globular cluster population available
in the literature. We find that the Auriga simulations do produce sufficient mass
at radii and metallicities that are typical for the MW GCS, although we observe
a varying mass-excess for the different $R_{\text{GC}}$-[Fe/H] bins. This implies
different values for the combined product of the bound cluster formation efficiency
and the globular cluster disruption rate. We investigate wether these differences
could result from formation insitu vs. accreted star particles. We find ... TODO.
Furthermore we test whether any of the Auriga galaxies has a metallicity and radial
distribution that is consistent with the MW (M31) GCS. For all of the Auriga haloes
we reject the null hypothesis that the simulated and observed metallicities are 
drawn from the same distribution at the 99.99\% confidence level, for the GCS
of the Milky Way as well as that of the Andromeda galaxy. The same holds true for
the distribution of galactocentric radius.
\end{abstract}

% Select between one and six entries from the list of approved keywords.
% Don't make up new ones.
\begin{keywords}
methods: numerical -- galaxies: formation -- galaxies: star clusters: general.
\end{keywords}

%%%%%%%%%%%%%%%%%%%%%%%%%%%%%%%%%%%%%%%%%%%%%%%%%%

%%%%%%%%%%%%%%%%% BODY OF PAPER %%%%%%%%%%%%%%%%%%

\section{Introduction}
\todo[inline]{Paragraph: General introduction of GCs}
Globular clusters (GCs) are old, ubiquitous, and bright. 
GC ages range from $6-14$~Gyr, although observational age-estimates can be 
uncertain and a definitive age threshold to classify star clusters as `globular' 
remains topic of debate.
GCs are observed at scales ranging from clusters of galaxies down to 
dwarf galaxies 
The globular cluster system (GCS) of galaxy clusters can comprise 
a few hundred thousand GCs (citation needed), 
brightest cluster galaxies a few ten thousand (citation needed), 
spiral galaxies a few hundred (citation needed), 
and a handful GCs are expected to be present in dwarf galaxies with masses as low 
as $\sim10^{8}$~M\Sun (citation needed).

\todo[inline]{
    `The population of galaxies bound to a rich galaxy cluster can host 
    hundreds of thousands of GCs \citep{2013ApJ...775...20A}'
    \citep[stolen from][]{2018MNRAS.475.4309P}
}
\todo[inline]{
    `brightest cluster galaxies can host tens of thousands 
    \citep{2008ApJ...681..197P, 2017ApJ...835..101H}'
    \citep[stolen from][]{2018MNRAS.475.4309P}
}
\todo[inline]{
    \citep[for recent reviews, see e.g.][]{2006ARA&A..44..193B, 2014CQGra..31x4006K}
}
\todo[inline]{
    `the Milky Way (MW) and M31 are known to host a few hundred
    \citep{1991ARA&A..29..543H}'
    \citep[stolen from][]{2018MNRAS.475.4309P}
}
\todo[inline]{
    `Even at masses as low as $10^{8}$~M\Sun, the majority of galaxies still
    contain at least one GC \citep[e.g.][]{2010MNRAS.406.1967G}' 
    \citep[stolen from][]{2018MNRAS.475.4309P}
}

The properties of the GCs themselves are correlated with those of their host
galaxies, in particular their metallicities. 
\todo[inline]{
    `The mean metallicities of GC systems have long been 
    known to scale with the metallicity of their host (van den Bergh
    1975; Brodie \& Huchra 1991)' \citep[from][]{2008ApJ...681..197P}
}

% \begin{itemize}
%     \item \citet{2005MNRAS.364..367D}: ``The radial profile of the stellar halo
%     and metal-poor globular clusters of the Milky Way suggest that these components 
%     formed in rare early peaks above $2.5 \sigma$ at redshift above 10. ''
% 
%     \item \citet{2017MNRAS.465.3622R}: ``GCs among oldest astrophysical objects.
%     GCs form in the early Universe in highest density peaks
%     \citep[e.g.][]{2005MNRAS.364..367D, 2009ApJ...706L.192B}''
% 
%     \item ``Hence, they witness most of the formation and evolution processes
%     of galaxies, and can be used to probe them'' \citep{2006ARA&A..44..193B}
% \end{itemize}

\todo[inline]{Paragraph: scientific context / state of the field: briefly summarise
    previous work and references work of other groups}
% \begin{itemize}
% \item Origin of the Milky Way globular clusters \citep{2017MNRAS.465.3622R}
% \item GCs in FIRE \citep{2018MNRAS.474.4232K}
% \item EMOSAICS project \citep{2018MNRAS.475.4309P}
% \item GC-DMhalo connection \citep{2017MNRAS.472.3120B}
% \item Origin of GC bimodality? \citep{2018MNRAS.479..200F}
% \item \textit{GAIA} DR2: GC kinematics \citep{2018A&A...616A..12G}
% \item Dating GC Tidal Disruption \citep{2018ApJ...859L..13B}
% \item GC in N-body simulation \citep{2018ApJ...861...69C}
% \item (Role of GC mass evolution on stream properties \citep{2018MNRAS.474.2479B})
% \item GC formation from dwarfs to giants \citep{2018MNRAS.480.2343C}
% \item (GC contribution to EOR \citep{2018MNRAS.479..332B})
% \item Early Universe supermassive star / GC formation \citep{2018MNRAS.478.2461G}
% \item GC formation in cold filaments \citep{2018ApJ...861..148M}
% \item GC formation in high-redshift dwarf galaxies \citep{2018MNRAS.477..480Z}
% \item GCs in MW outer region \citep{2017arXiv170804542P}
% \item Impact of the Cutoff of the Cluster Initial Mass Function \citep{2018arXiv181001888C}
% \item Metallicity gradients in the globular cluster systems of early-type galaxies: in situ and accreted components \citep{2018MNRAS.479.4760F}
% \item Globular clusters in M31, Local Group, and external galaxies \citep{2016IAUS..317..120L}
% \item Globular Clusters Formed within Dark Halos I: present-day abundance, distribution and kinematics \citep{2019MNRAS.482..219C}
% \item (The mass of the Milky Way from satellite dynamics \citep{2018arXiv180810456C})
% \item Globular cluster formation and evolution in the context of cosmological galaxy assembly: open questions \citep{2018RSPSA.47470616F}
% \item The kinematics of globular clusters systems in the outer halos of the Aquarius simulations \citep{2016A&A...592A..55V}
% \item Star Cluster Formation in Cosmological Simulations \citep{2017ApJ...834...69L, 2018ApJ...861..107L, 2018arXiv181011036L}
% \item (A systematic analysis of star cluster disruption by tidal shocks - I. Controlled N-body simulations and a new theoretical model \citep{2018arXiv181200014W})
% \item Spatial mixing of binary stars in multiple-population globular clusters \citep{2018MNRAS.tmp.3147H}
% \item Star Clusters Across Cosmic Time \citep{2018arXiv181201615K}
% \item Kinematics of Subclusters in Star Cluster Complexes: Imprint of their Parental Molecular Clouds \citep{2018arXiv181201858F}
% \item Investigating the population of Galactic star formation regions and star clusters within a Wide-Fast-Deep Coverage of the Galactic Plane  \citep{2018arXiv181203025P}
% \item Globular cluster number density profiles using \textit{Gaia} DR2 \citep{2019arXiv190108072D}
% \item Clustering Clusters: unsupervised learning on GC observations (of log $M$,
%     log $\sigma_0$, log $R_e$ , [Fe/H], log $|Z|$) to determine number of groups
%     $k=2$ (disk/halo) or $k=3$ (disk/inner-halo/outer-halo): the optimal number
%     of groups given the data is $k=3$. \citep{2019arXiv190105354P}.
% \item There is a well-known hypothesis that Open Stars Clusters (OSCs) could form as a result
%     of GCs crossing the disk. New kinematical data does not support
%     this claim for two particular OSC-GC pairs, and does seem consistent
%     with this scenario for six OSC-GC pairs \citep{2019arXiv190106481B}. (Note
%     that GC crossing the disk could also trigger formation of multiple OSCs!)
% \item DM halo mass can be inferred from number of globular cluster in galaxy 
%     \citep{2019arXiv190100900B}
% \item GCs in M31 outer halo \citep{2019arXiv190111229W}
% % TODO: last checked arXiv Mar 2, 2019
% \end{itemize}
 
 


\todo[inline]{Paragraph, narrowing open questions in GC research down to formation,
    possibly ``Bimodality suggests two formation mechanisms''. In-situ 
    formation vs. accretion in hierarchical build-up of galaxies naturally produces 
    two populations of globular cluster.}
% \begin{itemize}
%     \item ``colour bimodality, blue and red clusters \citep[e.g.][]{1985ApJ...293..424Z, 1999AJ....118.1526G, 2001AJ....121.2974L, 2006ApJ...639...95P}
%     \item ``blue metal-poor (with distribution peaking at [Fe/H] $\approx -1.5$ for the Milky Way), no sign of rotation as a population (..) more metal-rich (peak at [Fe/H] $\approx -0.5$ in the Milky  Way) more spatially concentrated and rotating with the galaxy.'' \citep{1996AJ....112.1487H}
%     \item ``Blue clusters from in early Universe in galaxies that merge later. In (wet) merger process starbursts generate red population \cite{1992ApJ...384...50A, 1987nngp.proc...18S}''
%     \item ``\citet{1997AJ....113.1652F} propose instead that blue globulars form when the protogalaxy itself collapses, in a metal-poor and turbulence media. The red population would form later, once the galactic disc has settled. The formation of globular clusters would then be a multiphase process, with the first phase being interrupted possibly by cosmic reionization \citep{2002MNRAS.333..383B}.''
%     \item ``\citet{2005ApJ...623..650K, 2014ApJ...796...10L} advocate
% that major mergers are at the origin of both sub-populations: blue
% clusters form during early mergers (z > 4) while the red ones appear
% in mergers at lower redshifts (even after z = 1). Although, this
% scenario, combined with star formation enhancement in mergers,
% seems appropriate in dense galactic environment leading to the
% assembly of massive elliptical galaxies, like in the Virgo Cluster
% as tested by \citet{2014ApJ...796...10L}, it does not apply to Milky Way-
% like systems where no recent major merger took place (Wyse 2001;
% Deason et al. 2013; Ruchti et al. 2014, 2015).
% \item ``\citet{1998ApJ...501..554C} argue that red clusters form in situ while the blue ones are accreted, either via merging satellite galaxies, or by tidal capture of the clusters themselves \citep[see also][]{2013ApJ...762...39T}.''
% \end{itemize}



\todo[inline]{Paragraph: narrow the scientific motivation down to the scope of 
    this particular work}
\begin{itemize}
    \item The star formation model implemented in the Auriga simulations is capable of producing a suite/population of realistic Milky Way-like galaxies at redshift zero.
    \item Therefore the question naturally arises whether or not the Auriga simulations are also capable of faithfully producing a globular cluster population as observed in the Milky Way (or Andromeda).
    \item Globular cluster formation in cosmological zoom simulations is very interesting for two reasons. First of all, extragalactic observations typically show the integrated properties of globular clusters rather than that of the individual stars within the clusters. Moreover, the typical mass scale of globular clusters is comparable to the numerical (mass) resolution of cosmological zoom simulations. The detailed small scale physics that is at play for real world globular clusters appears in observations as the combined effect of the $10^{3-6}$ M\Sun, compared to a mass resolution of $10^{3-5}$ M\Sun for the Auriga simulations. Globular clusters can therefore serve as an ultimate test to the star formation model that is implemented in the numerical simulations. Secondly, cosmological zoom simulations provide an accurate recording of the full and detailed merger history of the simulated galaxy. This is important because theoretical paradigms for globular cluster formation in the literature know two distinct classes of GCs that are separated by their exact formation sites: an in-situ versus an accreted population. Cosmological zoom simulations uniquely allow for an investigation into globular cluster formation with particular focus on the in-situ and accreted populations.
\end{itemize}


\todo[inline]{Paragraph: Paper outline}
We summarise the relevant characteristics of the Auriga simulations in 
section~\ref{sec:auriga}, followed by a summary of the observations of the
Milky Way (MW) globular cluster system (GCS) in section~\ref{sec:observations}
that we use to compare our simulations to in section~\ref{sec:results}. We
discuss our findings in section~\ref{sec:discussion} to come to our conclusions 
in section~\ref{sec:conclusions}.


\section{The Auriga simulations}
\label{sec:auriga}
We use the Auriga simulations \citep[][hereafter G17]{2017MNRAS.467..179G}, a 
suite of high-resolution cosmological zoom simulations of Milky Way-mass 
selected initial conditions. The simulations are performed with the 
state-of-the art code \textsc{arepo} \citep{2010MNRAS.401..791S,
2016MNRAS.455.1134P}, that solves the magnetohydrodynamical equations on a 
moving mesh, and an elaborate galaxy formation model that produces realistic 
spiral galaxies at redshift $z=0$. 

The interstellar medium is modelled using a sub-grid approach which implements
the physical processes most relevant to galaxy formation and evolution. 
This model was tailored to the \textsc{arepo} code and calibrated to reproduce
key observables of galaxies, such as the history of the cosmic star formation rate 
density, the stellar mass to halo mass relation, and galaxy luminosity functions.

The sub-grid includes primordial and metal-line cooling with self-shielding 
corrections. Reionization is completed at redshift six by a time-varying 
spatially uniform UV background \citep{2009ApJ...703.1416F, 2013MNRAS.436.3031V}
The interstellar medium is described by an equation of state for a two-phase medium
in pressure equilibrium \citep{2003MNRAS.339..289S} with stochastic star formation
in thermally unstable gas with a density threshold of $n = 0.13 \text{cm}^{-3}$,
and consecutive stellar evolution is accounted for. Stars provide feedback by 
stellar winds \citep{2014MNRAS.437.1750M, 2017MNRAS.467..179G}, and further 
enrich the ISM with metals from SNIa, SNII, and AGB stars \citep{2013MNRAS.436.3031V}.
The formation of black holes is modelled which results in feedback from active 
galactic nuclei \citep{2005MNRAS.361..776S, 2014MNRAS.437.1750M, 2017MNRAS.467..179G}.
Finally, the simulations follow the evolution of a magnetic field of $10^{-14}$
(comoving)~G seeded at $z = 127$ \citep{2013MNRAS.432..176P, 2014ApJ...783L..20P}.
See G17 for further details of the numerical setup as well as the galaxy formation
model.

\todo[inline]{
TODO: paraphrase ``The diversity in morphological properties of these simulated 
galaxies reflects the stochasticity inherent to the process of galaxy formation 
and evolution 
\citep[e.g.][]{2005ApJ...635..931B, 2010MNRAS.406..744C, 2010ApJ...708.1398T}.''
}

\todo[inline]{Paragraph about Auriga's stellar haloes.
``The Auriga Stellar Haloes: Connecting stellar population properties with
accretion and merging history'' \citet{2018arXiv180407798M}
}


\section{Relevant observational data}
\label{sec:observations}
We summarise relevant observations of the globular cluster system of the Milky 
Way in Sec.~\ref{sec:milkyway}, and of Andromeda (M31) in Sec.~\ref{sec:andromeda}.

\subsection{Milky Way}
\label{sec:milkyway}
\citet[][2010 edition; hereafter H96e10]{1996AJ....112.1487H} provides a
catalogue\footnote{See \url{https://www.physics.mcmaster.ca/Fac_Harris/mwgc.dat}} 
of the Milky Way globular cluster system that contains properties of 
157 GCs. The authors initially estimated the size of the MW GCS to be 180~$\pm$~10,
thus, their catalogue to be $\sim$85\% complete. However, an additional 59 GCs 
have since been discovered by various authors. The total confirmed number of GCs 
in the MW adds up to 216 with new estimates now anticipating an additional thirty 
GCs yet to be discovered \citep[e.g.][and references therein]{2018ApJ...863L..38R}.

\citet{2019AJ....157...12B} communicate the latest efforts to aggregate the
available data, presented in their CatClu catalog. Amongst 10978 star clusters 
and alike objects in the Milky Way, the catalog contains 200 GCs and 94 GC 
candidates. The CatClu catalog contains reference papers, positions, distances,
and total absolute V magnitude. Therefore we rely on the H96e10 dataset for all
other quantities, but we caution that the Harris catalogue is now believed to be (only) 53-72\% complete. 

Specifically, the relevant data fields that we use from H96e10 are the 
metallicity [Fe/H], the Galactic distance components $X$, $Y$, and $Z$ (in 
kpc)\footnote{In a Sun-centered coordinate system: $X$ points toward Galactic 
center, $Y$ in direction of Galactic rotation, and $Z$ toward the North Galactic 
Pole. We calculate the galactocentric radius $R_{\text{GC}}=\sqrt{(X-R_\odot)^2
+ Y^2 + Z^2}$, assuming the solar radius $R_\odot=8$~kpc.}, and absolute 
magnitude in the V-band $M_V$. We use the latter to calculate mass-estimates by 
assuming $M_{V,\odot}=4.83$ and a mass to light ratio $M/L_V = 1.7$~M/L$_{\odot}$, 
the mean for MW clusters \citep{2005ApJS..161..304M}.

\subsubsection*{Age estimates}
We supplement H96e10 with age-estimates from isochrone fits to stars
near the main-sequence turnoff in 55 GCs \citep[][hereafter V13]{2013ApJ...775..134V}.
The mean value of the age-estimates in this data set is $11.9 \pm 0.1$~Gyr and 
the dispersion is $0.8$~Gyr. Furthermore, only one of the 55 GC age-estimates is
below $10$~Gyr. 

\subsubsection*{Distribution of total GC mass in metallicity-radial bins}
In Fig.~\ref{fig:MW-FeHRgc} we show the two-dimensional mass-weighted 
metallicity-radial distribution of the MW GCS. In Sec.~\ref{sec:FeHRgc} 
we investigate whether the star formation model implemented in the Auriga 
simulations can produce sufficient total mass in GC candidates in the same bins.

\begin{figure}
    \includegraphics[width=\columnwidth]
        {{MW_RgcFeH_HistogramMassWeighted_Harris1996ed2010data}.png}
    \includegraphics[width=\columnwidth]
        {{M31_RgcFeH_HistogramMassWeighted_CaldwellRomanowsky2016data}.png}
    \caption{
        \emph{Top}: Mass-weighted r$_{\text{gc}}$-[Fe/H] distribution of
        151 GCs in the MW \citep[data from][2010 ed.]{1996AJ....112.1487H}, which
        is 98.19 \% of the total MW GCS mass in the Harris catalog.
        \emph{Bottom}: Same for M31, showing 366 GCs and 88.38 \% of the total 
        mass in CR16 \citep[data from][]{2016ApJ...824...42C}.
        \label{fig:MW-FeHRgc}
    }
\end{figure}

\subsection{Andromeda}
\label{sec:andromeda}
The fifth revision of the revised bologna catalogue (RBC~5, last updated 
August, 2012) is the latest edition of three decades of systematically 
collecting integrated properties of the globular cluster system of the 
Andromeda galaxy \citep[][and references therein]{2004A&A...416..917G}. One
contribution to RBC~5 is the work by \citet[][hereafter C11]{2011AJ....141...61C},
subsequently updated by \citet[][hereafter CR16]{2016ApJ...824...42C}.

C11 and CR16 present a uniform set of spectroscopic observations calibrated 
on the Milky Way GCS of the inner $1.6^\circ~(\sim21)$~kpc that 
is believed to be 94\% complete. GCs in the outer stellar halo, up to 
$R_{\text{proj}}\sim150$~kpc, are observed in the Pan-Andromeda Archaeological
Survey \citep[PAndAS, ][hereafter H14]{2014MNRAS.442.2165H}, but see also
\citet{2014MNRAS.442.2929V} and \citet{2019MNRAS.484.1756M}. H14 presents the
discovery of 59 new GCs and publishes updates to RBC~5. The work of H14 is
incorporated in the latest public release\footnote{Last revised 23 Sep 2015, see
\url{https://www.cfa.harvard.edu/oir/eg/m31clusters/M31_Hectospec.html}} 
of the C11 dataset, further revised by CR16. It seems that CR16 is the most 
recent aggregated dataset of M31's GCS that contains properties of interest
for our study as it contains GCs in the inner region and in the outer halo. The 
relevant fields in the CR16 dataset that we use are the age, metallicity, and the 
mass-estimate\footnote{The authors assumed $M/L_V = 2$ (independent of [Fe/H])}.

% \todo[inline]{
% ``For M31 GC masses we combine the catalogues of Caldwell et al. (2011, using the 
% given masses) and Huxor et al. (2014, again assuming $M/L_V = 1.7$~M/L$_{\odot}$, 
% e.g. Strader, Caldwell \& Seth 2011).''
% }

\subsubsection*{Age estimates}
For M31 we find an age distribution with a mean value of 11.0$\pm 0.2$~Gyr and a 
dispersion of $2.2$~Gyr. Furthermore, $24$ GCs have age-estimates below $10$~Gyr, and
the minimum age is $4.8$~Gyr. 

We present a mass-weighted histogram of the age-estimates of the $55$ MW GCs in V13
and $85$ GCs in M31 for which age-estimates are available in CR16, see 
Fig.~\ref{fig:MW-M31-age}.

\begin{figure}
    \includegraphics[width=\columnwidth]{{MW_M31_Age_HistogramMassWeighted}.png}
    \caption{
        Mass-weighted age distribution of 55 GCs in the MW
        \citep[data from][]{2013ApJ...775..134V} and 85 GCs
        in M31 \citep[data from][]{2016ApJ...824...42C}.
        \label{fig:MW-M31-age}
    }
\end{figure}

\subsubsection*{Calculation of galactocentric radius}
The projected galacocentric radius $R_{\text{proj}}$ is calculated following
\citet[][Sec.~4.1, eq.~4]{2019arXiv190111229W}. Specifically,
\begin{align}
    X &= A \sin{(\text{PA})} + B \cos{(\text{PA})} \nonumber \quad \text{and} \\
    Y &= - A \cos{(\text{PA})} + B \sin{(\text{PA})} \text{,}
\end{align}
where $A = \sin{(\alpha - \alpha_0)} \cos{\delta}$ and
$B = \sin\delta \cos\delta_0 - \cos{(\alpha - \alpha_0)} \cos\delta \sin\delta_0$
for the right ascension $\alpha$ (hourangle) and declination $\delta$ (degree) 
of the GCs. For the central position of M31 we take the preferred position 
from the NASA Extragalactic Database\footnote{\url{https://ned.ipac.caltech.edu/}}
entry of M31: $(\alpha_0, \, \delta_0) = 
(0^{\text{h}}42^{\text{m}}44.35^{\text{s}}, \, +41^{\circ}16'08.63")$.
The projected galactocentrid radius is calculated as $R_\text{proj} = 
\sqrt{X^2 + Y^2}$ (for $X$ and $Y$ in arcsec), and using the small angle 
approximation with a distance of $D_{\text{M31}} = 778$~kpc to convert to kpc.
TODO: McConnachie et al. 2005; Conn  et  al. 2012. Finally, we convert the 
projected distance ($R_{\text{proj}}$) to an `average deprojected distance` 
via the relationship $R_{\text{GC}} = R_{\text{proj}} \times (4/\pi)$.

\section{Results}
\label{sec:results}
We define GC candidates in the Auriga simulations as all star particles older 
than $10$~Gyr based on the age distribution of the MW GCS in 
Fig.~\ref{fig:MW-M31-age} and following \citet{2017MNRAS.465.3622R}.

Through out the analysis we use six sub sets of star particles: 
\emph{all stars}, and \emph{old stars} (age $>10$~Gyr, GC candidates hereafter). 
Both sub sets are further split up in (old) star particles that have formed 
\emph{in-situ} (bound to the most-massive halo/subhalo in the first snapshot that 
the star particle was recorded), and (old) \emph{accreted} star particles (i.e. 
those that have formed ex-situ and are bound to the most-massive halo/subhalo
at $z=0$). 

In Sec.~\ref{sec:metallicity} we investigate the metallicity distribution of the
GC candidates, in Sec.~\ref{sec:radial} we show the distribution of galactocentric
radii of the globular cluster candidates within the Auriga simulations, and we 
combine both in Sec.~\ref{sec:FeHRgc}. 
% We continue our analysis with an investigation
% of the properties of the proto-galaxies at times of birth of the accreted globular 
% cluster population in Sec.~\ref{sec:birth-properties-of-accreted-gc-candidates}

\subsection{Metallicity distribution}
\label{sec:metallicity}
We investigate whether the star formation model implemented in the Auriga 
simulations is capable of producing sufficient mass in old star particles in 
each metallicity bin in comparison to the Milky Way and Andromeda globular 
cluster systems. In Fig.~\ref{fig:FeH} we show a mass-weighted metallicity 
distribution where the lines show the median value of all thirty Auriga level 4 
haloes for \emph{all stars} (orange dotted), \emph{old stars / GC candidates} 
(orange solid), \emph{old in-situ} stars (blue solid), and \emph{old accreted} stars
(red solid). The shaded regions indicate the $1\sigma$ interval around the median 
(i.e. the scatter between runs with different initial conditions and merger
histories). The MW (M31) GCS is shown in purple (magenta). We use the 
same bin sizes for the simulations as for the observations, explicitly shown for
the observed profiles.

\begin{figure}
    \includegraphics[width=\columnwidth]{{logMFeH_withRatios-trim}.png}
    \caption{
        Mass-weighted metallicity distribution of star particles in the Auriga 
        simulations. We show the median value of all Auriga haloes for all
        stars (orange dotted) and globular cluster candidates (i.e. stars with 
        age $>$~10~Gyr; orange solid). The latter sub set is further split up
        between stars that formed in-situ (blue solid), and those that were accreted
        (red solid). Shaded regions indicate the $1\sigma$ interval. The MW (M31)
        GCS is shown in purple (magenta). The middle (bottom) panel shows the
        ratio of the simulated mass to the mass in the MW (M31) GCS.
        \label{fig:FeH}
    }
\end{figure}

The peak of the distribution shifts down from [Fe/H] $\thicksim 0.0$ to 
$\thicksim -0.6$ for the \emph{old stars} compared to \emph{all stars} while the 
mass at the peak lowers by roughly one dex. Moreover, we find that the 
metallicity range $-3 <$ [Fe/H] $< -1$ is only populated by the GC candidates, 
while [Fe/H] $> -1$ is dominated by star particles younger than 10~Gyr.
Furthermore, the \emph{old accreted} sub set contributes most significantly to 
the range $-3 <$ [Fe/H] $< -1$, and the contribution of the \emph{old in-situ}
stars at these metallicities declines steeper with declining metallicity 
than the \emph{old accreted} population. We note that the scatter between different
Auriga haloes is much smaller than the difference between the MW and M31 GCSs.
We conclude that old star particles in the Auriga simulation suite as a whole 
cannot be consistent with both the Milky Way and the Andromeda globular cluster
system.

The middle (bottom) panel shows the ratio of the simulated mass to the mass in
the MW (M31) GCS. We observe an increasing trend with increasing metallicity 
for the Milky Way over the entire range of the data, while the M31 GCS shows
this increase only in the range [Fe/H]$>$-0.5 (although not for the \emph{old
accreted} component.


Furthermore we test the null hypothesis that the metallicity distribution of the
MW (M31) GCS and the \emph{old}, \emph{old insitu}, and \emph{old accreted} star 
particles in the Auriga simulations are drawn from the same underlying 
distribution. We calculate the two-sample Kolmogorov-Smirnov test statistic for
all thirty Auriga level 4 haloes and reject the null hypothesis for every halo, 
for every sub set of star particles at least at the 99.99\% confidence level.
In addition, the null hypothesis that the metallicity distributions of the MW 
and M31 GCS are drawn from the same distribution is rejected at the 99.99997\%
confidence level.

% We notice a scatter\footnote{Here scatter means the width of the 1$\sigma$ interval.
% The difference between the minimum and maximum values is $0.9$~dex.} in mass 
% normalization of $0.3$ dex between the different Auriga galaxies. We investigate 
% whether there is a correlation between the total stellar mass in (old) star particles
% and the virial mass\footnote{The virial mass is defined as the mass contained 
% inside the radius $r_{200,c}$ at which the average (spherical) mass density 
% equals two hundred times the critical density of the Universe} $M_{200,c}$ of
% the host halo. We fit $a \cdot \left(\text{M}_{200,c}[1\text{e}10 \text{M\Sun}]
% - 140\right) + b$ to log$_{10}\left(\Sigma_{i} m_i(\text{\scriptsize [Fe/H] = 
% -3})\right)$. % \limits_{i, \text{if age(i) $>$ 10~Gyr}}
% Our fit thus provides the slope $a$ and normalization of (old) stellar mass $b$
% at [Fe/H]=-3 for $M_{200,c}$ at 140e10~M\Sun. We find $a = 0.00388$, $b = 6.597$
% and conclude that there is a small positive correlation of mass in (old) star 
% particles with the virial mass of the host halo.
% 
% \begin{figure}
% \includegraphics[width=0.49\textwidth]{{logMFeH_OldStars_normalization_fit3}.png}
%     \caption{
%         Plot of log$_{10}$(mass normalization) [i.e. $b$ obtained above] against 
%         the virial mass M$_{200,c}$ of the Auriga haloes. We fit a linear relation 
%         to see whether there is a correlation, and find $a = 0.00388$, $b = 6.597$. 
%         % Colours indicate resolution: L3 green, L4 orange, and L5 blue.
%         The red dotted line shows the `central' value of M$_{200,c}$, the yellow 
%         region shows the 1$\sigma$ interval around the best-fit relation, and the
%         blue dashed lines shows the intrinsic scatter. The error bars show 1\% 
%         of the obtained values.
%         \label{fig:Mnorm-fit}
%     }
% \end{figure}

\subsection{Radial distribution}
\label{sec:radial}

\begin{figure}
    \includegraphics[width=\columnwidth]{{logMRgc_withRatios-trim}.png}
    \caption{
        Mass-weighted radial distribution of star particles in the Auriga 
        simulations. We show the median value of all Auriga haloes for all
        stars (orange dotted) and globular cluster candidates (i.e. stars with 
        age $>$~10~Gyr; orange solid). The latter sub set is further split up
        between stars that formed in-situ (blue solid), and those that were accreted
        (red solid). Shaded regions indicate the $1\sigma$ interval. The MW (M31)
        GCS is shown in purple (magenta). The middle (bottom) panel shows the
        ratio of the simulated mass to the mass in the MW (M31) GCS.
        \label{fig:Rgc}
    }
\end{figure}


\todo[inline]{
    Harris: `Somewhat arbitrarily, I will take the region $r_p > 3$ kpc 
    (containing 75 clusters) as the fiducial Milky Way sample. If we were to
    view the Milky Way at the same inclination angle to the disk as we see 
    M31, this cutoff in projected distance would correspond roughly to the 
    inner distance limits in the M31 halo sample.'
}


Is the spatial distribution of the GC candidates in the Auriga simulations consistent with the MW GCS?

\subsection{Metallicity-radial distribution}
\label{sec:FeHRgc}
What age-metallicity distribution is produced by star formation events in the Auriga simulations?
?
\begin{figure}
    \includegraphics[width=0.49\textwidth]
        {{Au-all_RgcFeH_HistogramMassWeighted_iold_mean}.png}
    \caption{
        Mass-weighted [Fe/H]-R$_{\text{GC}}$ distribution of all Auriga haloes 
        (level 3, 4 and 5). Here we consider the old ($>$~10~Gyr) stars in all 
        simulations and color-code the \textbf{mean value} (of 40 Auriga haloes)
        \label{fig:Au-FeHRgc}
    }
\end{figure}

\begin{figure}
    \includegraphics[width=0.49\textwidth]
        {{Au-all_RgcFeH_HistogramMassWeighted_iold_36bins_mean-trim}.png}
    \caption{
        Ratio
        \label{fig:Au-FeHRgc-ratio}
    }
\end{figure}



\subsection{Metallicity-radial distribution: higher resolution}
\todo[inline]{Do we even want this?}
\label{sec:FeHRgc-morebins}
\begin{figure}
    \includegraphics[width=\columnwidth]
        {{logMRgcFeH_accreted-insitu-notitle}.png}
    \caption{
        TODO: add 1$\sigma$ interval around both relations, and flip x and y.
        \label{fig:todo2}
    }
\end{figure}



\subsection{From star particles to globular clusters}
\label{sec:star-to-gc}
Star particles are not globular clusters. Many stars do form in clusters, but
not all clusters end up gravitationally bound. Star particles in the
Auriga simulations represent single-age stellar populations that have formed
at the same location within the galaxy. Therefore one could assume a model
for the star cluster formation efficiency $\Gamma$, which could be used to 
`convert' star particles to bound star clusters e.g. Kruijssen (2012). This 
model relies on the local birth properties of the star particles. However, 
in our analysis we can retrieve the properties of the star particle in the 
first snapshot it was recorded, but not the gas properties at times of birth.
Therefore we are unable to model the formation of star clusters in more detail.

Furthermore, we compare star particles to present-day globular clusters, thereby
ignoring the effects of (dynamical) disruption of globular clusters over nearly
a Hubble time. As shown by Pfeffer et al. (2018), a detailed model of the tidal
history of star clusters requires a temporal resolution of order mega year. For
the Auriga level 4 simulations we have 128 snapshots for the age of the Universe,
thus, a far too coarse temporal resolution for meaningful calculations.

Therefore we investigate the over-production of simulated mass in the metallicity
and radial bins and use the term `efficiency' to refer to the combined product
of bound cluster formation and globular cluster disruption. In Fig.~\ref{fig:todo} 
we show the efficiencies that we find when we compare the simulations to the 
globular cluster systems of the Milky Way as well as that of Andromeda (M31).



\subsection{Properties of birth haloes of the accreted population}
\label{sec:birth-properties-of-accreted-gc-candidates}

Auriga galaxy /w 5 Myr snapshots.



\subsection{Age-metallicity distribution}
\label{sec:agemetallicity}
What age-metallicity distribution is produced by star formation events in the Auriga simulations?


\subsection{Formation history} 
\label{sec:history}
Can we identify particular star formation events that generate GC candidates with the correct age, metallicity, and radial properties as expected or the MW GCS?

Can we distinguish between particles that have formed in-situ and those that have been accreted? Can we identify specific features in the age-metallicity plane, or in the $R_\textbf{GC}$-[Fe/H] plane, that result from one of both populations? How does this connect to proposed mechanisms for globular cluster formation in the literature?


Orbits: are the pericentres different? Look at velocity + specific angular momentum distribution in the different FeH/Rgc bins as proxy for the pericenter




\section{Discussion}
\label{sec:discussion}

We investigate all star particles in the Auriga simulations that are older than 10~Gyr,
an approach equal to the method of \citep{2017MNRAS.465.3622R}. This approach does not
take the bound cluster fraction \citep[e.g.][]{2012MNRAS.426.3008K} into account. This
means that our sub set, which is based on selection by age, comprises both stars in the
field as well as globular clusters. We compare the total mass in the simulations in 
metallicity ([Fe/H]), galactocentric radius $R_{\text{GC}}$, and [Fe/H]-$R_{\text{GC}}$ 
bins to the total mass in the MW GCS (using the H96e10 data set). The mass excess in the
simulations gives a maximum mass loss `budget' for the product of cluster formation 
efficiency and dynamical evolution. We find fractions that vary with metallicity,
radius and metallicity-and-radius.

\todo[inline]{
``The fraction of all star formation that occurs in bound stellar clusters (the cluster formation efficiency, hereafter CFE) follows by integration of these local clustering and survival properties over the full density spectrum of the ISM, and hence is set by galaxy-scale physics. We derive the CFE as a function of observable galaxy properties, and find that it increases with the gas surface density'' \citep{2012MNRAS.426.3008K}
}

\subsection{Age cut}
Although the age distribution of M31 Perhaps an age cut of 6~Gyr would would be more 
appropriate for M31, see Fig.~\ref{fig:MW-M31-age}.


\todo[inline]{
\citet{2016ApJ...824...42C} writes:
``there are two broad, well-established differences:
(1) the M31 GC system is more populous than the MW system, by a factor of
$\sim$2-3, and (2) it does not exhibit the same obvious bimodality in
metallicity \citep{2000AJ....119..727B, 2009A&A...508.1285G, 2011AJ....141...61C, 
2013A&A...549A..60C}. Both of these aspects may be reflections of dramatic 
differences discovered in these galaxies stellar halos, where the M31 halo 
appears much more metal-enriched, with massive substructures suggesting a more 
active satellite accretion history \citep[e.g.][]{2009Natur.461...66M}''
}


\section{Summary and conclusions}
\label{sec:conclusions}


\section*{Acknowledgements}
TLRH acknowledges support from the International Max-Planck Research School (IMPRS) on Astrophysics.


\todo[inline]{Check Auriga boilerplate that we need to acknowledge}
RG and VS acknowledge support by the DFG Research Centre SFB-881 `The
Milky Way System' through project A1. This work has also been
supported by the European Research Council under ERC-StG grant
EXAGAL- 308037. Part of the simulations of this paper used the
SuperMUC system at the Leibniz Computing Centre, Garching,
under the project PR85JE of the Gauss Centre for Supercomputing.
This work used the DiRAC Data Centric system at Durham
University, operated by the Institute for Computational Cosmology
on behalf of the STFC DiRAC HPC Facility `www.dirac.ac.uk'.
This equipment was funded by BIS National E-infrastructure capital 
grant ST/K00042X/1, STFC capital grant ST/H008519/1 and
STFC DiRAC Operations grant ST/K003267/1 and Durham University. 
DiRAC is part of the UK National E-Infrastructure.

%%%%%%%%%%%%%%%%%%%%%%%%%%%%%%%%%%%%%%%%%%%%%%%%%%

%%%%%%%%%%%%%%%%%%%% REFERENCES %%%%%%%%%%%%%%%%%%

% The best way to enter references is to use BibTeX:

\bibliographystyle{mnras}
\bibliography{AurigaGCS} % if your bibtex file is called example.bib


% Alternatively you could enter them by hand, like this:
% This method is tedious and prone to error if you have lots of references
%\begin{thebibliography}{99}
%\bibitem[\protect\citeauthoryear{Author}{2012}]{Author2012}
%Author A.~N., 2013, Journal of Improbable Astronomy, 1, 1
%\bibitem[\protect\citeauthoryear{Others}{2013}]{Others2013}
%Others S., 2012, Journal of Interesting Stuff, 17, 198
%\end{thebibliography}

%%%%%%%%%%%%%%%%%%%%%%%%%%%%%%%%%%%%%%%%%%%%%%%%%%

%%%%%%%%%%%%%%%%% APPENDICES %%%%%%%%%%%%%%%%%%%%%
\clearpage
\appendix
\section{Scatter between individual Auriga haloes, and numerical convergence} 
\label{sec:scatter-convergence}
We check whether the properties of the Auriga globular cluster candidates 
are well converged between the three different resolution levels used for the 
Auriga simulations. Here we consider all three Auriga haloes for which simulation
runs were performed at all three resolution levels: Au6, Au16, and Au24. Here
we can investigate differences between individual haloes. 

Fig.~\ref{fig:logMFeH_res} shows the mass-weighted metallicity distribution, 
Fig.~\ref{fig:logMRgc_res} shows the mass-weighted radial distribution, and
Fig.~\ref{fig:logMRgcFeH_res}



\begin{figure*}
    \includegraphics[width=0.31\textwidth]{{logMFeH_Au6}.png}
    \includegraphics[width=0.31\textwidth]{{logMFeH_Au16}.png}
    \includegraphics[width=0.31\textwidth]{{logMFeH_Au24}.png}
    \caption{Same as Fig.~\ref{fig:FeH}, but here the colours indicate resolution
        level: L3 green, L4 orange, and L5 blue. \emph{Left:} Auriga halo 6.
        \emph{Mid:} Auriga halo 16. \emph{Right:} Auriga halo 24. For all three
        haloes we find marginal increases in the mass normalization with increasing 
        resolution level.
        \label{fig:logMFeH_res}
    }

\end{figure*}
\begin{figure*}
    \includegraphics[width=0.31\textwidth]{{logMRgc_Au6}.png}
    \includegraphics[width=0.31\textwidth]{{logMRgc_Au16}.png}
    \includegraphics[width=0.31\textwidth]{{logMRgc_Au24}.png}
    \caption{Same as Fig.~\ref{fig:Rgc}, but here the colours indicate resolution
        level: L3 green, L4 orange, and L5 blue. \emph{Left:} Auriga halo 6.
        \emph{Mid:} Auriga halo 16. \emph{Right:} Auriga halo 24. For all three 
        haloes we find marginal increases in the mass normalization with increasing 
        resolution level.
        \label{fig:logMRgc_res}
    }

\end{figure*}
%%%%%%%%%%%%%%%%%%%%%%%%%%%%%%%%%%%%%%%%%%%%%%%%%%


% Don't change these lines
\bsp    % typesetting comment
\label{lastpage}
\end{document}

% End of mnras_template.tex
