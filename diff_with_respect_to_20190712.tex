% mnras_template.tex
%DIF LATEXDIFF DIFFERENCE FILE
%DIF DEL AurigaGCS_20190712.tex   Thu Aug  1 12:31:34 2019
%DIF ADD AurigaGCS.tex            Fri Aug  2 21:58:49 2019
%
% LaTeX template for creating an MNRAS paper
%
% v3.0 released 14 May 2015
% (version numbers match those of mnras.cls)
%
% Copyright (C) Royal Astronomical Society 2015
% Authors:
% Keith T. Smith (Royal Astronomical Society)

% Change log
%
% v3.0 May 2015
%    Renamed to match the new package name
%    Version number matches mnras.cls
%    A few minor tweaks to wording
% v1.0 September 2013
%    Beta testing only - never publicly released
%    First version: a simple (ish) template for creating an MNRAS paper

%%%%%%%%%%%%%%%%%%%%%%%%%%%%%%%%%%%%%%%%%%%%%%%%%%
% Basic setup. Most papers should leave these options alone.
\documentclass[a4paper,fleqn,usenatbib]{mnras}

% MNRAS is set in Times font. If you don't have this installed (most LaTeX
% installations will be fine) or prefer the old Computer Modern fonts, comment
% out the following line
\usepackage{newtxtext,newtxmath}
% Depending on your LaTeX fonts installation, you might get better results with one of these:
%\usepackage{mathptmx}
%\usepackage{txfonts}

% Use vector fonts, so it zooms properly in on-screen viewing software
% Don't change these lines unless you know what you are doing
\usepackage[T1]{fontenc}
\usepackage{ae,aecompl}


%%%%% AUTHORS - PLACE YOUR OWN PACKAGES HERE %%%%%

% Only include extra packages if you really need them. Common packages are:
\usepackage{graphicx}    % Including figure files
\usepackage{amsmath}    % Advanced maths commands
\usepackage{amssymb}    % Extra maths symbols

\usepackage{todonotes}
\graphicspath{{figures/}}   % Set figure path

%%%%%%%%%%%%%%%%%%%%%%%%%%%%%%%%%%%%%%%%%%%%%%%%%%

%%%%% AUTHORS - PLACE YOUR OWN COMMANDS HERE %%%%%

% Please keep new commands to a minimum, and use \newcommand not \def to avoid
% overwriting existing commands. Example:
%\newcommand{\pcm}{\,cm$^{-2}$}    % per cm-squared
\newcommand{\Sun}[0]{\ensuremath{_{\odot}}}
\renewcommand{\deg}{\ensuremath{^{\circ}}}


%%%%%%%%%%%%%%%%%%%%%%%%%%%%%%%%%%%%%%%%%%%%%%%%%%

%%%%%%%%%%%%%%%%%%% TITLE PAGE %%%%%%%%%%%%%%%%%%%

% Title of the paper, and the short title which is used in the headers.
% Keep the title short and informative.
\title[Auriga GCS]{The globular cluster system of the Auriga simulations}

% The list of authors, and the short list which is used in the headers.
% If you need two or more lines of authors, add an extra line using \newauthor
\author[T. L. R. Halbesma et al.]{\parbox[t]{\textwidth}{
    Timo L. R. Halbesma$^{1}$\thanks{E-mail: Halbesma@MPA-Garching.MPG.DE},
    Robert J. J. Grand$^{1}$,
    Volker Springel$^{1}$,
    Facundo A. G\'{o}mez$^{2,3}$,
    Federico Marinacci$^{4,5}$,
    R\"{u}diger Pakmor$^{1}$,
    Wilma Trick$^{1}$,
    Philipp Busch$^{1}$,
    Simon D. M. White$^{1}$
} \vspace{10pt} \\
$^{1}$ Max-Planck-Institut f\"ur Astrophysik, Karl-Schwarzschild-Str. 1,
    85741 Garching, Germany \\
$^{2}$ Instituto de Investigaci\'{o}n Multidisciplinar en Ciencia yTecnolog\'{i}a,
    Universidad de La Serena, Ra\'{u}l Bitr\'{a}n 1305, La Serena, Chile \\
$^{3}$ Departamento de F\'{i}sica y Astronom\'{i}a, Universidad de La Serena, Av.
    Juan Cisternas 1200 N, La Serena, Chile \\
$^{4}$ Department of Physics, Kavli Institute for Astrophysics and Space Research,
    MIT, Cambridge, MA 02139, USA \\
$^{5}$ Harvard-Smithsonian Center for Astrophysics, 60 Garden Street, Cambridge,
    MA 02138, USA \\
}

% These dates will be filled out by the publisher
\date{Accepted XXX. Received YYY; in original form ZZZ}

% Enter the current year, for the copyright statements etc.
\pubyear{2019}

% Don't change these lines
%DIF PREAMBLE EXTENSION ADDED BY LATEXDIFF
%DIF UNDERLINE PREAMBLE %DIF PREAMBLE
\RequirePackage[normalem]{ulem} %DIF PREAMBLE
\RequirePackage{color}\definecolor{RED}{rgb}{1,0,0}\definecolor{BLUE}{rgb}{0,0,1} %DIF PREAMBLE
\providecommand{\DIFadd}[1]{{\protect\color{blue}\uwave{#1}}} %DIF PREAMBLE
\providecommand{\DIFdel}[1]{{\protect\color{red}\sout{#1}}}                      %DIF PREAMBLE
%DIF SAFE PREAMBLE %DIF PREAMBLE
\providecommand{\DIFaddbegin}{} %DIF PREAMBLE
\providecommand{\DIFaddend}{} %DIF PREAMBLE
\providecommand{\DIFdelbegin}{} %DIF PREAMBLE
\providecommand{\DIFdelend}{} %DIF PREAMBLE
\providecommand{\DIFmodbegin}{} %DIF PREAMBLE
\providecommand{\DIFmodend}{} %DIF PREAMBLE
%DIF FLOATSAFE PREAMBLE %DIF PREAMBLE
\providecommand{\DIFaddFL}[1]{\DIFadd{#1}} %DIF PREAMBLE
\providecommand{\DIFdelFL}[1]{\DIFdel{#1}} %DIF PREAMBLE
\providecommand{\DIFaddbeginFL}{} %DIF PREAMBLE
\providecommand{\DIFaddendFL}{} %DIF PREAMBLE
\providecommand{\DIFdelbeginFL}{} %DIF PREAMBLE
\providecommand{\DIFdelendFL}{} %DIF PREAMBLE
\newcommand{\DIFscaledelfig}{0.5}
%DIF HIGHLIGHTGRAPHICS PREAMBLE %DIF PREAMBLE
\RequirePackage{settobox} %DIF PREAMBLE
\RequirePackage{letltxmacro} %DIF PREAMBLE
\newsavebox{\DIFdelgraphicsbox} %DIF PREAMBLE
\newlength{\DIFdelgraphicswidth} %DIF PREAMBLE
\newlength{\DIFdelgraphicsheight} %DIF PREAMBLE
% store original definition of \includegraphics %DIF PREAMBLE
\LetLtxMacro{\DIFOincludegraphics}{\includegraphics} %DIF PREAMBLE
\newcommand{\DIFaddincludegraphics}[2][]{{\color{blue}\fbox{\DIFOincludegraphics[#1]{#2}}}} %DIF PREAMBLE
\newcommand{\DIFdelincludegraphics}[2][]{% %DIF PREAMBLE
\sbox{\DIFdelgraphicsbox}{\DIFOincludegraphics[#1]{#2}}% %DIF PREAMBLE
\settoboxwidth{\DIFdelgraphicswidth}{\DIFdelgraphicsbox} %DIF PREAMBLE
\settoboxtotalheight{\DIFdelgraphicsheight}{\DIFdelgraphicsbox} %DIF PREAMBLE
\scalebox{\DIFscaledelfig}{% %DIF PREAMBLE
\parbox[b]{\DIFdelgraphicswidth}{\usebox{\DIFdelgraphicsbox}\\[-\baselineskip] \rule{\DIFdelgraphicswidth}{0em}}\llap{\resizebox{\DIFdelgraphicswidth}{\DIFdelgraphicsheight}{% %DIF PREAMBLE
\setlength{\unitlength}{\DIFdelgraphicswidth}% %DIF PREAMBLE
\begin{picture}(1,1)% %DIF PREAMBLE
\thicklines\linethickness{2pt} %DIF PREAMBLE
{\color[rgb]{1,0,0}\put(0,0){\framebox(1,1){}}}% %DIF PREAMBLE
{\color[rgb]{1,0,0}\put(0,0){\line( 1,1){1}}}% %DIF PREAMBLE
{\color[rgb]{1,0,0}\put(0,1){\line(1,-1){1}}}% %DIF PREAMBLE
\end{picture}% %DIF PREAMBLE
}\hspace*{3pt}}} %DIF PREAMBLE
} %DIF PREAMBLE
\LetLtxMacro{\DIFOaddbegin}{\DIFaddbegin} %DIF PREAMBLE
\LetLtxMacro{\DIFOaddend}{\DIFaddend} %DIF PREAMBLE
\LetLtxMacro{\DIFOdelbegin}{\DIFdelbegin} %DIF PREAMBLE
\LetLtxMacro{\DIFOdelend}{\DIFdelend} %DIF PREAMBLE
\DeclareRobustCommand{\DIFaddbegin}{\DIFOaddbegin \let\includegraphics\DIFaddincludegraphics} %DIF PREAMBLE
\DeclareRobustCommand{\DIFaddend}{\DIFOaddend \let\includegraphics\DIFOincludegraphics} %DIF PREAMBLE
\DeclareRobustCommand{\DIFdelbegin}{\DIFOdelbegin \let\includegraphics\DIFdelincludegraphics} %DIF PREAMBLE
\DeclareRobustCommand{\DIFdelend}{\DIFOaddend \let\includegraphics\DIFOincludegraphics} %DIF PREAMBLE
\LetLtxMacro{\DIFOaddbeginFL}{\DIFaddbeginFL} %DIF PREAMBLE
\LetLtxMacro{\DIFOaddendFL}{\DIFaddendFL} %DIF PREAMBLE
\LetLtxMacro{\DIFOdelbeginFL}{\DIFdelbeginFL} %DIF PREAMBLE
\LetLtxMacro{\DIFOdelendFL}{\DIFdelendFL} %DIF PREAMBLE
\DeclareRobustCommand{\DIFaddbeginFL}{\DIFOaddbeginFL \let\includegraphics\DIFaddincludegraphics} %DIF PREAMBLE
\DeclareRobustCommand{\DIFaddendFL}{\DIFOaddendFL \let\includegraphics\DIFOincludegraphics} %DIF PREAMBLE
\DeclareRobustCommand{\DIFdelbeginFL}{\DIFOdelbeginFL \let\includegraphics\DIFdelincludegraphics} %DIF PREAMBLE
\DeclareRobustCommand{\DIFdelendFL}{\DIFOaddendFL \let\includegraphics\DIFOincludegraphics} %DIF PREAMBLE
%DIF LISTINGS PREAMBLE %DIF PREAMBLE
\RequirePackage{listings} %DIF PREAMBLE
\RequirePackage{color} %DIF PREAMBLE
\lstdefinelanguage{DIFcode}{ %DIF PREAMBLE
%DIF DIFCODE_UNDERLINE %DIF PREAMBLE
  moredelim=[il][\color{red}\sout]{\%DIF\ <\ }, %DIF PREAMBLE
  moredelim=[il][\color{blue}\uwave]{\%DIF\ >\ } %DIF PREAMBLE
} %DIF PREAMBLE
\lstdefinestyle{DIFverbatimstyle}{ %DIF PREAMBLE
	language=DIFcode, %DIF PREAMBLE
	basicstyle=\ttfamily, %DIF PREAMBLE
	columns=fullflexible, %DIF PREAMBLE
	keepspaces=true %DIF PREAMBLE
} %DIF PREAMBLE
\lstnewenvironment{DIFverbatim}{\lstset{style=DIFverbatimstyle}}{} %DIF PREAMBLE
\lstnewenvironment{DIFverbatim*}{\lstset{style=DIFverbatimstyle,showspaces=true}}{} %DIF PREAMBLE
%DIF END PREAMBLE EXTENSION ADDED BY LATEXDIFF

\begin{document}
\label{firstpage}
\pagerange{\pageref{firstpage}--\pageref{lastpage}}
\maketitle

% Abstract of the paper
\begin{abstract}
We investigate whether the galaxy formation model used for the Auriga simulations
can produce a realistic globular cluster (GC) population at redshift zero. We
compare properties of the simulated star particles in the Auriga haloes with
catalogues of observations of the Milky Way (MW) and Andromeda (M31) globular
cluster populations available in the literature. We find that the Auriga 
simulations do produce sufficient mass at radii and metallicities that are 
typical for the MW globular cluster system (GCS), although we observe a varying 
mass-excess for different $R_{\text{GC}}$-[Fe/H] bins. This implies different 
values for the combined product of the bound cluster formation efficiency and 
the globular cluster disruption rate. Furthermore we test whether any of the 
Auriga galaxies has a metallicity and radial distribution that is consistent with 
the MW or M31 GCS. For any one of the Auriga haloes we reject the null hypothesis
that the observed and simulated metallicities of GCs (GC candidates) are drawn
from the same distribution at $\geq98.32$~\% confidence level, for the GCS of 
the Milky Way as well as that of the Andromeda galaxy. The same holds true for 
the distribution of galactocentric radius. Overall, the Auriga simulations produce 
old star particles with higher metallicities than the MW and M31 GCS and at 
larger radii. The formation efficiency would have to linearly decrease with 
increasing metallicity for the Auriga GC candidates to be consistent with the 
MW GCS, if cluster disruption is equally efficient regardless of metallicity.
In addition, the number of GCs relative to field stars would have to be smaller
for higher metallicity star particles at large radii than for those with lower 
metallicity at smaller radii, assuming GC disruption is more efficient in the
inner regions of the Galaxy.

\end{abstract}

% Select between one and six entries from the list of approved keywords.
% Don't make up new ones.
\begin{keywords}
methods: numerical -- galaxies: formation -- galaxies: star clusters: general.
\end{keywords}

%%%%%%%%%%%%%%%%%%%%%%%%%%%%%%%%%%%%%%%%%%%%%%%%%%

%%%%%%%%%%%%%%%%% BODY OF PAPER %%%%%%%%%%%%%%%%%%

\section{Introduction}
Globular clusters (GC)s are old, bright, ubiquitous, and various properties of
GC systems show correlations with their host galaxies. GC systems are believed to
retain information about the galactic (gas) conditions at times of formation,
thus, could offer unique insight into the (chemodynamical) evolution of their 
parent galaxies, if the formation and evolution of GCs- and GC systems themselves 
is adequately understood. However, despite decades of research (the extensive 
literature on GCs is summarised in several books and review articles, e.g.~\citealt{
1991ARA&A..29..543H, Harris2001, 2004Natur.427...31W, 2006ARA&A..44..193B, 
2012A&ARv..20...50G, 2014CQGra..31x4006K, 2018RSPSA.47470616F}), consensus on an 
exhaustive picture of the formation of GCs is yet to be reached.

The literature offers a wealth of formation scenarios. \citet{1968ApJ...154..891P}
and \citet{1984ApJ...277..470P} argue that GCs form as the earliest bound
structures in the Universe (i.e. prior to formation of the main galaxy), 
noting that the Jeans length and mass shortly after recombination is consistent
with typical GC masses and sizes. \citet{2005MNRAS.364..367D} and \citet{
2009ApJ...706L.192B} argue that (the blue subpopulations of) GCs form in biased 
dark matter halos at high redshift. Other recent hypotheses of GC formation prior 
to collapse of the proto-galaxy include formation in colliding supershells 
\citep{2017Ap&SS.362..183R}, in supersonically induced gaseous objects
\citep{2019arXiv190408941C}, or in high-speed collisions of dark matter subhaloes
\citep{2019arXiv190508951M}. \citet{2017MNRAS.472.3120B} suggests
that (the blue subpopulation of) GCs could form in high density regions along
the cosmic filament before or during collapse.

Other models also date GCs formation during formation of the proto-galaxy itself,
for example as a result of thermal instabilities in hot gas-rich haloes 
(\citealt{1985ApJ...298...18F}, also see the discussions in \citealt{
1990ApJ...363..488K}). Alternative formation trigggers are also explored, such as 
(other causes of) shock compression, or cloud-cloud colissions \citep[e.g.][]{
1980glcl.conf..301G, 1992ApJ...400..265M, 1994ApJ...429..177H, 1995ApJ...442..618V,
1996ASPC...92..241L, 2001ApJ...560..592C}. 

Yet another hypothesis is that star cluster formation is triggered by (major) 
gas-rich (spiral) galaxy mergers \citep{1987nngp.proc...18S, 1992ApJ...384...50A},
which is naturally expected within the framework of hierarchical assembly. One
testable prediction of this scenario is the formation of young clusters in
interacting and merging galaxies, which has been observed and are found to show 
remarkable similarities with globulars in the Milky Way \citep[e.g.][]{
1995AJ....109..960W, 1996AJ....112..416H, 1999AJ....118..752Z, 1999AJ....118.1551W}.
Moreover, modelling efforts of this framework yield GC (sub)populations consistent 
with various observables \citep[e.g.][]{2010ApJ...718.1266M, 2018MNRAS.480.2343C}, 
and the recent numerical simulation of an isolated dwarf-dwarf merger executed at 
very high resolution (baryonic mass $m_b \sim$~4M\Sun; softening $\epsilon = 0.1$) 
produce star clusters that could be globular progenitors \citep{
2019arXiv190509840L}.

As for the formation timeline, the scenarios (and flavours thereof) of \citet{
1985ApJ...298...18F} and \citet{1992ApJ...384...50A} are intertwined because
accretion and mergers continuously occurs trough out hierarchical build-up of
galaxies. Various (other) hierarchical formation channels thus combine different
aspects of the aforementioned paradigms, such as GC formation in (small) galactic
disks before they are accreted onto an assembling galaxy \citep[e.g.][]{
2000ApJ...533..869C, 2002ApJ...567..853C, 2002MNRAS.333..383B, 2003egcs.conf..224G}. 
We refer to \citet{2001astro.ph..8034G} for a scoreboard of the aforementioned 
paradigms compared to observations of the Milky Way globular cluster system.

An idea that has recently been studied in detail is the hypothesis that the physical 
processes responsible for the formation of young massive clusters \citep[YMCs, 
see][for a review]{2010ARA&A..48..431P}, as observable in detail in the local 
Universe in great
, also governs GC formation at high redshifts. In this framework, differences
between both classes of objects are caused by nearly a Hubble time of (dynamical)
evolution \citep[e.g.][]{1987degc.book.....S}. This picture is based on observed 
similarities between YMCs and GCs \citep[e.g.][]{1992AJ....103..691H,1999AJ....118.1551W}
and strengthened by observations of gravitationally lensed objects at high redshifts
($z = 2-6)$. These sources have properties reminiscent of (local) YMCs and may be
GC progenitors at times of formation \citep{2017MNRAS.467.4304V,2017ApJ...843L..21J}.
The modelling work by \citet{2011MNRAS.414.1339K,2012MNRAS.421.1927K,2015MNRAS.454.1658K} 
is now incorporated into cosmological zoom simulations and shows promising results
\citep{2018MNRAS.475.4309P,2019MNRAS.486.3134K}.

Simulation resolutions have reached the mass range populated by GCs, and the
gravitational force softening can be as low several parsec. A number of groups 
can thus incorporate formation of (globular) star clusters into their 
high-resolution hydrodynamical simulations. \citet{2016ApJ...831..204R} 
run parsec scale simulations of the high-redshift universe 
prior to reionization (the simulations stop at $z=9$), \citet{2017ApJ...834...69L} 
implement a new subgrid model for star (cluster) formation and run simulations 
that reach $z=3.3$, and the run of \citet{2017MNRAS.465.3622R} reaches $z=0.5$. 
\citet{2018MNRAS.474.4232K} find that mergers can push gas to high density that
quickly forms clustered stars that end up tightly bound by the end of the 
simulation. A somewhat different approach couples semi-analytical models to DM-only
simulations \citep{2010ApJ...718.1266M,2014ApJ...796...10L,2018MNRAS.480.2343C,
2019MNRAS.486..331C,2019arXiv190505199C}, as is done in the work by 
\citet{2019MNRAS.482.4528E}.

In this work we use state of the art simulations that produce realistic spiral 
galaxies at redshift zero for which several global properties are consistent with 
the observations. The simulations yield realistic Milky Way analogues, and real
galaxies host globular cluster systems. The question thus naturally arises whether 
the star formation histories of the simulations (also) give rise to a GC system 
similar to the Milky Way and/or Andromeda globular cluster systems. In particular, 
we use the Auriga suite of cosmological zoom simulations 
(\citealt{2017MNRAS.467..179G}, and described in Sec.~\ref{sec:auriga}) to
investigate whether the star formation model implemented produces metallicity, 
radial, and metallicity-radial distributions that are consistent with the MW 
and/or M31 GC systems, and whether the model produces enough stellar mass with
the right properties to allow for formation efficiencies lower than unity and
the expected (dynamical) mass loss over the cluster lifetime (not modelled 
explicitly in this study). 

The plan of the paper is as follows. We summarise the relevant characteristics 
of the Auriga simulations in section~\ref{sec:auriga}, followed by a summary of 
the observations of the Milky Way (MW) and Andromeda globular cluster system (GCS)
in section~\ref{sec:observations} that we use to compare our simulations to in
section~\ref{sec:results}. We discuss our findings in section~\ref{sec:discussion}
to come to our conclusions in section~\ref{sec:conclusions}.


\section{The Auriga simulations}
\label{sec:auriga}
We use the Auriga simulations \citep[][hereafter G17]{2017MNRAS.467..179G}, a
suite of high-resolution cosmological zoom simulations of Milky Way-mass
selected initial conditions. The simulations are performed with the \textsc{arepo} 
\citep{2010MNRAS.401..791S, 2016MNRAS.455.1134P} code that solves the 
magnetohydrodynamical equations on a moving mesh. The galaxy formation model 
produces realistic spiral galaxies at redshift $z=0$.

The interstellar medium is modelled using a sub-grid approach which implements
the physical processes most relevant to galaxy formation and evolution.
This model was tailored to the \textsc{arepo} code and calibrated to reproduce
key observables of galaxies, such as the history of the cosmic star formation rate
density, the stellar mass to halo mass relation, and galaxy luminosity functions.

The sub-grid includes primordial and metal-line cooling with self-shielding
corrections. Reionization is completed at redshift six by a time-varying
spatially uniform UV background \citep{2009ApJ...703.1416F, 2013MNRAS.436.3031V}
The interstellar medium is described by an equation of state for a two-phase medium
in pressure equilibrium \citep{2003MNRAS.339..289S} with stochastic star formation
in thermally unstable gas with a density threshold of $n = 0.13 \text{cm}^{-3}$,
and consecutive stellar evolution is accounted for. Stars provide feedback by
stellar winds \citep{2014MNRAS.437.1750M, 2017MNRAS.467..179G}, and further
enrich the ISM with metals from SNIa, SNII, and AGB stars \citep{2013MNRAS.436.3031V}.
The formation of black holes is modelled which results in feedback from active
galactic nuclei \citep{2005MNRAS.361..776S, 2014MNRAS.437.1750M, 2017MNRAS.467..179G}.
Finally, the simulations follow the evolution of a magnetic field of $10^{-14}$
(comoving)~G seeded at $z = 127$ \citep{2013MNRAS.432..176P, 2014ApJ...783L..20P}.
See G17 for further details of the numerical setup as well as the galaxy formation
model.

The Auriga suite has a fiducial resolution level L4, accompanied by the lower
(higher) level L5 (L3) that is available for selected initial condition runs.
The baryonic mass resolution in order of increasing level is $m_b$~=~[$4 \times 10^5$,
$5 \times 10^4$, $6 \times 10^3$]~M\Sun \, with gravitational softening of
collissionless particles $\epsilon$~=~[$738, 369, 184$]~pc. The mass resolution
of the Auriga simulations is thus close to the characteristic peak mass of the
lognormal GC mass distribution of $10^{5}$~M\Sun \citep{1991ARA&A..29..543H},
although the gravitational softening is two orders of magnitudes larger than
typical GC radii. High-density gaseous regions are thus not expected to produce
surviving stellar clumps with masses and radii consistent with GCs because
such objects would numerically disperse, even in the highest-resolution runs.
On the other hand, we can investigate (statistical) properties of age-selected
GC candidates because each star particle represents a single stellar population
with a total mass that could be consistent with one globular cluster. This means
that their formation sites and that of a real-world GC may be consistent.

\section{Observational data}
\label{sec:observations}
We describe the observations of the MW GCS in Sec.~\ref{sec:milkyway},
and of the Andromeda (M31) GCS in Sec.~\ref{sec:andromeda}


\subsection{Milky Way}
\label{sec:milkyway}
\citet[][2010 edition; hereafter H96e10]{1996AJ....112.1487H} provides a
catalogue\footnote{See \url{https://www.physics.mcmaster.ca/Fac_Harris/mwgc.dat}}
of the Milky Way globular cluster system that contains properties of
157 GCs. The authors initially estimated the size of the MW GCS to be 180~$\pm$~10,
thus, their catalogue to be ${\sim}$85\% complete. However, an additional 59 GCs
are claimed to have been discovered by various authors. The total number of GCs
in the MW might add up to 216 with recent estimates now anticipating an additional 
thirty GCs yet to be discovered \citep[e.g.][and references therein]{2018ApJ...863L..38R}.
We still use data from the Harris catalogue, but caution that it may (only) be
53-72\% complete. Specifically, the relevant data fields that we use from H96e10
are the metallicity [Fe/H], the Galactic distance components $X$, $Y$, and $Z$ (in
kpc)\footnote{In a Sun-centered coordinate system: $X$ points toward Galactic
center, $Y$ in direction of Galactic rotation, and $Z$ toward the North Galactic
Pole. We calculate the galactocentric radius $R_{\text{GC}}=\sqrt{(X-R_\odot)^2
+ Y^2 + Z^2}$, assuming the solar radius $R_\odot=8$~kpc.}, and absolute
magnitude in the V-band $M_V$. We use the latter to calculate mass-estimates by
assuming $M_{V,\odot}=4.83$ and a mass to light ratio $M/L_V = 1.7$~M/L$_{\odot}$,
the mean for MW clusters \citep{2005ApJS..161..304M}. We supplement the catalogue
with age-estimates from isochrone fits to stars near the main-sequence turnoff 
in 55 GCs \citep[][hereafter V13]{2013ApJ...775..134V}.

% \citet{2019AJ....157...12B} communicate the latest efforts to aggregate the
% available data, presented in their CatClu catalog. Amongst 10978 star clusters
% and alike objects in the Milky Way, the catalog contains 200 GCs and 94 GC
% candidates. The CatClu catalog contains reference papers, positions, distances,
% and total absolute V magnitude. Therefore we rely on the H96e10 dataset.

\subsection{Andromeda}
\label{sec:andromeda}
The fifth revision of the revised bologna catalogue (RBC~5, last updated
August, 2012) is the latest edition of three decades of systematically
collecting integrated properties of the globular cluster system of the
Andromeda galaxy \citep[][and references therein]{2004A&A...416..917G}. One
contribution to RBC~5 is the work by \citet[][hereafter C11]{2011AJ....141...61C},
subsequently updated by \citet[][hereafter CR16]{2016ApJ...824...42C}.

C11 and CR16 present a uniform set of spectroscopic observations calibrated
on the Milky Way GCS of the inner $1.6^\circ~({\sim}21)$~kpc that
is believed to be 94\% complete. GCs in the outer stellar halo, up to
$R_{\text{proj}}\sim150$~kpc, are observed in the Pan-Andromeda Archaeological
Survey \citep[PAndAS, ][hereafter H14]{2014MNRAS.442.2165H}, but see also
\citet{2014MNRAS.442.2929V} and \citet{2019MNRAS.484.1756M}. H14 presents the
discovery of 59 new GCs and publishes updates to RBC~5. The work of H14 is
incorporated in the latest public release\footnote{Last revised 23 Sep 2015, see
\url{https://www.cfa.harvard.edu/oir/eg/m31clusters/M31_Hectospec.html}}
of the C11 dataset, further revised by CR16. It seems that CR16 is the most
recent aggregated dataset of M31's GCS that contains properties of interest
for our study as it contains GCs in the inner region and in the outer halo. The
relevant fields in the CR16 dataset that we use are the age, metallicity, and the
mass-estimate\footnote{The authors assumed $M/L_V = 2$ independent of [Fe/H]}.
\DIFaddbegin \DIFadd{Radii are calculated from RA and DEC, further discussed in Sec.~\ref{sec:Rgc}.
}\DIFaddend 

\subsection{Age-estimates}
\DIFdelbegin \DIFdel{Figure~\ref{fig:MW-M31-age} }\DIFdelend \DIFaddbegin \DIFadd{The top panel of Figure~\ref{fig:MW-M31-age-Rgc} }\DIFaddend shows a histogram of the age-estimates
of the $55$ MW GCs in V13 and $88$ GCs in M31 for which age-estimates are available
in CR16. The mean age of the MW GCs in this data set is $11.9 \pm 0.1$~Gyr and the 
dispersion is $0.8$~Gyr. Furthermore, only one of the 55 GC age-estimates is
below $10$~Gyr. The M31 GCS has a mean age of $11.0 \pm 0.2$~Gyr with a dispersion 
of $2.2$~Gyr, and $24$ GCs have age-estimates below $10$~Gyr with a minimum age 
of $4.8$~Gyr. Based on these data, we find that the age distributions of the MW 
and M31 GCS's are not statistically consistent. Moreover, the Milky Way appears 
to host a globular cluster system that is somewhat older than the Andromeda galaxy,
which is somewhat surprising given that M31 is generally considered to be earlier
type than the MW meaning that an older stellar population would naively be expected.
However, we do caution that both data sets are incomplete and the age 
measurements have large uncertainties (of $1-2$~Gyr). On the other hand, the 
magnitude of the uncertainty is insufficient to explain the low-age tail in M31.

\DIFdelbegin %DIFDELCMD < \begin{figure}
%DIFDELCMD <     \includegraphics[width=\columnwidth]{{MW_M31_Age_Histogram-trim}.png}
%DIFDELCMD <     %%%
%DIFDELCMD < \caption{%
{%DIFAUXCMD
\DIFdelFL{Age distribution of 55 GCs in the MW
        \mbox{%DIFAUXCMD
\citep[data from][]{2013ApJ...775..134V} }\hspace{0pt}%DIFAUXCMD
and 88 GCs
        in M31 \mbox{%DIFAUXCMD
\citep[data from][]{2016ApJ...824...42C}}\hspace{0pt}%DIFAUXCMD
.
        }%DIFDELCMD < \label{fig:MW-M31-age}
%DIFDELCMD <     %%%
}
%DIFAUXCMD
%DIFDELCMD < \end{figure}
%DIFDELCMD < 

%DIFDELCMD < \subsection{Total GC mass in metallicity-radial space}
%DIFDELCMD < \label{sec:observations_FeHRgc}
%DIFDELCMD < %%%
\DIFdel{We show the two-dimensional mass-weighted metallicity-radial }\DIFdelend \DIFaddbegin \subsection{Radial distribution}
\label{sec:Rgc}
\DIFadd{The bottom panel of Figure~\ref{fig:MW-M31-age-Rgc} shows the radial }\DIFaddend distribution
of the MW \DIFdelbegin \DIFdel{(}\DIFdelend \DIFaddbegin \DIFadd{and }\DIFaddend M31 \DIFdelbegin \DIFdel{) GCSin the top (bottom) panel of Figure~\ref{fig:observations_FeHRgc}}\DIFdelend \DIFaddbegin \DIFadd{GCS}\DIFaddend . Both quantities are readily available in H96e10 (assuming
$R_{\odot}=8.0$~kpc), but the galactocentric radius of GCs in M31 is not available
in CR16. Therefore we follow \citet[][Sec.~4.1]{2019arXiv190111229W} to calculate
the projected radius $R_{\text{proj}}$ from the observed positions, adopting M31's
central position from the NASA Extragalactic 
Database\footnote{\url{https://ned.ipac.caltech.edu/}} $(\alpha_0, \, \delta_0) =
(0^{\text{h}}42^{\text{m}}44.35^{\text{s}}, \, +41^{\circ}16'08.63")$
and distance $D_{\text{M31}} = 780$~kpc \citep{2005MNRAS.356..979M,2012ApJ...758...11C}\DIFaddbegin \DIFadd{.
}\DIFaddend We calculate $R_{\text{GC}}$ as `average deprojected distance`
$R_{\text{GC}} = R_{\text{proj}} \times (4/\pi)$. 
\DIFaddbegin 

\DIFadd{The solid lines show the distributions using all available data (because the
sky coordinates are known for each GC), while the subset for which age-estimates
are available is indicated using dotted lines. The latter shows a narrower range
of radii than the full data set: it appears that age-estimates are neither
available for the innermost ($<1$~kpc) GCs, nor for those beyond roughly twenty
kpc (the halo GCs). We compare both distributions of the full data set (solid 
lines) and find that the Milky Way has more GCs in the range $1-4$~kpc than M31
(when accounting for the larger number of total GCs in M31). Interestingly, both
distributions show a similar trend for $R_{\text{GC}} > 4$~kpc and host a 
subpopulation of halo GCs. However, both radial distributions are not 
statistically consistent due to substantial differences at intermediate radii.
}

\begin{figure}
    \includegraphics[width=\columnwidth]{{MW_M31_Age_Histogram-trim}.png}
    \includegraphics[width=\columnwidth]{{MW_M31_Rgc_Histogram-trim}.png}
    \caption{
        \emph{\DIFaddFL{Top}}\DIFaddFL{: Age distribution of 55 GCs in the MW
        \mbox{%DIFAUXCMD
\citep[data from][]{2013ApJ...775..134V} }\hspace{0pt}%DIFAUXCMD
and 88 GCs
        in M31 \mbox{%DIFAUXCMD
\citep[data from][]{2016ApJ...824...42C}}\hspace{0pt}%DIFAUXCMD
.
        }\emph{\DIFaddFL{Bottom:}} \DIFaddFL{Distribution of galactocentric radius in the MW and M31.
        The dotted lines show the subset of data that also have age measurements
        (i.e. the same sample as used in the top panel).
        }\label{fig:MW-M31-age-Rgc}
    }
\end{figure}

\subsection{Total GC mass in metallicity-radial space} \label{sec:observations_FeHRgc}
\DIFadd{We show the two-dimensional mass-weighted metallicity-radial distribution of the
MW (M31) GCS in the top (bottom) panel of Figure~\ref{fig:observations_FeHRgc}. 
}\DIFaddend The observations indicate that
no GCs with high metallicities are to be expected at large radii (the three bins
in the upper right corner, both for MW and M31), and relatively few GCs at large
radii in general ($R_{\text{GC}} > 30$~kpc; right column: $11$ GCs or 7.3\% in
the MW and $17$ or 4.6\% in M31). Moreover, the M31 hosts more metal-rich 
([Fe/H]$ > -1$) GCs in each radial bin in comparison to the MW GCSs than what
would be expected when accounting for the fact that Andromeda hosts a larger GC 
system than the MW. Finally, given that both marginalized (i.e. the metallicity 
and radial) distributions are not statistically consistent, we find that the 
two-dimensional distributions are also not consistent. More generally, the GC
systems of the Milky Way and that of Andromeda both differ significantly. We 
compare these observations to the Auriga simulations later on in 
Sec.~\ref{sec:results_FeHRgc}.

\begin{figure}
    \includegraphics[width=\columnwidth]
        {{MW_RgcFeH_HistogramMassWeighted_Harris1996ed2010data}.png}
    \includegraphics[width=\columnwidth]
        {{M31_RgcFeH_HistogramMassWeighted_CaldwellRomanowsky2016data}.png}
    \caption{
        \emph{Top}: Mass-weighted $R_{\text{GC}}$-[Fe/H] distribution of
        151 GCs in the MW \citep[data from][2010 ed.]{1996AJ....112.1487H}, which
        is 98.6 (92.2) \% of the total mass (clusters) of the MW GCS in the Harris
        catalog. \emph{Bottom}: Same for M31, showing 366 GCs and 88.4 (83.9) \%
        of the total mass (clusters) in CR16 \citep[data from][]{2016ApJ...824...42C}.
        Note that the range of the colourmap differs in both figures.
        \label{fig:observations_FeHRgc}
    }
\end{figure}


\section{Results}
\label{sec:results}
We define GC candidates in the Auriga simulations as all star particles older
than $10$~Gyr based on the age distribution of the MW GCS (\DIFdelbegin \DIFdel{Figure~\ref{fig:MW-M31-age}}\DIFdelend \DIFaddbegin \DIFadd{top panel of 
Figure~\ref{fig:MW-M31-age-Rgc}}\DIFaddend ), and following \DIFdelbegin \DIFdel{the analysis of }\DIFdelend \citet{2017MNRAS.465.3622R}.

\DIFdelbegin \DIFdel{Trough out }\DIFdelend \DIFaddbegin \DIFadd{Troughout }\DIFaddend our analysis we compare the distributions of three subsets of star
particles: \emph{old stars} (age $>10$~Gyr, or GC candidates), \emph{old \DIFdelbegin \DIFdel{insitu}\DIFdelend \DIFaddbegin \DIFadd{in~situ}\DIFaddend }
stars (defined as those bound to the most-massive halo/subhalo in the first
snapshot that the particle was recorded), and \emph{old accreted} star particles
(those that have formed \DIFdelbegin \DIFdel{ex-situ }\DIFdelend \DIFaddbegin {\it \DIFadd{ex~situ}} \DIFaddend and are bound to the most-massive halo/subhalo
at $z=0$). For comparison we also include the results for \emph{all stars} (when
no additional selection criterion is applied to the star particles). We consider
the metallicity distribution in Sec.~\ref{sec:results_FeH}, the distribution of
galactocentric radii in Sec.~\ref{sec:results_Rgc}, and the combination of both
in Sec.~\ref{sec:results_FeHRgc}.


\subsection{Metallicity distribution}
\label{sec:results_FeH}
We investigate whether the star formation model implemented in Auriga produces
metallicity distributions consistent with the MW and M31 GC systems, and whether
the \DIFdelbegin \DIFdel{subgrid generates }\DIFdelend \DIFaddbegin \DIFadd{simulations generate }\DIFaddend sufficient total mass at metallicities typical for the
MW and M31 GCS's. To visually inspect the former we show the normalized metallicity 
distribution of three specific Auriga galaxies in Figure~\ref{fig:Au4-10and21_FeH}
in comparison to the MW and M31 [Fe/H] distributions.

\begin{figure*}
    \includegraphics[width=\columnwidth]{{Au4-10_FeH_cleaner-trim}.png}
    \includegraphics[width=\columnwidth]{{Au4-4_and_Au4_21_FeH-trim}.png}
    \caption{
        \emph{Left:} Metallicity distribution of Au4-10 (bottom panel). We show
        the GG candicates in green. We split the GC candidates into two
        subpopulations, those that have formed \DIFdelbeginFL \DIFdelFL{insitu }\DIFdelendFL \DIFaddbeginFL {\it \DIFaddFL{in~situ}} \DIFaddendFL (blue), and those that have
        been accreted (red). The dotted green line shows all star particles.
        The solid purple (magenta) line in the top panel shows the GC system
        of the MW (M31). \emph{Right:} Au4-4 (bottom), and Au4-21 (top).
        \label{fig:Au4-10and21_FeH}
    }
\end{figure*}
We select \mbox{Au4-10}\footnote{The nomenclature is `Au' for Auriga, followed by the
resolution level (4) and halo number (10, indicating which set of initial
conditions was used the run).}, \mbox{Au4-21}, and \mbox{Au4-4} to show the
distributions of three individual simulation runs that we consider representative
for specific behaviour, and to highlight that different runs could give rise to 
different distributions. We plot the age-selected GC candidates in green, those
that are accreted in red, and the \DIFdelbegin \DIFdel{insitu }\DIFdelend \DIFaddbegin {\it \DIFadd{in~situ}} \DIFaddend subpopulation in blue. The top 
half of the left figure shows the MW (M31) GC system in purple (magenta), where 
we overplot a double Gaussian for the MW GCS (the purple dashed lines) by 
adopting literature values of the mean $\mu$ and standard deviation $\sigma$ of 
the metal-rich and metal-poor subpopulations \citep[][p. 38]{Harris2001}.

We find that the age cut, on average, lowers the mean metallicity from $0.0$ to
$-0.6$. Furthermore, the old accreted stars generally have lower mean metallicities
than the old \DIFdelbegin \DIFdel{insitu }\DIFdelend \DIFaddbegin {\it \DIFadd{in~situ}} \DIFaddend stars with differences of roughly ${\sim}0.3$ for the majority 
of the simulation runs. This behaviour can be seen in \mbox{Au4-21}, while a 
slightly larger difference of ${\sim}0.5$ dex is seen for \mbox{Au4-10} (as plotted), 
\mbox{Au4-16}, \mbox{Au4-17}, \mbox{Au4-18} and \mbox{Au4-22}. However, this trend 
is reversed for \mbox{Au4-1} and \mbox{Au4-4} for which the old \DIFdelbegin \DIFdel{insitu }\DIFdelend \DIFaddbegin {\it \DIFadd{in~situ}} \DIFaddend population 
has a lower mean metallicity instead. We caution that \mbox{Au4-1} is undergoing
a major merger at redshift zero and note that we find $\mu = -1.51 \, ({-0.74})$ 
for the old \DIFdelbegin \DIFdel{insitu }\DIFdelend \DIFaddbegin {\it \DIFadd{in~situ}} \DIFaddend (accreted) although the former consists of only $1019$ 
particles ($1.3$\% of all GC candidates, and with a total mass of 5e7 M\Sun). 
For \mbox{Au4-4}, ${10.8}$\% of the GC candidates is classified as \DIFdelbegin \DIFdel{insitu }\DIFdelend \DIFaddbegin {\it \DIFadd{in~situ}} \DIFaddend (compared 
to \DIFdelbegin \DIFdel{insitu }\DIFdelend \DIFaddbegin {\it \DIFadd{in~situ}} \DIFaddend fractions of ${40-80}$\% for other haloes). \DIFdelbegin \DIFdel{Overall }\DIFdelend \DIFaddbegin \DIFadd{After inspection of
the same figure for every one of the thirty Auriga L4 haloes }\DIFaddend we find that the 
simulations produce (sub)populations of GC candidates that are more metal-rich
than the MW and M31 GC systems. Moreover, none of the simulations has a 
population of GC candidates with a bimodal metallicity distribution (the green
curves).

\begin{figure}
    \includegraphics[width=\columnwidth]{{FeH_mu_sigma-trim}.png}
    \caption{
        First vs second central moment of the Auriga L4 metallicity distributions.
        Each cross (for a given colour) represents one simulation. The green (blue)
        [red] crosses show the values calculated using the old (old \DIFdelbeginFL \DIFdelFL{insitu}\DIFdelendFL \DIFaddbeginFL {\it \DIFaddFL{in~situ}}\DIFaddendFL )
        [old accreted] star particles. Green triangles indicate that all stars
        were used. The purple (magenta) cross denotes our calculation using all
        MW (M31) observations (which would be appropriate for a unimodal
        distribution). The black solid (open) dots indicate the literature values
        of a bimodal Gaussian fit to the data \citep[values from][p. 38]{1998gcs..book.....A},
        showing the metal-rich (metal-poor) component of the MW.
        \label{fig:FeH_mu_sigma}
    }
\end{figure}

We show the mean metallicity and standard deviation of all thirty Auriga L4 haloes
in Figure~\ref{fig:FeH_mu_sigma} to show the behaviour for the full set of Auriga
galaxies. The green crosses are to be compared to the purple (magenta) cross, 
which shows the mean value of all MW (M31) GCs. In addition, we show the metal-rich 
(metal-poor) population of the MW GCS using a solid (open) dot. We caution that
these literature values result from individual Gaussian fits to sub sets of the 
observational data cut at [Fe/H]~$= -1$. We do not include corresponding data 
points calculated using such an artificial cut for the simulated star particles.
Having said that, we do find it interesting that the mean metallicities of the 
old \DIFdelbegin \DIFdel{insitu }\DIFdelend \DIFaddbegin {\it \DIFadd{in~situ}} \DIFaddend populations appear roughly consistent with that of the metal-rich 
population of the MW although the simulations show larger dispersions. The latter 
could simply be caused by the hard separation of the \DIFaddbegin \DIFadd{MW }\DIFaddend data into two groups, 
which means the range is smaller and the resulting dispersion \DIFdelbegin \DIFdel{could be }\DIFdelend lower. With 
regard to the M31 GCS, we are uncertain whether definitive consensus is reached
\DIFaddbegin \DIFadd{in the literature }\DIFaddend concerning uni- bi- or trimodality in the [Fe/H] distribution, 
but CR16 argues that the data, after removal of younger objects due to improved 
age classification, hints at three populations separated at [Fe/H]~$>-0.4$ and 
[Fe/H]~$ < 1.5$. None of the Auriga sub sets has a mean value that offers much 
hope that the simulated distribution is consistent with the lowest metallicity 
group in the M31 GCS. This is also true for the metal-poor population of the 
MW with a mean value of $-1.6$. The main take-away from this plot, again, is 
that all Auriga L4 galaxies have metallicity distributions with (much) larger
mean values than what is observed for the MW and M31 GCS, and that we 
\DIFdelbegin \DIFdel{systematiclly }\DIFdelend \DIFaddbegin \DIFadd{systematically }\DIFaddend find lower mean metallicities for the accreted GC candidates
than for those that have formed \DIFdelbegin \DIFdel{in situ}\DIFdelend \DIFaddbegin {\it \DIFadd{in~situ}}\DIFaddend .

\DIFdelbegin \DIFdel{Finally, we test the null hypothesis that MW (M31) and the GC candidates (all, insitu, and
accreted) are drawn from the same underlying distribution by calculating the KS
test statistic (i.e. six KS tests per simulation). We reject $H_0$ at the 100.00~\%
confidence level for virtually all subsets of all simulations, both for MW and M31,
except when comparing the old accreted stars of \mbox{Au4-10} to the MW GCS. In this
case we still reject $H_0$ with, but with $\leq 98.68$~\% confidence. When comparing
to the M31 GCS, the `best' matches are the old accreted subpopulations of \mbox{Au4-13},
\mbox{Au4-15}, and \mbox{Au4-17} ($H_0$ rejected with $\leq$ 99.82~\% confidence, 99.95~\%,
and 98.32~\%). Here we find that the cumulative distributions are more or less
similar up to }%DIFDELCMD < [%%%
\DIFdel{Fe/H}%DIFDELCMD < ]%%%
\DIFdel{~$-0.5$, above which the simulations yield higher number
counts which drives the KS test statistic over the critical values. This affirms
our finding that the simulations produce GC candidates that are more metal-rich
than the MW (M31) GCS. The main reasons that $H_0$ is rejected with such
high confidence are under-production of old stars with }%DIFDELCMD < [%%%
\DIFdel{Fe/H}%DIFDELCMD < ]%%%
\DIFdel{~$\leq$~-1.5,
and over-production of stars with }%DIFDELCMD < [%%%
\DIFdel{Fe/H}%DIFDELCMD < ]%%%
\DIFdel{$\geq$~-0.5.
}\DIFdelend %DIF >  Finally, we test the null hypothesis that MW (M31) and the GC candidates (all,
%DIF >  {\it in~situ}, and accreted) are drawn from the same underlying distribution 
%DIF >  by calculating the KS test statistic (i.e. six KS tests per simulation). We 
%DIF >  reject $H_0$ at the 100.00~\% confidence level for virtually all subsets of all 
%DIF >  simulations, both for MW and M31, except when comparing the old accreted stars 
%DIF >  of \mbox{Au4-10} to the MW GCS. In this case we still reject $H_0$ with, but 
%DIF >  with $\leq 98.68$~\% confidence. When comparing to the M31 GCS, the `best' 
%DIF >  matches are the old accreted subpopulations of \mbox{Au4-13}, \mbox{Au4-15}, and 
%DIF >  \mbox{Au4-17} ($H_0$ rejected with $\leq$ 99.82~\% confidence, 99.95~\%,
%DIF >  and 98.32~\%). Here we find that the cumulative distributions are more or less
%DIF >  similar up to [Fe/H]~$-0.5$, above which the simulations yield higher number
%DIF >  counts which drives the KS test statistic over the critical values. This affirms
%DIF >  our finding that the simulations produce GC candidates that are more metal-rich
%DIF >  than the MW (M31) GCS. The main reasons that $H_0$ is rejected with such
%DIF >  high confidence are under-production of old stars with [Fe/H]~$\leq$~-1.5,
%DIF >  and over-production of stars with [Fe/H]$\geq$~-0.5.



\begin{figure}
    \includegraphics[width=\columnwidth]{{logMFeH_withRatios-trim}.png}
    \caption{
        Mass-weighted metallicity distribution of star particles in the Auriga
        simulations. We show the median value of all Auriga haloes for all
        stars (green dotted) and globular cluster candidates (i.e. stars with
        age $>$~10~Gyr; green solid). The latter sub set is further split up
        between stars that formed \DIFdelbeginFL \DIFdelFL{in-situ }\DIFdelendFL \DIFaddbeginFL {\it \DIFaddFL{in~situ}} \DIFaddendFL (blue solid), and those that were accreted
        (red solid). Shaded regions indicate the $1\sigma$ interval. The MW (M31)
        GCS is shown in purple (magenta). The middle (bottom) panel shows the
        ratio of the simulated mass to the mass in the MW (M31) GCS.
        \label{fig:FeH}
    }
\end{figure}
We now turn to the second goal, which is to see whether sufficient total mass in 
GC candidates \DIFaddbegin \DIFadd{is }\DIFaddend produced by the Auriga simulations. To answer this question we plot
a mass-weighted metallicity distribution in Figure~\ref{fig:FeH}. We show the 
median (coloured lines) for all thirty Auriga L4 haloes with the $1\sigma$ 
interval around it (shaded regions, which shows the scatter between runs that 
have different initial conditions, thus, have unique merger histories). We chose 
to aggregate the data to indicate general trends that we find when the GC candidates 
are split up according to birth location, rather than selecting typical examples 
of individual (simulated) galaxies. Once again we notice that the peak metallicity 
shifts down from $0$ to $-0.6$ for old stars (green solid) compared to all stars 
(green dotted), and we learn that the mass at the peak lowers by roughly one dex. 
The mass budget of the old stars is dominated by the old \DIFdelbegin \DIFdel{insitu }\DIFdelend \DIFaddbegin {\it \DIFadd{in~situ}} \DIFaddend population (blue 
solid) below [Fe/H]~=~$-1$, and by the old accreted stars (red solid) above this 
value. We show the MW (M31) GCS in purple (magenta) and notice that the difference 
between the MW and M31 distributions is substantially larger than the scatter
between different Auriga galaxies, particularly around [Fe/H]~$=-1$. In addition, 
Andromeda does host (a hand full) GCs with [Fe/H]~$< -2.5$ as well as GCs with 
[Fe/H]~$ > 0$ while the Milky Way does not.

We show the ratio of the  simulated to the observed profiles in the middle (bottom) 
panel. This mass excess can be thought of the `mass budget' that the Auriga GC 
candidates can `afford to lose' due to a combination of smaller than unity bound
cluster formation efficiencies combined with a Hubble time of dynamical evolution,
while still producing sufficient mass at the right metallicities. In particular,
the cluster formation efficiency would have to \DIFdelbegin \DIFdel{linearly }\DIFdelend decrease with decreasing 
metallicity for Auriga GC candidates to produce a population of GC candidates 
that is consistent with the MW. For the GC candidates in M31 we find a constant 
mass excess up to $-0.9$, above which the simulations produce a higher mass excess 
with increasing metallicity. If dynamical evolution is not expected to more 
efficiently disrupt GCs of higher metallicity, then we would find that the 
efficiency to form bound star clusters would have to decrease with increasing 
metallicity.


\subsection{Radial distribution}
\label{sec:results_Rgc}

\begin{figure}
    \DIFdelbeginFL %DIFDELCMD < \includegraphics[width=\columnwidth]{{Rgc_mu_sigma-trim}.png}
%DIFDELCMD <     %%%
\DIFdelendFL \DIFaddbeginFL \includegraphics[width=\columnwidth]{{LogRgc_mu_sigma-trim}.png}
    \DIFaddendFL \caption{
        Mean and standard deviation of the \DIFaddbeginFL \DIFaddFL{logarithm of }\DIFaddendFL radial distribution 
        of star particles in each of the thirty Auriga L4 haloes compared to
        the MW (M31) GCS shown in purple (magenta).
        \label{fig:Rgc_mu_sigma}
    }
\end{figure}
We repeat the analysis of the previous subsection for the distribution of
galactocentric radius instead of metallicity. We look for general trends present
in all Auriga L4 galaxies. Figure~\ref{fig:Rgc_mu_sigma} shows the mean and 
standard deviation of \DIFdelbegin \DIFdel{the radial distribution }\DIFdelend \DIFaddbegin \DIFadd{$\log_{10}(R_{\text{GC}})$ }\DIFaddend of star particles in all Auriga
L4 simulations. \DIFaddbegin \DIFadd{The purple (magenta) cross again shows the mean of the MW (M31)
GCs (which are to be compared to the green crosses), but solid (open) dot now
shows our calculation of the mean value of the metal-rich (metal-poor) population 
as no literature values were available. We split the radii up into the 
metal-rich/metal-poor groups by taking a metallicity cut at }[\DIFadd{Fe/H}]\DIFadd{~=~-1 as is 
done in the literature.
}

\DIFaddend We notice that the old \DIFdelbegin \DIFdel{insitu }\DIFdelend \DIFaddbegin {\it \DIFadd{in~situ}} \DIFaddend populations are more centrally 
distributed, whereas the old accreted subsets have a larger radial extent. This 
is not surprising because the classification of \DIFdelbegin \DIFdel{in situ }\DIFdelend \DIFaddbegin {\it \DIFadd{in~situ}} \DIFaddend requires star particles 
to have formed within the virial radius, thus they could naturally be expected 
to end up at small galactocentric radii. Accreted star particles, on the other 
hand, have formed in another (sub)halo beyond the virial radius, thus, would 
first have to migrate inwards in order to populate the innermost radii. Moreover,
we \DIFdelbegin \DIFdel{notice that the dispersion increases with increasing mean value of the radial 
distribution as may be expected for a non-negative quantity. The purple (magenta)
cross indicates the values for the MW (}\DIFdelend \DIFaddbegin \DIFadd{find that the simulations have a wider dispersion in $\log(R_{\text{GC}}$ than
the MW GCS, while the dispersion of }\DIFaddend M31 \DIFdelbegin \DIFdel{) GCS, which are to be compared to the
green crosses. The }\DIFdelend \DIFaddbegin \DIFadd{seems to lie witin the range of dispersions
found in the Auriga galaxies. Furthermore, the mean of the }\DIFaddend MW GCS lies roughly 
within the range of values \DIFdelbegin \DIFdel{seen for the 
radial distributions }\DIFdelend produced by the Auriga simulations, whereas the mean 
of M31 is \DIFdelbegin \DIFdel{larger and }\DIFdelend \DIFaddbegin \DIFadd{somewhat larger and slightly }\DIFaddend closer to typical mean values of the 
accreted GC candidates in the Auriga simulations. The larger radial \DIFdelbegin \DIFdel{extend }\DIFdelend \DIFaddbegin \DIFadd{extent 
}\DIFaddend of the M31 GCS is generally believed to hint at a more accretion-dominated origin 
of the GCS, and may reflect a richer accretion history of the galaxy in comparison 
to the Milky Way.

\begin{figure}
    \includegraphics[width=\columnwidth]{{logMRgc_withRatios-trim}.png}
    \caption{
        Mass-weighted radial distribution of star particles in the Auriga
        simulations. We show the median value of all Auriga haloes for all
        stars (green dotted) and globular cluster candidates (i.e. stars with
        age $>$~10~Gyr; green solid). The latter sub set is further split up
        between stars that formed \DIFdelbeginFL \DIFdelFL{in-situ }\DIFdelendFL \DIFaddbeginFL {\it \DIFaddFL{in~situ}} \DIFaddendFL (blue solid), and those that 
        were accreted (red solid). Shaded regions indicate the $1\sigma$ 
        interval. The MW (M31) GCS is shown in purple (magenta). The middle 
        (bottom) panel shows the ratio of the simulated mass to the mass in the 
        MW (M31) GCS.
        \label{fig:Rgc}
    }
\end{figure}
Figure~\ref{fig:Rgc} shows the mass-weighted radial distribution of the
Auriga L4 haloes. We notice a subtle peak around $10$~kpc for all star
particles that is not present for the GC candidates, indicating that the
stellar disc is no longer present when applying the latter selection criterion.
Furthermore, we find that the dominant contribution to the total mass in GC
candidates changes from those formed \DIFdelbegin \DIFdel{insitu }\DIFdelend \DIFaddbegin {\it \DIFadd{in~situ}} \DIFaddend to the accreted population around
$10$~kpc. Again we show the mass excess of the simulations compared to the Milky
Way and Andromeda GCS. We find a decreasing mass excess with increasing radius
in the range $0.2$ to ${\sim}5$ kpc, followed by an increase attributed
to the accreted subpopulation. For the Milky Way we notice that significantly
fewer GCs are found beyond $40$~kpc than for M31, and that accreted GC candidates
contribute mostly to the Auriga GCS at these radii. We further investigate a 
breakdown of the total mass in Auriga GC candidates into bins of both metallicity 
and radius in the following section.

\subsection{Total mass in metallicity-radial space}
\label{sec:results_FeHRgc}

We investigate whether the Auriga simulations still produce sufficient mass when
the GC candidates are two-dimensionally binned in [Fe/H] and $R_\text{GC}$. First
we sum the total simulated mass in each bin (for an individual Auriga simulation),
then we calculate the median over all thirty Auriga L4 haloes (in each bin, see
Figure~\ref{fig:Au-FeHRgc} which can be compared to
Figure~\ref{fig:observations_FeHRgc}). Finally, we divide these value by the total
mass in the MW (M31) GCS to obtain the median mass excess produced by the star
formation model implemented in the Auriga simulations. See the top (bottom) panel
of Figure~\ref{fig:Au-FeHRgc-ratio} for mass excess with respect to the MW (M31) GCS.

\begin{figure}
    \includegraphics[width=0.49\textwidth]
        {{Au4-median_RgcFeH_HistogramMassWeighted_iold-trim}.png}
    \caption{
        Mass-weighted [Fe/H]-$R_{\text{GC}}$ distribution of all thirty Auriga
        L4 haloes. Here we consider only the GC candidates (age $>$~10~Gyr) stars
        in and color-code by the \textbf{median} (values also shown in each bin).
        The numbers in parenthesis show how many star particles fall within the
        bin. Note that the range of the colourmap again differs (for improved
        contrast within the plot).
        \label{fig:Au-FeHRgc}
    }
\end{figure}

We notice in Figure~\ref{fig:Au-FeHRgc} that the Auriga simulations, on average,
produce fewer GCs in the lowest metallicity bins ($-2.5$ to $-1.5$) than at higher 
metallicities in the three innermost radial bins. In addition, Auriga produces 
fewer high-metallicity GCs at large radii, similar to the MW and M31 GC systems. 
However, the three bins in the upper right corner are populated in the Auriga
simulations, in contrast to the observations. The top panel of 
Figure~\ref{fig:Au-FeHRgc-ratio} shows that the mass excess for Auriga GCs within
the Solar radius decreases with decreasing metallicity. This means that either
the formation efficiency should increase with decreasing metallicity, or that 
the lowest-metallicity GCs in the Auriga simulations are less strongly affected
by disruptive processes, if the star formation histories in the Auriga simulations
are to give rise to a GC system that can be similar to the GCS that is observed
in the Milky Way. Moreover, either a lower formation efficiency is required at
large radii, or stronger disruption. Disruption of star clusters is expected to
be more efficient at smaller radii due to higher tidal forces in the innermost
region for GCs on orbits that pass through the disk, or with pericenters close 
to the bar or bulge. Compared to Andromeda, the mass excess (bottom panel of 
Figure~\ref{fig:Au-FeHRgc-ratio}) shows a less clear \DIFdelbegin \DIFdel{trent }\DIFdelend \DIFaddbegin \DIFadd{trend }\DIFaddend than for the Milky Way\DIFaddbegin \DIFadd{,
which suggests that both galaxies have diverse GC formation histories}\DIFaddend .
Most bins show mass surplus factors of few tens, meaning that high formation 
efficiencies would be required in addition to little disruption for the Auriga 
simulations to yield a GC system consistent with that of the Andromeda galaxy.

\begin{figure}
    \includegraphics[width=0.49\textwidth]
        {{Au4-median_RgcFeH_HistogramMassWeighted_MW_iold-trim}.png}
    \includegraphics[width=0.49\textwidth]
        {{Au4-median_RgcFeH_HistogramMassWeighted_M31_iold-trim}.png}
    \caption{
        The top (bottom) panel shows the logarithm of the ratio of simulated mass
        to mass in the MW (M31) GCS, i.e. the logarithm of the mass excess. The
        color-coded values are also shown in each bin. Note that the three bins
        in the upper right corner are left blank because the observations have
        zero mass there, and that the numbers indicate how much (median) mass is
        produced by the simulations in those bins.
        \label{fig:Au-FeHRgc-ratio}
    }
\end{figure}



\section{Discussion}
% ``The Auriga Stellar Haloes: Connecting stellar population properties with
% accretion and merging history'' \citet{2018arXiv180407798M}

\label{sec:discussion}
\subsection{Metallicity distribution}
\label{sec:discussion_FeH}
Globular clusters are ${\sim}0.5$ dex more metal-poor than spheroid stars observed 
at the same radius for almost all galaxies \citep{1991ARA&A..29..543H}. Our 
selection function (age cut) does lower the mean metallicity by $0.5$ dex, but
we still find that the metallicity distributions of GC candidates in the Auriga
simulations are more metal-rich than the MW (M31) GCS. Although the (old) star
particles represent single stellar populations with a mass (resolution) consistent 
with that of GCs, they are in fact statistical tracers of the stellar population 
of the galaxy as a whole. Therefore only a (small) fraction of the star particles 
may represent plausible formation sites of GCs, whereas the majority represents 
(halo) field stars - the disk component effectively falls outside our selection 
of star particles due to the age cut. \DIFaddbegin \DIFadd{It is implausible that all halo stars 
come from disrupted GCs because GC systems that survive down to redshift zero 
are generally few orders of magnitude less massive than galactic stellar haloes.
}\DIFaddend 

In general, the colour (metallicity) distribution of most GC systems is bimodal 
with typical separating value [Fe/H]$\sim -1$ \citep{1985ApJ...293..424Z,
1999AJ....118.1526G,2001AJ....121.2974L,2006ApJ...639...95P}. Indeed, 
\citet[][p. 234]{1998gcs..book.....A} and \citet[][p. 38]{Harris2001} find that
the MW GCS has a bimodality [Fe/H] distribution: the latter fit a double Gaussian 
which peaks at [Fe/H] = $-1.59$ (metal-poor) and $-0.51$ (metal-rich) and 
dispersions of $0.34$ and $0.23$. Observations of GCs in M31, however, may be
best split into three distinct metallicity groups, one found at small radii with 
[Fe/H] $>-0.4$, one intermediate [Fe/H] group, and a dominant (metal-poor) group 
with [Fe/H] $< -1.5$ \citep{2016ApJ...824...42C}. The numerical simulation of
\citet{2017MNRAS.465.3622R} does yield a bimodal metallicity distribution where
the metal-poor population is dominated by accreted star particles and the metal-rich
population by \DIFdelbegin \DIFdel{insitu }\DIFdelend \DIFaddbegin {\it \DIFadd{in~situ}} \DIFaddend stars. 

We find that none of the Auriga simulations produces a bimodal 
metallicity distribution for age-selected GC candidates. Interestingly, the 
cross-over point above (below) which the mass-weighted metallicity distribution
of GC candidates is dominated by those that have formed \DIFdelbegin \DIFdel{insitu }\DIFdelend \DIFaddbegin {\it \DIFadd{in~situ}} \DIFaddend (were
accreted) does coincide with the separation between the metal-rich and metal-poor
populations of GC systems at [Fe/H]~=~$-1$. Moreover, we do find that the mean
metallicity shifts when we split the GC candidates up according to birth location
(accreted or \DIFdelbegin \DIFdel{in-situ}\DIFdelend \DIFaddbegin {\it \DIFadd{in~situ}}\DIFaddend ). In particular, the mean values of the old 
\DIFdelbegin \DIFdel{insitu }\DIFdelend \DIFaddbegin {\it \DIFadd{in~situ}} \DIFaddend GC candidates are roughly consistent with the metal-rich MW GCs.
However, we also notice that a substantial number of simulations has similar 
mean values for all GC candidates, and the simulated \DIFdelbegin \DIFdel{insitu }\DIFdelend \DIFaddbegin {\it \DIFadd{in~situ}} \DIFaddend GC candidates 
have larger dispersions than the metal-rich GCs. Moreover, the offset between 
the mean of the metal-rich and metal-poor populations in the MW is $1$ dex, a
factor $2-3$ larger than the offset that we find between the \DIFdelbegin \DIFdel{insitu }\DIFdelend \DIFaddbegin {\it \DIFadd{in~situ}} \DIFaddend and 
accreted populations. The Auriga simulations therefore do not support the view 
that the metal-poor GCs could have formed in satellites that have later been 
accreted whereas the metal-rich GC subpopulation has formed \DIFdelbegin \DIFdel{insitu}\DIFdelend \DIFaddbegin {\it \DIFadd{in~situ}}\DIFaddend . With 
this statement, however, we do caution that \citet{2019MNRAS.486.3134K} suggests 
to reserve the `\DIFdelbegin \DIFdel{ex situ}\DIFdelend \DIFaddbegin {\it \DIFadd{ex~situ}}\DIFaddend ' classification for accretion after $z=2$, when
the central galaxy has formed and accretion unambigously contributes to the
radially extended halo GC population. In fact, our classification appears flawed
in Au4-1.

Finally, we note that \citet{2006ARA&A..44..193B} compares the number of metal-poor 
GCs to the stellar halo mass and find\footnote{the quantity $T$ is the number of 
GCs per $10^9$ M\Sun \, of galaxy stellar mass} $T^n_{\text{blue}} \sim 100$, 
while the number of metal-rich GCs compared to the bulge mass yields 
$T^n_{\text{red}} \sim 5$, and therefore conclude that the formation efficiency
of metal-poor GCs is twenty times higher than the metal-rich GCs with respect to 
field stars. We find that gradual linear increase in formation efficiency of GCs
with respect to field stars would be required with decreasing metallicity for
the GC candidates in the Auriga simulations to yield sufficient total mass to
be consistent with the Milky Way GCS.

\subsection{Radial distribution}
\label{sec:discussion_Rgc}
The M31 GCS has a factor 2-3 more GCs at large radii compared to the MW GCS,
which may indicate that M31 has a richer merger history than the MW 
\citep{2016ApJ...824...42C}. In the Auriga simulations we find that the GC 
candidates at radii larger than ${\sim}20$~kpc are indeed dominated by accreted
star particles. However, we select star particles that are bound to the main 
halo and main subhalo, which means that we include particles up to the virial 
radius $R_{200}$. The Auriga simulations have no problem populating the stellar 
halo up to the virial radius, even with our additional age cut. On the other hand, 
the MW and M31 have fewer GCs at large radii. We expect tidal disruption to be
less efficient at larger radii, thus, the formation efficiencies of the accreted 
GC candidates would have to be lower. Our classification as GC candidates could, 
for example, be improved by selecting a number of globular clusters based on the 
$M_{vir}-N_{\text{GC}}$ correlation found by \citet{2019arXiv190100900B}, and 
taking a virial mass below which none of the star particles is treated as GC candidate.

% Copy-pasted from \citet{2019MNRAS.486.5838R}
% ``Out of the cluster formation models considered, the `no formation physics' model,
% in which both the CFE and the ICMF are fixed throughout cosmic time, is
% approximately equivalent to those studies  that  use  `particle-tagging'
% techniques  to  identify  GCs  in cosmological simulations (e.g. Tonini 2013,
% Renaud et al. 2017). We find that this cluster formation model continues to form GCs
% at a vigorous rate until the present time, which implies it overproduces
% the total GC mass in the Milky Way by a factor
% 5.5 relative to the mass formed in the fiducial model. Likewise, the continued
% formation  of  GCs  in  this  model  predicts  the  on-going  formation
% of massive (M $>10^5$ M\Sun) clusters with [Fe/H]$ <-0.5$ should be observed in
% Milky Way-mass galaxies at $z=0$. For these reasons, an environmentally independent
% cluster formation description is not compatible with observations.''



\section{Summary and conclusions}
\label{sec:conclusions}

We investigate GC candidates in the Auriga simulations and draw the following
conclusions.

\begin{itemize}
    \item The star formation model implemented in the Auriga simulations produces
    metallicity distributions that are more metal-rich \DIFaddbegin \DIFadd{than }\DIFaddend the Milky Way and M31
    globular cluster systems.

    \item We reject the null hypothesis $H_0$ for any one of the Auriga simulations 
    that the GC candidates and the MW GCs are drawn from the same underlying 
    distribution\DIFdelbegin \DIFdel{with $\geq 98.68$~\% confidence}\DIFdelend . This statement holds true when we select all old star particles 
    (GC candidates), but also when we select the old accreted or the old \DIFdelbegin \DIFdel{insitu }\DIFdelend \DIFaddbegin {\it \DIFadd{in~situ}} 
    \DIFaddend star particles. In addition, this is also the case when we test against the 
    M31 GCS instead of the MW GCS. \DIFaddbegin \DIFadd{Moreover, the difference between the MW and
    M31 is big compared to the scatter in GC candidate properties in the Auriga
    simulations.
}\DIFaddend 

    \item GC candidates in the Auriga simulations may be found out to $R_{200}$,
    given that our selection function of star particles selects all stars bound
    to the main subhalo in the main halo. The stellar mass is dominated by accreted 
    star particles at radii beyond $20$~kpc. The GCs in the MW and M31, on the 
    other hand, have a much smaller radial \DIFdelbegin \DIFdel{extend}\DIFdelend \DIFaddbegin \DIFadd{extent}\DIFaddend .

    \item The cluster formation efficiency would have to increase with decreasing
    metallicity for GC candidates in the Auriga simulations to be consistent
    with the Milky Way GC system, given that we expect dynamical evolution to 
    more strongly affect GC candidates at smaller radii. This \DIFdelbegin \DIFdel{trent }\DIFdelend \DIFaddbegin \DIFadd{trend }\DIFaddend of over-production
    of old star particles that are metal-rich and at large radii compared to 
    observed GCs is less clear for the M31 GCS\DIFdelbegin \DIFdel{.
}%DIFDELCMD < 

%DIFDELCMD <     \item %%%
\item%DIFAUXCMD
\DIFdel{The Auriga simulations are not consistent with the picture that all
    metal-rich GCs formed insitu and that all metal-poor GCs were accreted}\DIFdelend .

\end{itemize}


\section*{Acknowledgements}
TLRH acknowledges support from the International Max-Planck Research School (IMPRS) on Astrophysics.
\todo[inline]{Check Auriga boilerplate that we need to acknowledge}
RG and VS acknowledge support by the DFG Research Centre SFB-881 `The
Milky Way System' through project A1. This work has also been
supported by the European Research Council under ERC-StG grant
EXAGAL- 308037. Part of the simulations of this paper used the
SuperMUC system at the Leibniz Computing Centre, Garching,
under the project PR85JE of the Gauss Centre for Supercomputing.
This work used the DiRAC Data Centric system at Durham
University, operated by the Institute for Computational Cosmology
on behalf of the STFC DiRAC HPC Facility `www.dirac.ac.uk'.
This equipment was funded by BIS National E-infrastructure capital
grant ST/K00042X/1, STFC capital grant ST/H008519/1 and
STFC DiRAC Operations grant ST/K003267/1 and Durham University.
DiRAC is part of the UK National E-Infrastructure.


The analysis in this work was performed using the Python \citep{python}
programming language, the IPython \citep{2007CSE.....9c..21P} environment, 
the NumPy \citep{2011CSE....13b..22V}, SciPy \citep{scipy}, and Astropy 
\citep{2013A&A...558A..33A} packages. Plots were created with
Matplotlib \citep{2007CSE.....9...90H}.



%%%%%%%%%%%%%%%%%%%%%%%%%%%%%%%%%%%%%%%%%%%%%%%%%%

%%%%%%%%%%%%%%%%%%%% REFERENCES %%%%%%%%%%%%%%%%%%

% The best way to enter references is to use BibTeX:

\bibliographystyle{mnras}
\bibliography{AurigaGCS} 


% Alternatively you could enter them by hand, like this:
% This method is tedious and prone to error if you have lots of references
%\begin{thebibliography}{99}
%\bibitem[\protect\citeauthoryear{Author}{2012}]{Author2012}
%Author A.~N., 2013, Journal of Improbable Astronomy, 1, 1
%\bibitem[\protect\citeauthoryear{Others}{2013}]{Others2013}
%Others S., 2012, Journal of Interesting Stuff, 17, 198
%\end{thebibliography}

%%%%%%%%%%%%%%%%%%%%%%%%%%%%%%%%%%%%%%%%%%%%%%%%%%

%%%%%%%%%%%%%%%%% APPENDICES %%%%%%%%%%%%%%%%%%%%%
% \clearpage
% \appendix
% \section{Scatter between individual Auriga haloes, and numerical convergence}
% \label{sec:scatter-convergence}
% We check whether the properties of the Auriga globular cluster candidates
% are well converged between the three different resolution levels used for the
% Auriga simulations. Here we consider all three Auriga haloes for which simulation
% runs were performed at all three resolution levels: Au6, Au16, and Au24. Here
% we can investigate differences between individual haloes.
% 
% Figure~\ref{fig:logMFeH_res} shows the mass-weighted metallicity distribution,
% Figure~\ref{fig:logMRgc_res} shows the mass-weighted radial distribution, and
% Figure~\ref{fig:logMRgcFeH_res}
% 
% 
% 
% \begin{figure*}
%     \includegraphics[width=0.31\textwidth]{{logMFeH_Au6}.png}
%     \includegraphics[width=0.31\textwidth]{{logMFeH_Au16}.png}
%     \includegraphics[width=0.31\textwidth]{{logMFeH_Au24}.png}
%     \caption{Similar to Figure~\ref{fig:FeH}, but showing one individual Auriga
%         halo, where colours indicate the resolution level: L3 green, L4 orange,
%         and L5 blue. \emph{Left:} Auriga halo 6. \emph{Mid:} Auriga halo 16.
%         \emph{Right:} Auriga halo 24. For all three haloes we find marginal
%         increases in the mass normalization with increasing resolution level.
%         \label{fig:logMFeH_res}
%     }
% 
% \end{figure*}
% \begin{figure*}
%     \includegraphics[width=0.31\textwidth]{{logMRgc_Au6}.png}
%     \includegraphics[width=0.31\textwidth]{{logMRgc_Au16}.png}
%     \includegraphics[width=0.31\textwidth]{{logMRgc_Au24}.png}
%     \caption{Similar to Figure~\ref{fig:Rgc}, but showing one individual Auriga
%         halo, where colours indicate the resolution level: L3 green, L4 orange,
%         and L5 blue. \emph{Left:} Auriga halo 6. \emph{Mid:} Auriga halo 16.
%         \emph{Right:} Auriga halo 24. For all three haloes we find marginal
%         increases in the mass normalization with increasing resolution level.
%         \label{fig:logMRgc_res}
%     }
% 
% \end{figure*}
%%%%%%%%%%%%%%%%%%%%%%%%%%%%%%%%%%%%%%%%%%%%%%%%%%


% Don't change these lines
\bsp    % typesetting comment
\label{lastpage}
\end{document}

% End of mnras_template.tex
