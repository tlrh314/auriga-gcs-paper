Introduction: GC properties correlate with those of their host galaxies,
particularly the metallicities.

In the local universe, we see massive star clusters forming wherever
there are high star formation rate (SFR) surface densities
( Larsen \& Richtler 2000), providing a connection that suggests
that the properties of star cluster populations (age, metallicity,
mass) should scale quite closely with field stars formed in the same
events. The properties of globular cluster systems do in fact correlate
strongly with the properties of the field stars of their host
galaxies. The mean metallicities of GC systems have long been
known to scale with the metallicity of their host (van den Bergh
1975; Brodie \& Huchra 1991), and the mean metallicities of both
the metal-rich and metal-poor subpopulations also correlate with
the luminosity and mass of the host galaxy (Larsen et al. 2001;
Peng et al. 2006a, hereafter Paper IX, and references therein).
However, if GC systems directly followed the underlying field
light in every way, they might be less interesting. For instance,
although the metallicities of GC systems may track those of galaxies, 
they are consistently offset to lower values by 0.5–0.8 dex
in [Fe/H] (e.g., Jordan et al. 2004a; Lotz et al. 2004). Most conspicuously,
even the most massive and metal-rich galaxies have
GC systems dominated by metal-poor star clusters ([Fe/H] >= 1).
This suggests a disconnect between the formation of ``halo'' stellar
populations and the bulk of the galaxy.
\citep[from][]{2008ApJ...681..197P}
